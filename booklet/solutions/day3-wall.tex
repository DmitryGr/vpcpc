\bigheading{Wall}

\authors{ACM Regionals Central Europe 2004}{Błażej Magnowski \& Marek Sommer}{Marek Sommer}

For the sake of simplicity let's create an additional point on the wall
	in which the robot starts repairment, and set its cost coefficients to $0$ $0$ (so that it won't change the result).
Let's name all the points (including the false one) in order they appear on the wall: $W_1$, $W_2$, $W_3$, $\ldots$, $W_n$.
Let's say that the robot starts in point $W_k$.

First of all, we can see that the first cost coefficient $C_i$ does not really affect the order of visiting
	the points on the wall.
No matter what that order would be, these coefficients are constant and always should be added to the final cost.
We must count the sum of $C_i$ coefficients, add them to the result, and then we can ignore them to the rest of our algorithm,
	assuming, that from now on there is only one cost coefficient -- $D_i$.

Let's compute the prefix and suffix sums of points coefficients, which will help us later.
To be more precise, let's construct two arrays $P$ and $S$ such that for each $i \in \{1, 2, \ldots, n\}$:
\[ P[i] = \sum_{j = 1}^i D_j \]
\[ S[i] = \sum_{j = i}^n D_j \]
Also, let's assume that for $i \not \in \{1, 2, \ldots, n\}$, $P[i] = S[i] = 0$
	(that will help us deal with special cases when $i = 0$ or $i = n+1$).

After applying above three preparation steps, we can proceed to the algorithm.
We will use dynamic programming.
Let's compute the answers for some subsets of our points, which will lead us to the answer for all the points.
Let's create an array $dp[i][j][f]$ with the following meaning:
\begin{itemize}
	\item If $f = 0$, $dp[i][j][f]$ is equal to the minimal cost of repairing points
		$W_1, W_2, \ldots, W_i, W_j, W_{j+1}, \ldots, W_n$ with the robot starting at position $W_i$.
	\item If $f = 1$, $dp[i][j][f]$ is equal to the minimal cost of repairing points
		$W_1, W_2, \ldots, W_i, W_j, W_{j+1}, \ldots, W_n$ with the robot starting at position $W_j$.
\end{itemize}
In short, $dp[i][j][f]$ is equal to the cost of repairing points from only a prefix $1..n$ and a suffix $j..n$,
	starting from position $i$ or $j$ (which depends on $f$ -- if it is $0$ or $1$).
Not all the fields in this array will be used.
Only those which meet conditions: $0 \le i < j \le n+1$, $f \in \{0, 1\}$.
For simplicity, let's say that prefix $1..0$ is empty and suffix $(n+1)..n$ is also empty.

If $i=0$ and $j=n+1$, there are no points to repair, so:
\[ dp[0][n + 1][0] = dp[0][n + 1][1] = 0 \]

In some more general situations (let's assume that $i \neq 0$ and $j \neq n+1$):
\[ dp[0][j][0] = \textrm{(let's leave it undefined)} \]
\[ dp[0][j][1] = |X_j - X_{j+1}| \cdot S[j+1] + dp[0][j+1][1] \]
\[ dp[i][n+1][0] = |X_{i-1} - X_i| \cdot P[i-1] + dp[i-1][n+1][0] \]
\[ dp[i][n+1][1] = \textrm{(let's leave it undefined)} \]

In the most general situations (when $i \neq 0$ and $j \neq n+1$):
\[
	dp[i][j][0] = \min \left( |X_{i-1} - X_i| \cdot \left( P[i-1] + S[j] \right) + dp[i-1][j][0],
		|X_i - X_j| \cdot \left( P[i-1] + S[j] \right) + dp[i-1][j][1] \right)
\]
\[
	dp[i][j][1] = \min \left( |X_j - X_{j+1}| \cdot \left( P[i] + S[j+1] \right) + dp[i][j+1][1],
		|X_i - X_j| \cdot \left( P[i] + S[j+1] \right) + dp[i][j+1][0] \right)
\]

Two equations above jest check two possibilities to go left or to go right and they choose the one, with smaller cost.

All theses equations together allow to compute any value $dp[i][j][f]$.
The only difficulty left, is to process them in a correct order
	(to assure that every value on the right side of equation, is computed before the value on the left side).
We can notice that every equation uses answers for a smaller number of points,
	so we can process the dynamic states in order from the one with $0$ points to the ones with $n$ points
	(the order of states within the same number of points does not matter).

After computing answer for every state, the answer to the problem will
	be held under $dp[k][k + 1][0]$ or $dp[k - 1][k][1]$, where $k$ is the initial position of the robot
		(the false point we made at the beginning).
