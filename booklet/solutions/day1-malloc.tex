\bigheading{Dynamic memory allocation}

\authors{Peter `Bob' Fulla}{Peter `Bob' Fulla}{Peter `Bob' Fulla}

It is quite obvious how to process a query of either type in time $O(n)$, but
such a solution would certainly fail to achieve the perfect score for this task.
We will describe two approaches that lead to algorithms with time complexity
$O(\log n)$ per query.


\heading{Solution based on segment trees}

We build a segment tree on top of the array of bytes. In each node, we store the
following information about the segment represented by the node:
\begin{itemize}
\item the number of allocated bytes in the segment,
\item the length of the longest continuous block of free bytes in the segment,
\item the number of consecutive free bytes at the beginning of the segment,
\item the number of consecutive free bytes at the end of the segment.
\end{itemize}
Note that we can easily compute these values for a node from the values stored
in its two children.

\bigskip

To process an allocation query, we first check whether the longest block of free
bytes in the entire array would be long enough to satisfy the query. If it is
too short, we return $-1$. Otherwise, we need to find the leftmost block of
length at least $\ell$. This can be done using recursion: If the longest block
in the left-hand half of the array is of length $\ell$ or more, we continue
looking for the answer there. If this is not the case, we check whether we
obtain a sufficient block of free bytes by concatenating the block at the end of
the left-hand half of the array with the block at the beginning of the right-hand
half. If this option also fails, the sought block must lie completely in the
right-hand half of the array.

After we find the position of the block to allocate, we need to change the state
of bytes in it and make necessary updates to the values in the segment tree.
However, the number of bytes allocated in a query may reach $\Theta(n)$, so we
must employ a technique called lazy propagation: Whenever we want to allocate
all bytes in the segment corresponding to a node of the segment tree, we simply
mark the node with a flag meaning ``all bytes below are allocated'', update the
values stored in it, and do not descend to its subtree at all. When processing
a subsequent query, we may enter a marked node. Only at that time we unmark the
node and propagate the flag to its two children.

Freeing a block of bytes is very similar to allocating, we just need a different
flag meaning ``all bytes below are free''. To compute the number of allocated
bytes before a query, we simply add up the corresponding values stored in nodes
of the segment tree.

The lazy propagation technique guarantees that we visit at most $2 \log n$ nodes
when updating the tree. Therefore, the time complexity is $O(\log n)$ per query.


\heading{Solution based on self-balancing binary search trees}

This approach requires us to implement a customized self-balancing binary search
tree, for example a treap. The nodes of the tree will represent maximal blocks
of consecutive free bytes and they will be ordered by their positions in the
array. In addition, we need to maintain the following values in every node:
\begin{itemize}
\item the sum of lengths of blocks in the node's subtree,
\item the maximum of these lengths.
\end{itemize}

\bigskip

Using the precomputed maxima of subtrees, we can easily find the leftmost block
with length $\ell$ or more in time $O(\log n)$. To completely process an
allocation query, we just need to update the found block (or remove it, if its
length equals exactly $\ell$).

Dealing with freeing queries is now a bit more complicated, as they are not
necessarily aligned with blocks of free bytes. We split the tree before the
first block and after the last block contained by the freed interval, update the
adjacent blocks if necessary, and merge the parts of the tree together. These
operations can be implemented with time complexity $O(\log n)$.
