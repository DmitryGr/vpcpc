%%
\bigheading{Bus lines}
\authors{Filip Hlásek}{Filip Hlásek}{Filip Hlásek}

We are given an undirected connected graph with $N$ vertices and $N - 1$ edges,
therefore a tree.

Let's try to count the number of paths that go through a vertex $x$.
Imagine that the tree is rooted at $x$. It has $k$ children, each of them is
a root of the subtree of size $s_1, s_2, \dots, s_k$, respectively.
The paths that go through $x$ have either $x$ as one of the endpoints or have
the endpoints in two different subtrees of $x$'s children.
The total count equals:
\begin{align*}
numPaths(x) &= 2(N - 1) + 2\left(s_1s_2 + s_1s_3 + \dots s_{k - 1}s_k\right) \cr
            &= 2(N - 1) + 2\sum_{1\le i < j \le k} s_i s_j \cr
            &= 2(N - 1) + \left(\sum_{i = 1}^k s_i\right)^2 - \sum_{i = 1}^k s_i^2 \cr
            &= 2(N - 1) + (N - 1)^2 - \sum_{i = 1}^k s_i^2 \cr
            &= N^2 - 1 - \sum_{i = 1}^k s_i^2 \cr
\end{align*}

We can obtain the same result by considering all $N(N - 1)$ paths from which
$s_1(s_1 - 1) + \dots + s_k(s_k - 1)$ don't go through the vertex $x$.

So far we have counted the number of paths that go through some chosen vertex $x$.
Now we have to do it for every vertex in the tree, but with better time complexity than $O(N^2)$.

Let's root the tree at an arbitrary vertex. We will run a DFS algorithm from the root.
For each vertex $x$ it calculates the number of paths that go through $x$, store it
in some global variable, and return the size of the subtree rooted at $x$.
Please note, that to successfully use the above $numPaths(x)$ formula, for each
vertex $x$ we have to know the sizes of all the child subtrees if the whole tree
was actually rooted at $x$. In our DFS traversal we are missing the size of one
subtree for each non-root vertex. To calculate it, we can use the fact that
$\sum_{i=1}^k s_i = N - 1$.

The complete algorithm is illustrated by the following peseudocode:

\lstlang{python}\begin{lstlisting}
function DFS(x):
  ans = N * N - 1
  subtree_size = 1
  for v descendant of x:
    s = DFS(v)
    subtree_size += s
    ans -= s * s
  s = N - subtree_size - 1
  ans -= s * s
  // ans contains the number of busses that stops at x
  return subtree_size
\end{lstlisting}

The time complexity of the algoritm is $O(N)$
