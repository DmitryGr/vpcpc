\bigheading{Investigation}

\authors{Michal `Mimino' Danilák}{Michal `Mimino' Danilák}{Michal `Mimino' Danilák}

\heading{Terminology}

In the following text I will refer to the problem in the terms from graph theory,
where the houses are vertices and the roads are edges of the graph, which is also a
tree.

By \emph{querying a vertex} we mean to choose some house and search it.

By a \emph{strategy} we mean a correct decision tree of queries leading to discovery of the thief.
By an \emph{optimal strategy} we mean the decision tree of the lowest possible height.
In the following text we will consider only the reasonable strategies, where
we will never query a vertex for which the answer can be infered from the previous queries and answers.


\heading{What doesn't work}

During the contest we saw a plenty of submissions trying to solve the problem
with various incorrect strategies:
\begin{itemize}
\item choose the center \footnote{\url{http://en.wikipedia.org/wiki/Graph\_center}} of the tree,
\item choose a vertex, which minimizes the size of the biggest subtree,
\item choose a vertex, which minimizes the length of the longest path in any of the subtrees.
\end{itemize}

Even though every such strategy returns an asymptotically optimal solution,
requiring $O(\log N)$ queries, for each of them there exists a counter-example
where the strategy returns the answer worse by a constant factor.


\heading{Brute-force solution}

In each step we remember the set of vertices where the thief can possibly be hiding.
Then we try all the possible ways to choose the next vertex, solve the problem
recursively and return the best answer.

Clever implementation with bitmasks and memoization has time complexity $O(N\cdot2^N)$.
Implementing this solution first would not only give you 40 points, but it would also
help you find the counter-examples for the incorrect strategies you might come
up with and help you better understand the devil in this problem.


\heading{Optimal solution}

Imagine you already have a strategy which in the worst case requires $X$ queries.
We will now define a \emph{strategy function} $f : V \to \mathbb Z_{> 0}$ which maps
the vertices of the tree into positive integers. For each vertex $v$,
$f(v) = |\text{height of the node in the strategy's decision tree in which you query $v$}|$.
In other words, if $v_1$ is in the root of the decision tree (i.e. you first query $v_1$), then $f(v_1) = X$.
If vertices $v_2, v_3 ... v_i$ are on the second level of the decision tree
(i.e. based on the first answer you would query one of the $v_2, v_3 ... v_i$), then $f(v_2) = f(v_3) ... = f(v_i) = X-1$.
And so on, until you assign $1$ to the vertices deepest in the decision tree.
Because decision trees of reasonable strategies have for each vertex $v$ exactly
one node in which you query it, the function is always well-defined.

\begin{description}
\item [Lemma] (\emph{path property}):
For any strategy function holds that for each pair of distinct vertices $v_1, v_2 \in V$ if $f(v_1) = f(v_2)$, then
on the simple path from $v_1$ to $v_2$ there is a vertex $v_3$ such that $f(v_3) > f(v_1)$.
\item[Proof:]
It should be obvious from the construction of strategy function.
\end{description}

We will now show that actually every function $f : V \to \mathbb Z_{> 0}$ satisfying
the \emph{path property} represents a valid strategy which in the worst case
requires no more than $M_f$ queries, where $M_f$ is the maximum value of $f(v)$.
We will prove it by constructing a valid strategy from such function:

First vertex to query, $v$, will be the one with value $M_f$. There should be only
one such vertex in the tree, otherwise the \emph{path property} would not hold.
The second vertex to query in each of the subtrees separated by $v$ is the vertex with
the maximum value in that subtree. Again, from the \emph{path property} we know
there should be only one such vertex in each subtree. And so on.

So to determine the optimal strategy it suffices to compute a strategy function
with the lowest maximum value $M_f$.

We arbitrarily root the tree and compute $f(v)$ for each vertex in the bottom-up order.
For each $v$, $T(v)$ represents the subtree rooted at $v$.
Also we say that vertex $w \in T(v)$ is visible from $v$ if on the path from $v$
to $w$ there is no vertex $y$ such that $f(y) > f(w)$. If $w$ is visible from $v$,
we also say that value $f(w)$ is visible from $v$. Actually for each vertex
we can store the bitmask of all the values visible from $v$ in $T(v)$. For
example if values $\{5,4,2,0\}$ are visible, we will store $(110101)_2$. We will
call it a \emph{visibility mask} a write it $S(v)$.

For the bottommost vertices we set $f(v) = 1$.

For every other vertex $v$, $f(v)$ will be set to the lowest possible value, while
taking the visible values in $T(v)$ into account. For a vertex $v$ value $X$ is
forbidden in any of the two conditions hold:
\begin{itemize}
\item value $X$ is visible from $v$,
\item value $Y$ is visible from $v$ in at least two different subtrees and $Y \ge X$.
\end{itemize}

From the visibility masks $S(v_1), S(v_2) ... S(v_k)$ of the children of $v$ and the value $f(v)$
we can compute the value of $S(v)$ using the bit operators as follows:
\begin{align}
S_f &= 1 << f(v)
S(v) &= (S(v_1) | S(v_2) ... | S(v_k) | S_f) \& ~(S_f - 1).
\end{align}

What the complicated-looking formula does, is basically a bit OR of all the
children's visibility masks with the mask of $f(v)$ and then erasing all the
values lower than $f(v)$ from the mask, which are not visible anymore.

Finally the answer to the original problem is $M_f$, the maximum value $f(v)$ assigned
to any of the vertices.

Depending on how fast can we calculate the value of $f(v)$ the run time of the
algorithm can be $O(N)$ if we use the clever bit operations machinery and
calculate it in constant time. Otherwise we can simply sequentially find the lowest
possible value for $f(v)$, resulting in the total run time $O(N \cdot \log N)$ which
would also score 100 points.


\heading{Proof of correctness}

Now we will prove that the above algorithm always calculates the optimal strategy,
even though rooting the same tree by the different vertex can result into a possibly
different strategy, which is nevertheless optimal.

The function calculating the visibility mask $S(v)$ (i.e. the ugly bitwise expression above)
is \emph{monotone} and \emph{minimizing}. By \emph{monotone} we mean that by increasing
the value of the visibility masks of some of its children, we do not decrease
the value of the visibility mask of their parent. By \emph{minimizing} we mean
that for the given visibility masks of the children, we calculate the $S(v)$ with
the minimum possible value.

Lemma: If the function calculating $S(v)$ is \emph{monotone} and \emph{minimizing}, then
the bottom-up algorithm calculates a strategy function $f(v)$ for which $M_f$ is minimal.

Proof: First we will show that the computed visibility mask of the root is minimal. We will
prove it by induction on the height of the tree. Suppose that the masks computed for the
children of $v$ are minimal. It means that they are not greater than corresponding masks for
any other strategy function. Since $S(v)$ is \emph{monotone}, it wouldn't help if
any of the children values would be bigger. And because $S(v)$ is also \emph{minimizing},
there is no smaller visibility mask $S(v)$ for any strategy function.

Because the visibility mask at the root is minimal, so is $M_f$, because the visibility mask
from the root always contains $M_f$ and no greater values.
