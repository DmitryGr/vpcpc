\bigheading{Cubic Art}

\authors{Marián Horňák}{Marián Horňák}{Marián Horňák}

It is not hard to realize that the simple simulation
would work in quadratic time, which was enough to
score $50\%$ points! The problem is to write a bugless
code that simulates the cube. The key to do it
are the permutations.

\heading{Permutations}

Although I am sure the reader had met the permutations before,
I am going to explain basics. 
In the following discussion I will use array indexing
starting at $1$.

\smallskip

Permutation is an array $P$ containing each of the numbers
$1$\dots$k$ once. It is understood to be a rearrangement
of the array. $P[i]$ is the index $i$-th array element should
be moved to. Applying permutation $P$ to the array $A$ creates
an array $B = A \circ P$ such that
$\forall i: B[i] = (A \circ P)[i] = A[P[i]]$.
Permutation application can be implemented using simple loop.

\smallskip

It is easy to see that for array A and permutations P, Q holds:
$$\forall i: ((A \circ P) \circ Q)[i] = (A \circ P)[Q[i]] = A[P[Q[i]]] =
A[(P \circ Q)[i]] = (A \circ (P \circ Q))[i]$$ and thus
$(A \circ P) \circ Q = A \circ (P \circ Q)$.
What does this mean? Applying P to the A and
then Q to the result will have the same effect as
applying Q to the P and the the result to the A.
In other words, we can join P and Q to a single permutation
before we use them. Permutation joining is just
an application of the first one to the second.
The above equation also says, that the joining of the
permutations is associative (the order does not matter)

\smallskip

Permutation $I$ that satisfies $\forall i: I[i] = i$
is called identity and does nothing.
If $Q \circ P = I$ holds, $Q$ is called inverse
of $P$. Applying one permutation and then
another is same as doing nothing. Therefore,
the permutations must have opposite effects.
The equation $Q[P[i]] = I[i] = i$ can be used to
calculate $Q$ using $P$.

\heading{Cube simulation}

The simplest way to simulate the Rubik's cube,
is to represent any sequence of moves as a permutation
of small color squares. To keep it as simple as possible,
we can include all $54$ ($6\cdot3\cdot3$) squares
in the order they are in the input (and will be in
the output.)

\smallskip

Since the middles don't move, we can precalculate
the permutation for every elementary move. One way is
to do it all by hand. One can calculate rotating to
the other side using inverse. The other possibility is
to  write the foursomes that swaps in the cycle
and parse them to the permutation.

\smallskip

Now it's quite easy to make a simple simulation.
Read the moves and store corresponding permutations
in the array. After each update just rejoin whole
array and apply result to the original cube.
This works in $O(n \cdot m)$ time.

\heading{Optimal solution}

The only repetitive work we are doing is rejoining.
To make it effective we can use simple interval tree.

\smallskip

Let's enlarge the array of moves to the next
power of $2$ using identity permutation.
This twices the array in the worst, so it does not
affect the time complexity. Now build a complete
binary tree using the array as the leafs. Then let
each node to be a join of its children. Since
the permutation joining is associative, the root
will now contain the join of the whole array.

\smallskip

After updating a move and its permutation in the leaf,
it is sufficient to recalculate the path to the root.
This takes approximately $log(n)$ steps.
Therefore, the time complexity is $O(n + m \cdot log(n))$.
