\bigheading{Universities}

\authors{Michal `Mimino' Danilák}{Michal `Mimino' Danilák}{Michal `Mimino' Danilák}

In the following text I will refer to the problem in the terms from graph theory,
where the universities are vertices and the roads are edges of the graph, which is also a
tree. Each vertex is either black or white and has a cost associated with it.
We say that the path in the tree is \emph{balanced} if
it contains the same number of black and white vertices.
We are looking for the cost of the balanced path for which the sum of costs
of its vertices is maximized.

First I will explain the solution to the easier subproblem. Then I will apply
that solution to the original problem.


\heading{Subproblem}

You are given a vertex $v$, find the optimal balanced path that goes through $v$.

We root the tree by the vertex $v$. For each other vertex $w$ we calculate the path balance
$B[w] = \sum_{x \in path_{w->v}}balance_x$, where $balance_x$ is $1$ if the vertex is black and $-1$ otherwise.
Obviously if $B[w] = 0$, then the path from $w$ to $v$ is balanced.
For each vertex we also calculate the path cost
$C[w] = \sum_{x \in path_{w->v}}cost_x$.

Now we have two choices on how the optimal path can look like:
\begin{enumerate}
\item $v$ is one of the endpoints on the path,
\item $v$ is one of the inner vertices on the path.
\end{enumerate}

In the first case we can simply look at the vertices $w$ with $B[w] = 0$ and store
the best value $C[w]$ among them.

In the second case, let $c_1, c_2, ... c_k$ be the children of $v$, and $T(c_i)$
represent the subtree rooted at $c_i$. Then the endpoints of the optimal path
must lie in two different subtrees.
For each $c_i$ we will calculate the array $BEST_{c_i}[b] = \max_{w \in T(c_i), B[w] = b}C[w]$,
i.e. for each balance $b$ we will store the cost of the best path from $w$ to $v$ with balance $b$.
Because the indexes to this array can be negative, in C++ either use \emph{map} or
the following trick:
\lstlang{cpp}\begin{lstlisting}
// now you can index array best even with negative indexes with absolute value up to MAX_BALANCE
int _best[2*MAX_BALANCE], * best = _best + MAX_BALANCE;
\end{lstlisting}

If we want to find the optimal path with endpoints in subtrees $T(c_1)$ and $T(c_2)$, we can
iterate through all the values stored in $BEST_{c_2}$ and for each balance $b$
we can get the cost of the balanced path starting in $T(c_2)$ and ending in $T(c_1)$
as $BEST_{c_2}[b] + BEST_{c_1}[balance_v-b] - cost_v$ and update the best value found so far.
For simplicity the missing values in $BEST$ arrays have value $-\infty$.

After we processed the subtrees $T(c_1)$ and $T(c_2)$, we can merge their $BEST$ arrays
into one $BEST_{c_12}[b] = \max(BEST_{c_1}[b], BEST_{c_2}[b])$.
In the next step we will process the arrays $BEST_{c_12}$ and $BEST_{c_3}$ in the same way as described above.

Time complexity of this solution is linear in the number of vertices of the tree.


\heading{Original problem}

Back to the original problem, where the optimal path can go through any vertex.
We will define the strategy function previously introduced in the solution of \emph{Investigation} task.

Function $f : V \to \mathbb Z_{> 0}$, mapping the vertices into positive integers, is a \emph{strategy function} if
for each pair of distinct vertices $v_1, v_2 \in V$ with $f(v_1) = f(v_2)$ holds that
on the simple path from $v_1$ to $v_2$ there is a vertex $v_3$ such that $f(v_3) > f(v_1)$.

We can either construct an optimal strategy function as described in the solution of \emph{Investigation} task
or for the purposes of this problem any such function with the maximum value of $O(\log N)$ would be sufficient.
So you could also use any strategy described in \emph{What doesn't work} section of the \emph{Investigation} solution.

For each vertex $v$ we define $T_f(v) = \{w | w \in V, \forall y \in path_{w->v}: f(y) \leq f(v) \}$.
Basically it is the subtree consisting of vertices reachable from $v$ by traversing only the vertices for which $f(w) \leq f(v)$.

\begin{description}
\item[Lemma:] For each pair of distinct vertices $v_1, v_2 \in V$ with $f(v_1) = f(v_2)$ holds that
the subtrees $T_f(v_1)$ and $T_f(v_2)$ are disjoint.
\item[Proof:] Follows from the property of strategy function.
\end{description}

What will we do now is for each vertex $v$ we will `guess' that the optimal path
goes through $v$ and solve the subproblem with the algorithm described above.
But for the subproblem we will only consider the vertices of
the subtree $T_f(v)$, not the whole tree.

Imagine the optimal path in the graph. Let $M_path = \max\{f(v) | v \in path\}$.
There is exactly one vertex $v$ for which $f(v) = M_path$,
for which holds that $path \subseteq T_f(v)$. Therefore we would discover this
optimal path when solving the subproblem for the vertex $v$.

Time complexity of the whole algorithm is $O(N * \log N)$ - there are $O(\log N)$ different values of $f(v)$
and from the lemma we know, that the subtrees $\{T_f(v) | v \in V, f(v) = x\}$ are
pairwise disjoint, therefore solving all the subtrees with the same root value takes in total $O(N)$.

