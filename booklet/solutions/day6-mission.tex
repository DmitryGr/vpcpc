%%
\bigheading{Mission}
\authors{Filip Hlásek \& Michal `Mimino' Danilák}{Filip Hlásek \& Michal `Mimino' Danilák}{Filip Hlásek}

Every path from $B$ to $E$ to $H$ is equivalent to a pair of vertex disjoint
paths -- one from $E$ to $B$ and the other from $E$ to $B$. We can even add 
a new vertex $X$ and join it by edges of the same length with $B$ and $E$
respectively. Now the problem is to find two vertex disjoint paths from
$E$ to $X$.

We have reduced the task to a standard
\emph{Minimum-cost flow problem}\footnote{\url{
  http://en.wikipedia.org/wiki/Minimum-cost_flow_problem
}}
\footnote{\url{
  http://community.topcoder.com/tc?module=Static&d1=tutorials&d2=minimumCostFlow1
}}.
Usually there are edges with capacities, but in our case vertices
have capacities equal to 1.

\heading{Vertex capacities}

We must ensure that no vertex is visited more than once. We can do it by
splitting every vertex into two -- one is called input and the other output.
A directed edge $(a, b)$ from the original network goes from the output
part of vertex $a$ to the input part of vertex $b$ in the new network.
Additionally there is a directed edge from the input part to the output part
of a vertex and its capacity is equal to the vertex's capacity. The resulting
network is equivalent to the given network (in sense of flow), but only edges
have capacities. The size of network was only linearly increased.

\heading{Min-cost max-flow}

There are many approaches to this problem and all of them should be sufficient
to solve the task. We will present one of the easiest without the proof of
correctness.

\begin{enumerate}
  \item Find the shortest path from $E$ to $X$ using Bellman-Ford
  algorithm (or any other).
  \item For every edge on the path,
  decrease it's capacity by 1 and add an edge in the opposite
  direction with capacity equal to 1 and opposite cost.
  \item Find the shortest path from $E$ to $X$ using Bellman-Ford
  algorithm (or any other which can deal with negative cost edges).
  \item Sum of cost of the two paths is the required answer.
\end{enumerate}

Overall complexity of the given algorithm is $O(NM)$.
