\input sys/inputs.tex

\begin{document}

\bigheading{Dynamic memory allocation}

% \info{task_name}{infile}{outfile}{points}{timelimit}{memlimit}
% leave this values, if you are not interested
\info{malloc}{stdin}{stdout}{100}{1200 ms}{1 GB}

Kamila is once again devising a new amazing programming language that will be
superior to all other languages in every aspect. Writing code will become even
simpler than ABC, so little children will learn how to program before they will
know how to write comments. This is not a bug,%
\footnote{By the way, it will be impossible to introduce bugs into a program --
Kamila's brilliant compiler will optimize them out.}
as comments are absolutely unnecessary in Kamila's dreamt up language: Due to
its intuitive syntax, the purpose of any block of code is immediately crystally
clear. Also, the compiler will bake you a cake for your birthday.

There is still a handful of outstanding issues though. For example, the language
constructs for memory management do not work yet. Kamila prepared a precise
specification of the memory allocation and freeing procedures, but she did not
find enough time to implement them. Help her, and achieve instant fame for
contributing to such an important project!


\heading{Task}

The available memory is an array of $n$ bytes numbered $0$ through $n-1$. At the
beginning, all the bytes are free (i.e., not allocated). Then the memory
management system allocates and frees the bytes according to a sequence of
queries.

An \emph{allocation} query is specified by an integer $\ell$. The system finds a
block of $\ell$ consecutive free bytes, allocates them, and returns the position
of the first byte in this block. If there are multiple such blocks available,
the one starting at the position with the smallest number is chosen. If there is
no such block available, the query is rejected and the system returns $-1$.

A \emph{freeing} query is specified by two integers $x$ and $\ell$. The system
marks the block of $\ell$ consecutive bytes starting at the position $x$ as
free, and returns the number of actually freed bytes (i.e., the number of bytes
in this block that were not free before the query).


\heading{Input}

The first line of the input consists of two integers $n$ and $q$ -- the number
of bytes in the available memory and the number of queries ($1 \leq n, q \leq
3\cdot 10^5$).

Each of the following $q$ lines describes a query. The first integer in a line
determines the type of the query: $1$ is for allocation and $2$ for freeing. A
line with an allocation query then continues with one additional integer $\ell$
($1 \leq \ell \leq n$). Similarly, a line with a freeing query continues with
two integers $x$ and $\ell$ ($0 \leq x \leq n-1$, $1 \leq \ell \leq n-x$).


\heading{Output}

For every query in the given sequence, output one line with the value returned
by the system.


\heading{Samples}

\sampleIN
5 4
1 3
1 3
2 1 3
1 4
\sampleOUT
0
-1
2
1
\sampleEND


\end{document}
