\input sys/inputs.tex

\begin{document}

\bigheading{Wall}

% \info{task_name}{infile}{outfile}{points}{timelimit}{memlimit}
% leave this values, if you are not interested
\info{wall}{stdin}{stdout}{100}{1000 ms}{128 MB}

Mirek is a conservator.
His job is to maintain monuments, keeping them in good condition.
Mirek's today task is to repair the defense wall of an old fortress.
The wall is about to fall to pieces, so he must hurry.
He~searched through the Internet and found a robot designed to repair such walls extremely fast.
After purchasing the robot, he tried to make the optimal repairment plan, but this was way too difficult for him.

The wall can be considered as a straight line.
Mirek wrote down all points on the wall, which require repairing.
For each point he knows, what would be the cost $C_i$ of repairing it, and a coefficient $D_i$
	-- how would the cost increase if it wasn't repaired immediately.
If~the $i$-th point will be repaired after time $t$, then the cost of repairing it would be equal to:
\[ C_i + t \cdot D_i \]

\heading{Task}

Given the coordinates of all points on the wall and the initial position of the robot,
	and knowing that transportation of the robot from
	point $x_1$ to point $x_2$ would take $|x_1 - x_2|$
	time\footnote{Here, $x_1$ and $x_2$ are the coordinates of these two points (not their indices).},
	calculate the minimum cost of repairing all spoiled points.
You can assume that repairing a single point takes no time.

\heading{Input}

On the first line of input there are two integers $N$ and $P$ ($1 \le N \le 2\,000$, $0 \le P \le 10^9$)
	-- the number of points on the wall, which should be repaired, and the initial position of the robot.
Then, $N$ lines follow, $i$-th of these lines describes $i$-th point on the wall
	and contains three integers $X_i$, $C_i$ and $D_i$ ($0 \le X_i \le 10^9$, $0 \le C_i, D_i \le 10^6$, $X_i \neq P$)
	--~the position of the point and cost coefficients.
There are no two points with equal positions.

\heading{Output}

Output a single line with integer $C$, where $C$ is the minimal cost of repairing
	all points on the wall.

\heading{Sample}


\sampleIN
3 7
10 32 1
3 5 1
14 0 2
\sampleOUT
72
\sampleCOMMENT
The optimal plan of repairing points is:
\begin{itemize}
	\item Transport robot from point $7$ to point $10$ and repair the first point after time $3$.
	\item Transport robot from point $10$ to point $14$ and repair the third point after time $3 + 4 = 7$.
	\item Transport robot from point $14$ to point $3$ and repair the second point after time $3 + 4 + 11 = 18$.
\end{itemize}
Thus, the total cost of repairing the wall is $5 + 18 \cdot 1 + 32 + 3 \cdot 1 + 0 + 7 \cdot 2 = 72$.
\sampleEND


\end{document}
