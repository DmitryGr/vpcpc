\input sys/inputs.tex

\begin{document}

\bigheading{Cubic Art}

% \info{task_name}{infile}{outfile}{points}{timelimit}{memlimit}
% leave this values, if you are not interested
\info{rubik}{stdin}{stdout}{100}{1000 ms}{1 GB}

Modern art is unpredictable.
When Bob was tidying his room he found his old Rubik's cube.
Then the moment came.
He closed his eyes, listened to his inner voice,
made a few moves (up to $65\,000$) and the masterpiece was nearly complete.
But the final state was not to his liking. He realized he 
did some of the moves incorrectly. If only he could go back in time
and change them!

All he now needs to do is a few changes (again, up to $65\,000$ of them).
Each of the changes consists of replacing one move with some other move.
Bob would like to do see what each change does.
But it is annoying to repeat the entire sequence
of moves again and again.
%So someone has to code it.
%But who will code a Rubik's cube simulation?
%Why would one ever do it?

\heading{Task}

You are given the initial state of Bob's Rubik's cube.
(The cube is not necessarily solved in its initial state.)
You are also given the original sequence of moves Bob performed.

Finally, you are given a sequence of changes.
Each change is of the form ``change the $k$-th move into this new move''.
For each change, output the state of the cube \emph{at the end} of the
entire sequence of moves.

Note that the changes are permanent -- for example, the second change
should be applied to the sequence of moves with the first change, 
not to the original sequence.

\heading{Cube's I/O}

Let the cube's colors be $A$,$B$,$C$,$D$,$E$,$F$.
When you are playing with the cube,
the middle squares of its faces do not move.
Therefore, we will always use $A$
as the color of the center of the top face,
$B$, $C$, $D$, $E$ as the centers of the side faces (in order),
and $F$ as the center of the bottom face.
The surface of the cube can then be unfolded into the following form:\\
\texttt{
???\\
?A?\\
???\\
????????????\\
?B??C??D??E?\\
????????????\\
???\\
?F?\\
???\\
}

\heading{Input}

First 9 lines of the input contain the description of
the starting state of the cube. The description is given
in the above form. You may assume that the centers of the 
six faces are labeled as shown above.\footnote{You may also assume
that the starting state is a valid configuration that can be obtained
from the solved state. However, this is actually irrelevant in our problem.}

Then there is a line with two integers $n$ and $m$:
$n$ is the number of moves and $m$ is number of subsequent changes.

Next $n$ lines are describing Bob's original moves.
They have the form ``$C_i$ $d_i$'',
where $C_i$ is the color of the center of the rotated side
and $d_i$ is $-1$ in case of a clockwise move
and $1$ in case of a counterclockwise move.

The last $m$ lines describe the changes, in order.
Each one has the form ``$a_j$ $C_j$ $d_j$'',
where $a_j$ is the (1-based) index of the move that
is being replaced and $C_j$ $d_j$ describes the 
new move.

\heading{Constraints}

In all test cases, $n,m \leq 65\,000$. In $50\%$ of testcases $n,m \leq 1000$.

\newpage

\heading{Output}

Let $S_i$ be the sequence of moves obtained from the initial sequence
by applying the first $i$ changes.

For each $i$ between 1 and $m$, inclusive, output 9 lines: the final state 
of the cube obtained by starting in the initial configuration and performing
the sequence of moves $S_i$. Use the same format as in the input.

\heading{Samples}

\sampleIN
AAA
AAA
AAA
BBBCCCDDDEEE
BBBCCCDDDEEE
BBBCCCDDDEEE
FFF
FFF
FFF
8 4
E 1
E -1
B 1
B -1
F 1
F -1
E 1
E -1
8 C -1
2 C -1
6 A -1
4 D -1
\sampleCOMMENT
The original moves cancel each other out.
At the end of the original sequence of moves,
the cube is back in its initial state.

After we make the four changes described in the input,
we obtain a sequence of moves that flips everything
except for the centers of all sides.
\sampleOUT
DAD
DAD
DAD
ABACCCFDFEEE
ABACCCFDFEEE
ABACCCFDFEEE
BFB
BFB
BFB
FAF
FAF
FAF
DBDCCCBDBEEE
DBDCCCBDBEEE
DBDCCCBDBEEE
AFA
AFA
AFA
EDE
FAF
EDE
DCDFCFBEBAEA
ABADCDFDFBEB
DCDFCFBEBAEA
CBC
AFA
CBC
DDD
DAD
DDD
CCCFFFEEEAAA
CBCFCFEDEAEA
CCCFFFEEEAAA
BBB
BFB
BBB
\sampleEND


\end{document}
