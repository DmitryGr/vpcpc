\input sys/inputs.tex

\begin{document}

\bigheading{Cubic Art}

% \info{task_name}{infile}{outfile}{points}{timelimit}{memlimit}
% leave this values, if you are not interested
\info{rubik}{stdin}{stdout}{100}{1000 ms}{1 GB}

The modern art is unpredictable.
While Bob was tidying his room he has found his old Rubik's cube.
Then the moment came.
He closed his eyes, listened to his inner voice,
made few\footnote{$65000$}
moves and the masterpiece was nearly complete.
All he need to do is a few\footnote{again $65000$}
changes. Better said, he has to change some moves.
And he would like to do see what each change does.
But it is difficult to repeat the sequence
of moves again and again.
So someone has to code it.
But who will code a Rubik's cube simulation?
Why would one ever do it?

\heading{Task}

You are given a Rubik's cube (not neccessary solved),
the sequence of moves Bob did.
and the sequence of changes to the original moves.
Output the final state of cube after each change.

\heading{Cube's IO}

Let the cube's colors be $A$,$B$,$C$,$D$,$E$,$F$
When you are playing with the cube,
the relative position of the middle squares
does not change. Therefore, the cube
can be always rotated to have middle
squares in fixed positions.
Then you can expand the surface of the cube
and express it in this unique form:\\
\texttt{
???\\
?A?\\
???\\
????????????\\
?B??C??D??E?\\
????????????\\
???\\
?F?\\
???\\
}

\heading{Input}

First 9 lines of the inputs are describtion of
the starting state of the cube.

Then there is a line with two integers $n$ and $m$.
$n$ is the number of moves and $m$ is number of changes.

Next $n$ lines are describing original Bob's moves.
They have form of $C_i$ $d_i$,
where $C_i$ is the middle color of moved side
and $d_i$ is $-1$ in case of clockwise move
and $1$ in case of counterclockwise move.


The last $m$ lines describes changes.
Each one has form of $a_j$ $C_j$ $d_j$,
where $a_j$ is the number of move that
is being replaced and $C_j$ $d_j$ describes the 
new move.
The moves are numbered $1 \dots n$.

\smallskip

It holds $n,m \leq 65000$.\\
In the $50\%$ of testcases also holds
$n,m \leq 1000$.

\heading{Output}

Output 9 lines describing cube's final state after each of $m$ changes.

\heading{Samples}


\sampleIN
AAA
AAA
AAA
BBBCCCDDDEEE
BBBCCCDDDEEE
BBBCCCDDDEEE
FFF
FFF
FFF
8 4
E 1
E -1
B 1
B -1
F 1
F -1
E 1
E -1
2 C -1
4 D -1
6 A -1
8 C -1
\sampleCOMMENT
The original moves are interfering.
We transform them to the known trick
that rotates the middles.
\sampleOUT
DAD
DAD
DAD
ABACCCFDFEEE
ABACCCFDFEEE
ABACCCFDFEEE
BFB
BFB
BFB
CCC
DAD
CCC
AAABCBFFFDED
BBBFCFDDDAEA
AAABCBFFFDED
EEE
BFB
EEE
CDC
CAC
CDC
BCBFFFDEDAAA
BBBFCFDDDAEA
BCBFFFDEDAAA
EBE
EFE
EBE
DDD
DAD
DDD
CCCFFFEEEAAA
CBCFCFEDEAEA
CCCFFFEEEAAA
BBB
BFB
BBB
\sampleEND


\end{document}