\input sys/inputs.tex

\begin{document}

\bigheading{Bus lines}

% \info{task_name}{infile}{outfile}{points}{timelimit}{memlimit}
% leave this values, if you are not interested
\info{buslines}{stdin}{stdout}{100}{1000 ms}{1 GB}

In the Czech city called Kocourkov they have a spectacular public transportation system.
It consists of $N$ bus stops and $N-1$ bidirectional roads, each road connecting two bus stops.
It is possible to get from each bus stop to every other using a sequence of roads.

Every morning, for each pair of distinct bus stops $a$ and $b$ there is exactly one bus
that starts at $a$ and goes to $b$ (along the only direct path).
That is, there are a total of $N(N-1)$ buses.
Each bus stops at all bus stops it visits along the way.

At every bus stop there must be a timetable listing all the buses that stop there (including
buses that start or end their journey there).
You are now wondering how many buses are listed on each timetable.

\heading{Task}

You are given the description of the traffic system in Kocourkov.
For every bus stop in the city calculate the number of buses that stop on that particular stop.

\heading{Input}

First line contains an integer $N$, the number of bus stops in the city (stops are numbered from $1$ to $N$).
The following $N-1$ lines describe the roads in the city.
Each line contains two different integers $1 \le x, y \le N$ meaning that there is a road connecting bus stops $x$ and $y$.

\bigskip
\noindent
It holds $1 \leq N \leq 10^6$.\\
In the $20\%$ of testcases $N \leq 100$.\\
In the $40\%$ of testcases $N \leq 1000$.

\heading{Output}

The output consists of $N$ lines. The $i$-th line should contain a signle integer, the number of buses that stop on the $i$-th bus stop.

\heading{Samples}

\sampleIN
6
1 2
2 3
3 4
4 5
5 6
\sampleOUT
10
18
22
22
18
10
\sampleEND

\bigskip

\sampleIN
5
4 5
2 1
3 2
2 5
\sampleOUT
8
18
8
8
14
\sampleEND

\end{document}
