\input sys/inputs.tex

\begin{document}

\bigheading{Investigation}

% \info{task_name}{infile}{outfile}{points}{timelimit}{memlimit}
% leave this values, if you are not interested
\info{investigation}{stdin}{stdout}{100}{1000 ms}{1 GB}

There was a robbery in a city of Bytelandia.
The thief successfully escaped and hid somewhere in the city.
You are investigating this crime. Your goal is to find and arrest the thief.

The city consists of $N$ houses connected with each other with $N-1$ roads in
such a way, that there exists a unique path between any two houses (i.e. the city forms a tree structure).
The thief is hiding in one of the houses.

To locate the thief, you can choose a house $h$ and search it.
If it was the house where the thief was hiding, you arrest him.
Otherwise you interrogate the inhabitants of the house and they provide you with the following information:
``If you picture the city as a rooted tree, with $h$ as its root and $c_1, c_2 \ldots c_m$ as the children of $h$,
then the thief is hiding in one of the houses of the subtree rooted at $c_i$ (for some $i$, $1 \leq i \leq m$).''

You have to keep searching the houses until you find and arrest the thief.
You can suppose that the thief stays hidden in the same house during the whole investigation process (i.e. he doesn't change the location).

Obviously, the order in which you search the houses matters, because even if you don't
find the thief in some house, with the provided information you can highly reduce
the number of possible houses where the thief can be hiding.
So you need to come up with an optimal strategy that
minimizes the number of searched houses in the worst possible scenario.

\heading{Task}

You are given the description of the city.
Come up with a strategy, that minimizes the number
of houses you need to search in the worst possible scenario.

\heading{Input}

First line contains integer $N$, number of houses in the city (houses are numbered from $0$ to $N-1$).
Second line contains $N-1$ space separated integers, $v_1 v_2 \ldots v_{N-1}$.
Integer $v_i$ ($1 \leq i \leq N-1$) means there is a road connecting the houses with numbers $v_i$ and $i$ ($v_i < i$).

\smallskip

It holds $2 \leq N \leq 10^5$.\\
In the $20\%$ of testcases $N \leq 10$.\\
In the $40\%$ of testcases $N \leq 20$.\\
In the $60\%$ of testcases $N \leq 1000$.

\heading{Output}

Output exactly one integer, number of houses you need to search in the worst possible scenario,
when searching with the optimal strategy.

\heading{Samples}


\sampleIN
5
0 1 1 1
\sampleCOMMENT
The city looks like a star, with house 1 in the middle.
\sampleOUT
2
\sampleCOMMENT
First search the house 1.
If the thief wasn't there, after interrogation you will know in which house is the thief hiding.
\sampleEND


\bigskip


\sampleIN
8
0 1 2 1 3 5 6
\sampleOUT
3
\sampleEND


\end{document}
