\input sys/inputs.tex

\begin{document}

\bigheading{An inexperienced slalomer}

% \info{task_name}{infile}{outfile}{points}{timelimit}{memlimit}
% leave this values, if you are not interested
\info{slalom}{stdin}{stdout}{100}{300 ms}{1 GB}

It is Hubert's first slalom race, so naturally he feels very nervous. Moreover,
it is also his first day on skis and he has not learnt how to make turns yet --
he can only slide along a straight line. But nothing is lost yet. Maybe the
designer of this particular course was not careful enough, so it is actually
possible to pass through every gate without taking a single turn.


\heading{Task}

You are given the description of a slalom course with $n$ gates. The course runs
from left to right. Each gate is represented by a vertical line segment between
two poles. From a bird's-eye view, Hubert looks like a disk with diameter $d$
($d \geq 0$) and the trajectory of his center follows a straight line. He can
choose his start point anywhere to the left of the leftmost gate, and his finish
point anywhere to the right of the rightmost gate. To complete the course,
Hubert must pass with his entire body between the poles of all gates. Touching
the poles is allowed.

Find the largest diameter $d$ such that there is a trajectory enabling Hubert to
complete the course.


\heading{Input}

The first line of the input contains a single integer $n$ -- the number of gates
($1 \leq n \leq 10^5$). Each of the $n$ following lines describes a gate and
consists of three space-separated integers $x, y_1, y_2$ ($0 \leq x \leq 10^9$,
$0 \leq y_1 \leq y_2 \leq 10^9$). The described gate is the vertical line
segment with endpoints $[x, y_1]$ and $[x, y_2]$. No two gates have the same
$x$-coordinate.


\heading{Output}

Output a single line with the largest $d$ such that Hubert can complete the
course. We will accept answers with absolute or relative error less than
$10^{-9}$. In C++, you can output the answer using
\verb!printf("%.10lf\n", d);!

If there is no such non-negative value of $d$, output a single line with word
\texttt{Impossible}.


\heading{Samples}

\sampleIN
3
4 3 7
6 6 9
1 5 10
\sampleOUT
1.3728129460
\sampleEND

\sampleIN
2
3 7 9
10 4 4
\sampleOUT
0.0000000000
\sampleEND

\sampleIN
3
0 4 7
2 0 3
4 4 7
\sampleOUT
Impossible
\sampleEND


\end{document}
