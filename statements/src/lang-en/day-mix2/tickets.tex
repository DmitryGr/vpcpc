\input sys/inputs.tex

\begin{document}

\bigheading{Tickets}

% \info{task_name}{infile}{outfile}{points}{timelimit}{memlimit}
% leave this values, if you are not interested
\info{tickets}{stdin}{stdout}{100}{800 ms}{32 MiB}

The match of the year will be played next week. There are N seats in the stadium numbered by the integers 1 to $N$. Each fan can request one ticket and can specify the range of seats where he would be willing to sit. The range is specified by two integers $F$ and $L$, the first and last seat in the range, respectively. This means that the fan accepts any seat $S$ such that $F \leq S \leq L$ holds. The ticket office has already received $M$ requests from football fans and wants to select the maximal number of requests that can be simultaneously satisfied.

\heading{Task}
Write a program that computes the maximal number of fans that each can obtain a ticket with a suitable seat, and gives an adequate seat assignment. No two fans can get the same seat.

\heading{Input}
The first line of the input contains two integers $N$ ($1 \leq N \leq 100000$), the number of seats, and $M$ ($1 \leq M \leq 1000000$), the number of requests. The seats are numbered by $1,\ldots,N$. Each of the next $M$ lines contains two integers $F$ and $L$ ($1 \leq F \leq L \leq N$), a request specification. Requests are identified by the numbers $1,\ldots,M$ in the order of their appearance in the input.

\heading{Output}
The first line of the output must contain one integer $K$, the maximal number of the selected requests. Each of the next $K$ lines contains two integers $S$ $R$, a seat assignment, where $S$ is a seat number and $R$ is the number of the request obtaining the seat $S$. If there are multiple solutions, your program should output only one; it does not matter which one.

\heading{Samples}
\sampleIN
10 9
1 3
2 4
5 7
2 6
1 5
3 7
4 8
7 9
3 8
\sampleOUT
9
1 1
2 5
3 2
4 4
5 6
6 9
7 3
8 7
9 8
\sampleCOMMENT

\sampleEND
\bigskip

\end{document}
