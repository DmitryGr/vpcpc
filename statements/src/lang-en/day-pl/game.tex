\input sys/inputs.tex

\begin{document}

\bigheading{Game}

% \info{task_name}{infile}{outfile}{points}{timelimit}{memlimit}
% leave this values, if you are not interested
\info{game}{stdin}{stdout}{100}{1000 ms}{64 MB}

Mirek really likes playing with numbers. Together with his friend, Kamil, he plays a following game. At the beginning, there are two
non-negative integers -- $A$ and $B$. Let's say $A \leqslant B$. The players can perform one of two moves in turns:

\begin{itemize}
	\item Replace $B$ with $B - A^K$. Number $K$ can be any integer chosen by the player, considering the limitations that $K > 0$
	      and $B - A^K \ge 0$.
	\item Replace $B$ with $B\ mod\ A$. 
\end{itemize}

If $B \leqslant A$, similar moves are possible. The player who changes any number into $0$, wins. Mirek always starts. He likes this game, but he likes winning much more. Help him determine who will win, if both of them play optimally.

\heading{Task}

You are given the description of games played by Mirek and Kamil. For every game determine who will win, Mirek or Kamil.

\heading{Input}

First line contains an integer $T$ ($1 \le T \le 10^4$), the number of games played by boys.
In the next $T$ lines, there are descriptions of those games.
Every such line contains two integers $A$, $B$ ($1 \le A, B \le 10^{18}$) 
\smallskip

In the $30\%$ of testcases $A,B \le 1000$.

\heading{Output}

Output $T$ lines. The $i$-th line should contain the name of the player who wins the $i$-th game, \texttt{Mirek} or \texttt{Kamil}.

\heading{Samples}

\sampleIN
4
1 1
12 4
4 6
15 31
\sampleOUT
\texttt{Mirek}
\texttt{Mirek}
\texttt{Kamil}
\texttt{Mirek}
\sampleEND

\end{document}
