\input sys/inputs.tex

\begin{document}

\bigheading{Nyomozás}

% \info{task_name}{infile}{outfile}{points}{timelimit}{memlimit}
% leave this values, if you are not interested
\info{investigation}{stdin}{stdout}{100}{1000 ms}{1 GB}

Egy bűnözőt keresnek a városban, ahol $N$ ház van. A házakat $N-1$ út köti össze úgy, hogy bármely két ház között pontosan egy útvonal van (tehát az úthálózat fa szerkezetű).
Meg kell találni a bűnözőt, aki az egyik házban tartózkodik.


A keresés úgy zajlik, hogy kiválasztanak egy $h$ házat és átvizsgálják, hogy ott van-e a bűnöző. Ha abban a házban van a bűnöző, akkor a nyomozás sikeresen befejeződik. Ha nem ott van, akkor kikérdezik a ház lakóit. A feltett kérdés az úthálózat fa szerkezetén alapszik. Ha a fában a $h$ ház közvetlen szomszédai a $c_1, c_2 \ldots c_m$ házak, akkor az a kérdés, hogy melyik $c_i$ gyökerű részfában lévő házban bújt el a bűnöző? A ház lakói meg tudják válaszolni az ilyen kérdést.
A keresést addig kell folytatni, amíg meg nem találják a bűnözőt. A bűnöző a nyomozás közben nem változtatja helyét.

Nyilvánvaló, hogy a megtaláláshoz szükséges kérdések száma függ a kérdezési sorrendtől.
Ezért olyan optimális keresési stratégiát kell választani, amely a legrosszabb esetet tekintve a lehető legkevesebb kérdést igényli. 
Tehát annyi kérdést, amellyel a legrosszabb esetben is kitalálható, de ennél kevesebbel nem.

\heading{Feladat}
Írj olyan programot, amely kiszámítja, hogy a legrosszabb esetben minimálisan hány házat kell átvizsgálni, hogy megtalálják a bűnözőt! 

\heading{Bemenet}

A bemenet első sora egy egész számot tartalmaz, a város házainak $N$ számát. A házakat az $0, \ldots,N-1$ számokkal azonosítjuk.\\
A második sor $N-1$ egész számot tartalmaz egy-egy szóközzel elválasztva: $v_1 v_2 \ldots v_{N-1}$.\\
A $v_i$ ($1 \leq i \leq N-1$) szám azt jelenti, hogy közvetlen út van a $v_i$ és az $i$ ($v_i < i$) házak között, azaz szomszédok.

\smallskip
\noindent \textbf{Korlátok:}\\
$2 \leq N \leq 10^5$.\\
A tesztesetek $20\%$-ában  $N \leq 10$.\\
A tesztesetek $40\%$-ában   $N \leq 20$.\\
A tesztesetek $60\%$-ában   $N \leq 1000$.

\heading{Kimenet}

A kimenet egyetlen egész számot tartalmazzon, azt, hogy a legrosszabb esetben minimálisan hány házat kell átvizsgálni, hogy megtalálják a bűnözőt!

\heading{Példa bemenet és kimenet}

\sampleIN
5
0 1 1 1
\sampleCOMMENT
A város csillag szerkezetű, az 1 ház van középpen.
\sampleOUT
2
\sampleCOMMENT
Elsőként a 1-es házat kell vizsgálni. Ha nem ott van a bűnöző, akkor megtudjuk, hogy melyik házban van.
\sampleEND


\bigskip


\sampleIN
8
0 1 2 1 3 5 6
\sampleOUT
3
\sampleEND


\end{document}
