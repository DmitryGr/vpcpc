\input sys/inputs.tex

\begin{document}

\bigheading{Buszjáratok}

% \info{task_name}{infile}{outfile}{points}{timelimit}{memlimit}
% leave this values, if you are not interested
\info{buslines}{stdin}{stdout}{100}{1000 ms}{1 GB}

Kocourkov város közlekedési hálózata nagyon speciális. $N$ buszmegálló van, és $N-1$ közvetlen utca a buszmegállók között. Bármely két buszmegálló között pontosan egy útvonal van (tehát az úthálózat fa szerkezetü).

A város közlekedési vállalata elhatározta, hogy bármely két megálló között külön buszjáratot üzemeltet.
Látható, hogy így összesen $\frac{N \cdot (N - 1)}{2}$ buszjárat lesz. Lesz olyan, amely csak két megállót tartalmaz, és olyan is, amely mind az $N$-et.

A vállalat minden megállóba ki akarja függeszteni, hogy az adott megállón hány járat megy keresztül (beleértve azt is, ha a megálló a járat valamelyik végpontja).

\heading{Feladat}

Írj olyan programot, amely minden megállóra kiszámítja, hogy ott hány buszjárat megy keresztül!

\heading{Bemenet}

A bemenet első sora egy egész számot tartalmaz, a buszmegállók $N$ számát. A buszmegállókat az $1,\ldots,N$ számokkal azonosítjuk.\\
A következő $N-1$ sor adja meg a város utcáit. Mindegyik sor két különböző egész számot tartalmaz: $x \,\, y$  ($1 \le x, y \le N$), ami azt jelenti, hogy az $x$ és $y$ buszmegállók között van közvetlen utca.\\
\smallskip
\noindent \textbf{Korlátok:}\\
 $1 \leq N \leq 10^6$.\\
A tesztesetek $20\%$-ában  $N \leq 100$.\\
A tesztesetek $50\%$-ában $N \leq 1000$.

\heading{Kimenet}
A kimenet pontosan $N$ sort tartalmazzon. Az $i$. sor azon buszjáratok számát tartalmazza, amelyek keresztül haladnak az $i$. buszmegállón!

\heading{Példa bemenet és kimenet}

\sampleIN
6
1 2
2 3
3 4
4 5
5 6
\sampleOUT
5
9
11
11
9
5
\sampleEND

\bigskip

\sampleIN
5
4 5
2 1
3 2
2 5
\sampleOUT
4
9
4
4
7
\sampleEND

\end{document}
