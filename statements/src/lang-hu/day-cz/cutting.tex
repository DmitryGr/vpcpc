\input sys/inputs.tex

\begin{document}

\bigheading{Tortaszeletelés}

% \info{task_name}{infile}{outfile}{points}{timelimit}{memlimit}
% leave this values, if you are not interested
\info{cutting}{text files}{text files}{100}{-}{-}

Mimino $N$-edik születésnapját ünnepli. Kamilla egy nagy\footnote{A torta olyan nagy, hogy egy végtelen síkkal reprezentálható} tortát sütött neki, és $N$ darab gyertyát helyezett el rajta. Mimino úgy akarja felszeletelni a tortát egyenes vonalakkal, hogy minden keletkező szeleten legfeljebb egy gyertya legyen. Mimino kíváncsi, hogy minimálisan hány szeleteléssel tudja ezt megoldani.


\heading{Feladat}

Adott a síkon $N$ darab pont. Add meg egyenesek olyan halmazát, amelyek úgy bontják a síkot részekre, hogy minden részbe legfeljebb egy pont essen! Törekedj arra, hogy minél kevesebb egyenessel oldd meg a feladatot!

A pontok koordinátái egész számok. Minden egyenest két pontjával kell megadni (egész koordinátákkal). Az egyenesek irányítottnak tekintendők, az első pontjuktól a második felé. Az irányításnak akkor van jelentősége, ha a bemenet egy pontja egy egyenesre esik, ekkor a pontot az egyenestől jobbra lévőnek kell tekinteni.

Az egyenesek csak vízszintesek, függőlegesek vagy átlósak lehetnek.

\heading{Bemenet}

A bemenet első sora a pontok $N$ számát tartalmazza. A következő $N$ sor mindegyike egy pont koordinátáit tartalmazza: az $i+1$. sor az $i$. pont $x$ és $y$ koordinátáját. 

\heading{Kimenet}

Az első sor az egyenesek $L$ számát tartalmazza. A következő $L$ sor mindegyike egy egyenest ad meg két pontjával. Mindegyik sor négy egész számot tartalmaz: $X_1$, $Y_1$, $X_2$, $Y_2$, a két pont koordinátáit. Az ($X_1$, $Y_1$) és ($X_2$, $Y_2$) pontoknak különbözőnek kell lennie, és teljesülnie kell az alábbiaknak:

\begin{itemize}
  \item $X_1 = X_2$ (függőleges egyenes)
  \item $Y_1 = Y_2$ (vízszintes egyenes)
  \item $X_1 - X_2 = Y_1 - Y_2$ (átlós egyenes)
  \item $X_1 - X_2 = Y_2 - Y_1$ (átlós egyenes)
\end{itemize}

A megoldásod legfeljebb $10\,000$ egyenest tartalmazhat, és pontjai koordinátáinak abszolút értékei legfeljebb $10\,000$ lehet.


\heading{Pontozás}

Ez egy nyílt bemenetű feladat. Letöltheted mind a $10$ teszteset bemenetét. Csupán a megfelelő kimeneti állományok feltöltése szükséges.

Ha a kimenet formailag hibás vagy az egyenesek nem megfelelően vágják részekre a síkot, akkor nulla pontot kapsz a tesztesetre.

Különben pontszámodat az alábbi képlet alapján számolják:

$$ 10 \cdot \left(1 - \sqrt{1 - L_{min} / L}\right)$$,

ahol $L$ az egyenesek száma a megoldásodban, $L_{min}$ pedig a versenyzők által adott megoldások legjobbika. Ez utóbbit a verseny után állapítják meg.

\pagebreak

\heading{Példa bemenet és kimenet}

\sampleIN
4
3 1
4 5
6 6
8 4
\sampleOUT
2
3 2 8 7
2 8 8 2
\sampleEND

\end{document}
