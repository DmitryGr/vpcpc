\input sys/inputs.tex

\begin{document}

\bigheading{Szlalom}

% \info{task_name}{infile}{outfile}{points}{timelimit}{memlimit}
% leave this values, if you are not interested
\info{slalom}{stdin}{stdout}{100}{300 ms}{1 GB}

Hubert kezdő síelő, olyannyira, hogy kanyarodni sem tud, csak egyenesen menni. Egy szlalom pályán szeretne lecsúszni. De lehet, hogy szerencséje van, mert a pálya tervezője talán nem volt elég körültekintő, és olyan pályát tervezett, amelyen kanyarodás nélkül is végig lehet menni.

\heading{Feladat}
Adott egy szlalompálya $n$ darab kapuval. A pálya balról jobbra halad. Minden egyes kaput a két kapurúd közötti függőleges szakasszal ábrázoljuk. Madártávlatból Hubertet egy $d$ átmérőjű körnek tekinthetjük ($d \geq 0$), és e kör középpontja egy egyenes vonalát követi. Hubert bármely olyan pontból indulhat, amely a legbaloldalibb kaputól balra van, és bármely olyan pontba érkezhet, amely a legjobboldalibb kaputól jobbra van. A pályán való végighaladáshoz Hubertnek teljes terjedelmével át kell haladni az összes kapun. A kapurudak érintése megengedett.

Számítsd ki azt a legnagyobb $d$ átmérőt, amellyel Hubert végig tud menni a pályán!  

\heading{Bemenet}
A bemenet első sora egyetlen egész számot tartalmaz, a kapuk $n$ számát ($1 \leq n \leq 10^5$). A következő $n$ sor mindegyike egy-egy kapu adatait tartalmazza három egész számmal: $x, y_1, y_2$ ($0 \leq x \leq 10^9$,
$0 \leq y_1 \leq y_2 \leq 10^9$), ahol $[x, y_1]$ és $[x, y_2]$ a két kapurúd koordinátája. A kapuk $x$ koordinátái különbözőek.

\heading{Kimenet}
A kimenetre a legnagyobb olyan $d$ számot kell kiírni, amellyel Hubert végig tud menni a pályán. A kiírt eredményt akkor fogadják el, ha a helyes megoldástól való eltérése kisebb, mint $10^{-9}$. C++-ban a megoldást a következőképpen lehet kiírni:
\verb!printf("%.10lf\n", d);!
ahol a $d$ egy double típusú kifejezés.

Ha nincs olyan nemnegatív $d$ érték, amellyel a feladat megoldható, akkor a kimenetre az \texttt{Impossible} szöveget kell írni.

\heading{Példa bemenet és kimenet}

\sampleIN
3
4 3 7
6 6 9
1 5 10
\sampleOUT
1.3728129460
\sampleEND

\sampleIN
2
3 7 9
10 4 4
\sampleOUT
0.0000000000
\sampleEND

\sampleIN
3
0 4 7
2 0 3
4 4 7
\sampleOUT
Impossible
\sampleEND


\end{document}
