\input sys/inputs.tex

\begin{document}

\bigheading{Újságkihordó}

% \info{task_name}{infile}{outfile}{points}{timelimit}{memlimit}
% leave this values, if you are not interested
\info{newspapers}{stdin}{stdout}{100}{2000 ms}{64 MB}

Mirkow városa $n$ útkereszteződést tartalmaz, amelyeket $n-1$ kétirányú utca köt össze. Bármely út\-ke\-resz\-te\-ző\-dés\-ből bármely másikba el lehet jutni az utcákon keresztül.

Mirek újságkihordóként dolgozik a városban. Minden utcáról tudja, hogy hányan laknak benne, akik újságot is rendelnek. Mirek minden reggel választ két kereszteződést, és a két kereszteződést összekötő legrövidebb útvonalon lakóknak viszi ki az újságot. Mirek napi bevétele arányos a kézbesített újságok átlagos számával, ahol az átlag a kézbesített újságok száma osztva az útvonalon lévő utcák számával. Mirek főnöke előírta, hogy csak olyan útvonalat választhat, amely legalább $k$ utcát tartalmaz.

Segíts Mireknek kiválasztani a legtöbb bevételt eredményező útvonalat!

\heading{Feladat}
Írj programot, amely meghatározza a lehetséges legnagyobb bevételt olyan útvonalon, amely legalább $k$ utcát tartalmaz!

\heading{Bemenet}
A bemenet első sora két egész számot tartalmaz, $n$-et és $k$-t ($1 \le n \le 100\,000$, $1 \le k \le n - 1$), ahol $n$ a kereszteződések száma és $k$ Mirek útvonalának minimális hossza. A következő $n-1$ sor mindegyikében három egész szám van, $a$, $b$ és $c$ ($1 \le a < b \le n$, $0 \le c \le 10^6$), ami azt jelenti, hogy az $a$ és $b$ kereszteződést utca köti össze, és ezen $c$ számú újságot kell kihordani.

Feltételezhető, hogy létezik legalább egy $k$ hosszú útvonal.

\heading{Korlátok}
A tesztesetek $30\%$-ában $n \le 1\,000$.

\heading{Output}
A kimenetre egyetlen valós (double) számot kell kiírni: egy optimális útvonalon kézbesített újságok átlagos számát, amely legalább $k$ utcát tartalmaz.
A kiírt értéket akkor fogadják el, ha a helyes megoldástól való eltérése kisebb, mint $10^{-6}$ .

\heading{Példa bemenet és kimenet}

\sampleIN
5 2
1 2 4
2 3 1
3 4 3
3 5 3
\sampleOUT
3.00000000
\sampleEND

\end{document}
