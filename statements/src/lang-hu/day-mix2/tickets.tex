\input sys/inputs.tex

\begin{document}

\bigheading{Tickets}

% \info{task_name}{infile}{outfile}{points}{timelimit}{memlimit}
% leave this values, if you are not interested
\info{tickets}{stdin}{stdout}{100}{800 ms}{32 MiB}

Az év meccsét a következő héten játsszák. A stadionban $N$ ülőhely van, ezeket az $1, \ldots, N$ számokkal azonosítjuk. Minden szurkoló egy jegyet igényelhet, és megadhatja az ülőhelyek azon tartományát, amelyen belül ülni szeretne. A tartományt az intervallum $F$ első és $L$ utolsó ülőhelyének sorszámával adja meg. Tehát a szurkoló elfogad minden olyan $S$ ülőhelyet, amelyre teljesül, hogy $F \leq S \leq L$. A szervező $M$ igényt kapott a szurkolóktól, és meg szeretné határozni az egyszerre teljesíthető legtöbb igényt.

\heading{Feladat}
Írj programot, amely kiszámolja a legtöbb egyszerre teljesíthető igények számát, és meg is ad egy megfelelő ülőhely kiosztást! Természetesen nem ülhet egy helyre két szurkoló.

\heading{Bemenet}
A bemenet első sora két egész számot tartalmaz, az ülőhelyek $N$ számát ($1 \leq N \leq 100000$) és az igények $M$ számát ($1 \leq M \leq 1000000$). Az ülőhelyeket az $1, \ldots, N$ számokkal azonosítjuk. A következő $M$ sor mindegyike egy-egy igényt ad meg $F$ és $L$ egész számokkal ($1 \leq F \leq L \leq N$). Az igényeket az $1,\ldots,M$ számokkal azonosítjuk bemenetbeli sorrendben.

\heading{Kimenet}
A kimenet első sora a kielégített igények maximális $K$ számát tartalmazza. A következő $K$ sor mindegyike egy ülőhely kiosztást ad meg két egész számmal: $S\,R$, ahol $S$ egy ülőhely sorszáma és $R$ annak az igénynek a sorszáma, amely az $S$ ülőhelyet kapja a kiosztásban. Az ülőhely kiosztás tetszőleges sorrendben megadható. Ha több megoldás is van, akkor bármelyik megadható.

\heading{Példa bemenet és kimenet}
\sampleIN
10 9
1 3
2 4
5 7
2 6
1 5
3 7
4 8
7 9
3 8
\sampleOUT
9
1 1
2 5
3 2
4 4
5 6
6 9
7 3
8 7
9 8
\sampleCOMMENT

\sampleEND

\end{document}
