\input sys/inputs.tex

\begin{document}

\bigheading{Küldetés}

% \info{task_name}{infile}{outfile}{points}{timelimit}{memlimit}
% leave this values, if you are not interested
\info{mission}{stdin}{stdout}{100}{1000 ms}{1 GB}

\noindent
Figyelem, katona!

\bigskip \noindent
Fontos feladat vár rád! Azonosítottunk egy ellenséges bázist, amit meg kell semmisítened! Kapsz egy térképet és elég robbanóanyagot az elpusztításához. Az akció elvégzése után egy helikopter vár rád az erdő szélén.

\bigskip \noindent
Könnyűnek hangzik, ugye? Találd meg a leggyorsabb útvonalat a küldetés teljesítéséhez úgy, hogy ugyanarra a helyre egynél többször nem mehetsz, különben felfedeznek!

\bigskip \noindent
Minden világos? Akkor... készülj, mert 10 perc múlva indulsz!

\bigskip \noindent
Sok szerencsét, és élve gyere vissza a vacsorára!

\heading{Feladat}
Adott egy irányítatlan gráf és három különböző pontja: a kiindulási bázis, az ellenséges bázis és a helikopter helye. Add meg annak a legrövidebb útnak a hosszát, amely a kiindulási bázistól a helikopter helyéig halad, és keresztül megy az ellenséges bázison úgy, hogy nem mehet keresztül kétszer egyetlen ponton sem!

\heading{Bemenet}
A bemenet első sora öt, szóközökkel elválasztott egészt tartalmaz: $N$, $M$, $B$, $E$, $H$ ($1 \le B, E, H \le N$, $B \neq E \neq H \neq B$). $N$ a gráf pontjainak száma, amelyeket az $1, \ldots, N$ számokkal azonosítunk. $M$ a gráf éleinek száma. $B$ a kiindulási bázis, $E$ az ellenséges bázis, $H$ pedig a helikopter helyének megfelelő pont azonosítója.

A következő $M$ sor mindegyike három egész számot tartalmaz: $v$, $w$ és $t$ ($1 \le v, w \le N$, $v \neq w$, $1 \le t \le 1\,000\,000$). Ez azt jelenti, hogy a $v$ és $w$ pontokat irányítatlan él köti össze, és ezen az élen $t$ idő alatt lehet az egyik pontból a másikba eljutni.

Két pont között legfeljebb egy él lehet.

\heading{Korlátok}
$3 \le N \le 1\,000$ és $0 \le M \le 1\,000$.

A tesztesetek $30\%$-ában $N\le 20$.

\heading{Kimenet}
A kimenetre egyetlen egész számot írj ki, a küldetés teljesítéséhez szükséges minimális időt! 

Ha a küldetés teljesítése lehetetlen, akkor a $-1$ számot kell kiírni.

\heading{Példa bemenet és kimenet}

\sampleIN
3 2 1 2 3
1 2 10
2 3 20
\sampleOUT
30
\sampleEND


\bigskip


\sampleIN
3 0 2 1 3
\sampleOUT
-1
\sampleEND


\bigskip


\sampleIN
4 4 3 2 4
2 3 5
3 1 1
1 4 1
2 4 100
\sampleOUT
105
\sampleEND

\end{document}
