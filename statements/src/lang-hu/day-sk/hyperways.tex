\input sys/inputs.tex

\begin{document}

\bigheading{Hipersztráda}

% \info{task_name}{infile}{outfile}{points}{timelimit}{memlimit}
% leave this values, if you are not interested
\info{hyperways}{stdin}{stdout}{100}{3000 ms}{1 GB}

Oods\&co intergalaktikus vállalat  hipersztrádát kíván építeni,
%\footnote{hipersztráda, röviden hyhi}
amely kapcsolatot létesít bolygók között. Már elkészítették az építési tervet (amely megadja a sztráda szakaszok építésének a sorrendjét). A következő feladat annak eldöntése, hogy a megépítendő sztráda teljesíti-e a biztonsági előírást. Egy sztráda szakasz két bolygó között biztonságos, ha nem ez az egyetlen összeköttetés a két bolygó között (azaz a két bolygó rajta van egy körön).

\heading{Feladat}

Adva van a sztráda szakaszok építésének a sorrendje. Minden egyes szakasz megépítése előidézheti egy vagy több szakasz biztonságossá válását. A programod írja ki minden újonnan épített szakasz után, hogy hány szakasz vált biztonságossá ezen szakasz megépítésével!\\
Természetesen egy szakasz megépítésével csak nem biztonságos válhat biztonságossá, fordítva nem lehet.

\heading{Bemenet}

A bemenet első sora két egész számot tartalmaz $n$ és $m$.
$n$ a bolygók száma, $m$ pedig a sztráda szakaszok száma.
A bolygókat az $1$ \dots $n$ számokkal azonosítjuk.
A további $m$ sor mindegyike két egész számot tartalmaz egy szóközzel elválasztva, azon két bolygó azonosítóját, amely között sztráda szakasz épül. A sztráda szakaszok kétirányú közlekedést biztosítanak. Két bolygó között lehet több sztráda szakasz is, továbbá lehet önmagát összekötő is.

\smallskip

 $n \leq 10^6$, $m \leq 2\cdot10^6$.\\
A tesztesetek $40\%$-ában $n \leq 1000$, $m \leq 2000$.

\heading{Kimenet}
A kimenet $m$ sort tartalmazzon, amennyi a sztráda szakaszok száma. Az $i$-edik sorban az $i$-edik szakasz megépítésével keletkező biztonságos szakaszok száma legyen (beleértve az $i$-edik szakaszt is)!

\heading{Minta bemenet és kimenet}


\sampleIN
5 8
1 2
3 3
4 5
2 3
4 5
3 4
4 1
5 2
\sampleOUT
0
1
0
0
2
0
4
1
\sampleEND


\end{document} 