\input sys/inputs.tex

\begin{document}

\bigheading{Házak árnyékai}

% \info{task_name}{infile}{outfile}{points}{timelimit}{memlimit}
% leave this values, if you are not interested
\info{shades}{stdin}{stdout}{100}{1000 ms}{1 GB}
Egy város házai egyetlen sorban, egymástól azonos távolságban helyezkednek el.
Az épületek árnyékképének magasságait megmérték és feljegyezték, az eredmény egy egész számokat tartalmazó számsorozat.
Az árnyékkép-sorozat elemei nem a házak valódi magasságát tartalmazzák, hanem torzítottak, de mindegyik azonos mértékben torzított (pl. mindegyik $t$-szerese a valódinak, azaz a torzítás mértéke $t$).\\
Egy adott mintának az összes előfordulását keressük.

\heading{Feladat}

Adott egy árnyékkép-sorozat és több minta. Minden minta szintén egész számokból álló sorozat.
Azt mondjuk, hogy a minta előfordul az árnyékkép-sorozatban, ha az árnyékkép-sorozatnak van olyan összefüggő részsorozata, hogy valamely $t$ torzítást erre alkalmazva az megegyezik a mintával.
Egy minta többször is előfordulhat az árnyékkép-sorozatban, átfedés is megengedett.


\heading{Bemenet}

A bemenet első sora egy $n$ egész számot tartalmaz -- a minták számát.
A következő $n$ sor mindegyike egy minta leírását tartalmazza. A sorban az első egész szám $l_i$ -- a minta hossza (a sorozat elemeinek a száma). Ezt követi egy-egy szóközzel elválasztva $l_i$ pozitív egész szám, a minta elemei.

A bemenet utolsó sora tartalmazza az árnyékkép-sorozatot. A sorban az első szám $m$ az árnyékkép hossza (a sorozat elemeinek a száma). Ezt követi egy-egy szóközzel elválasztva $m$ pozitív egész szám, az árnyékkép-sorozat elemei.

\bigskip

Constraints:
\begin{itemize}\itemsep -2pt
\item
$1 \leq m \leq 3\cdot10^5$
\item
$1 \leq l_i$
\item
$\sum^{n}_{i=1} l_i \leq m$
\item
All heights and lengths are between 1 and $10\,000$, inclusive.
\end{itemize}


%\noindent Korlátok:\\
%$1 \leq m \leq 3\cdot10^5$\\
%$1 \leq l_i$\\
%$\sum^{n}_{i=1} l_i \leq m$\\
%Minden sorozat minden eleme legfeljebb $10000$.

\smallskip

A tesztesetek $40\%$-ában $m \leq 1000$.

\heading{Kimenet}

A kimenet egyetlen egész számot tartalmazzon, a minták összes előfordulási számát az árnyékkép-sorozatban!

\heading{Minta bemenet és kimenet}

\sampleIN
4
1 47
2 21 42
2 34 17
3 1 2 1
7 3 6 3 6 12 6 3
\sampleOUT
15
\sampleCOMMENT
Az első minta 7-szer fordul elő, mert az árnyékkép-sorozat minden eleméhez van olyan $t$, hogy annak $t$-szerese megegyezik a minta egyetlen elemével.\\
A második és harmadik minta mindegyike 3-szor fordul elő.\\
A negyedik minta kétszer fordul elő, így jön ki a 15 végeredmény.
\sampleEND


\end{document} 