\input sys/inputs.tex

\begin{document}

\bigheading{Új fa}

% \info{task_name}{infile}{outfile}{points}{timelimit}{memlimit}
% leave this values, if you are not interested
\info{newtree}{stdin}{stdout}{100}{100 ms}{32 MiB}

A város parkjában egy új fát ültettek, amit a kertész a szokásos módon védeni akar. A módszer az, hogy kiválaszt három meglévő fát és ezeket körbefogja egy szalaggal úgy, hogy csak az új fa legyen a háromszögön belül. Nem lehet az új fa a háromszög oldalán sem.\\
A kertész a három közül már kiválasztott egy fát, a másik kettő kiválasztásában kell segíteni.

\heading{Feladat}
Írj olyan programot, amely meghatároz két olyan fát, hogy a már kiválasztott fával a három fa egy kívánt szabályos védelmi területet alkosson az új fa körül!

\heading{Bemenet}
A bemenet első sora két egész számot tartalmaz, a meglévő fák $N$ ($3 \leq N \leq 100000$) számát és a már kiválasztott fa $A$ ($1 \leq A \leq N$) azonosítóját. A fákat az $1,\ldots,N$ számokkal azonosítjuk, a bemenetbeli sorrendben. A második sor két egész számot tartalmaz, $x$-et és $y$-t, az új fa koordinátáit ($-1000000 \leq x,y \leq 1000000$). A következő $N$ sor mindegyike két egész számot tartalmaz: $x$ és $y$, egy meglévő fa koordinátáit ($-1000000 \leq x,y \leq 1000000$).\\
\smallskip
\noindent A tesztek $40 \%$-ában $N \leq 5000$.
\heading{Kimenet}
A kimenet első sora két egész számot tartalmazzon: $B$ és $C$, két meglévő fa azonosítóját. Teljesülni kell, hogy a három fa, $A$ (a már kiválasztott fa), $B$ és $C$ az órajárással ellentétes felsorolásban olyan háromszöget alkot, amely szabályos védelmi terület az új fa körül. Tehát teljesül, hogy az új fa szigorúan a háromszögön belül van, továbbá $A$, $B$ és $C$ kivételével egyetlen más fa sem esik a háromszögbe, vagy annak oldalára.\\
Több megoldás esetén bármelyik megadható.\\
Ha nincs megoldás, akkor a $0\,\, 0$ számpárt kell kiírni.

\heading{Példa bemenet és kimenet}

\sampleIN
7 1
9 3
3 1
8 7
9 5
11 5
12 4
9 1
13 6
\sampleOUT
4 6
\sampleCOMMENT

\sampleEND
\bigskip

\end{document}
