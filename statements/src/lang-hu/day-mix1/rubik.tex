\input sys/inputs.tex

\begin{document}

\bigheading{Kockajáték}

% \info{task_name}{infile}{outfile}{points}{timelimit}{memlimit}
% leave this values, if you are not interested
\info{rubik}{stdin}{stdout}{100}{1000 ms}{1 GB}

Bob megtalálta régi Rubik kockáját, és játszani kezdett vele. Csinált néhány\footnote{$65000$} lépést. De a végeredmény nem tetszett neki. Rájött, hogy néhány\footnote{megint $65000$} lépését más lépésre kellene cserélni. Azonban nehéz megismételni az összes lépést újra.

\heading{Feladat}

Adott a Rubik kocka kezdeti állapota (nem biztos, hogy a kirakott állapot), és a Bob által eredetileg elvégzett lépések sorozata.

Adott továbbá a cserélendő lépések sorozata. Mindegyik cserelépés a következő formában adott: ``cseréld le a $k$. eredeti lépést erre az új lépésre''. Minden egyes cserére írd ki a kocka állapotát az \emph{összes lépés elvégzése után}.

Megjegyzendő, hogy a cserelépések tartósak, azaz például a második cserelépést a már elvégzett első cserelépés eredményét figyelembe véve kell elvégezni, nem pedig csak az eredeti sorozatnak az adott elemét lecserélve.


\heading{Kocka kezelése}

A kocka oldalainak színei legyenek $A$,$B$,$C$,$D$,$E$,$F$. A kockán elvégzett lépések hatására az oldalak közép\-pontjainak egymáshoz való elhelyezkedése nem változik. Tehát mindig tarthatjuk a kockát úgy, hogy a felső oldal középső elemének a színe $A$, a függőleges oldalaké $B$, $C$, $D$, $E$ (ebben a sorrendben), és az alsóé pedig $F$. A kocka felületének síkba kiterített oldalait az alábbi ábra mutatja a fix oldalközéppontokat jelölve:\\
\texttt{
???\\
?A?\\
???\\
????????????\\
?B??C??D??E?\\
????????????\\
???\\
?F?\\
???\\
}

\heading{Bemenet}

Az első 9 sor a kocka kezdeti állapotát adja meg a fenti kiterített formában. Feltételezhető, hogy az oldalak középpontjainak jelölése a fentieknek felel meg.\footnote{Szintén feltételezhető, hogy a kezdeti állapot a kocka egy lehetséges állapota. Ugyanakkor ez lényegtelen a megoldás szempontjából.}

A következő sorban két egész szám található, $n$ és $m$: $n$ a Bob által megtett lépések száma, $m$ pedig a cserelépések száma.

Az ezt követő $n$ sor Bob eredeti lépéseit írja le. Minden lépés ``$C_i$ $d_i$'' formában adott, ahol $C_i$ a mozgatott oldal középpontjának színe, és $d_i$ a forgatás irányát adja meg: $-1$, ha az óramutató járásával megegyező, $1$, ha vele ellentétes. A lépéseket az $1, \ldots, n$ számokkal azonosítjuk.

A következő $m$ sor mindegyike egy-egy cserelépést ad meg az elvégzendő sorrendben három értékkel, \mbox{``$a_j$ $C_j$ $d_j$''} formában, ahol $a_j$ a lecserélni kívánt eredeti lépés sorszáma, és $C_j$ $d_j$ az új lépést adja meg.

\heading{Korlátok}

\noindent Minden tesztesetre $n,m \leq 65\,000$.\\
A tesztesetek $50\%$-ában $n,m \leq 1000$.

\heading{Kimenet}

Legyen $S_i$ az a lépéssorozat, amelyet Bob eredeti lépéssorozatából úgy kapunk, hogy az első $i$ darab cserelépést helyettesítettük.

Minden egyes $i$-re 1-től $m$-ig 9 sort kell kiírni: a kocka kezdeti állapotára alkalmazott $S_i$ lépéssorozat elvégzése utáni állapotát. Használd ugyanazt a formátumot, mint ami a bemenetnél van megadva.

\heading{Példa bemenet és kimenet}

\sampleIN
AAA
AAA
AAA
BBBCCCDDDEEE
BBBCCCDDDEEE
BBBCCCDDDEEE
FFF
FFF
FFF
8 4
E 1
E -1
B 1
B -1
F 1
F -1
E 1
E -1
8 C -1
2 C -1
6 A -1
4 D -1
\sampleCOMMENT
Az eredeti lépések kioltják egymást. Az eredeti lépéssorozat végére a kocka az eredeti állapotába tér vissza.

A bemenetben megadott négy cserelépést elvégezve, olyan lépéssorozatot kapunk, amely mindent elforgat, kivéve az oldalak középpontjait.
\sampleOUT
DAD
DAD
DAD
ABACCCFDFEEE
ABACCCFDFEEE
ABACCCFDFEEE
BFB
BFB
BFB
FAF
FAF
FAF
DBDCCCBDBEEE
DBDCCCBDBEEE
DBDCCCBDBEEE
AFA
AFA
AFA
EDE
FAF
EDE
DCDFCFBEBAEA
ABADCDFDFBEB
DCDFCFBEBAEA
CBC
AFA
CBC
DDD
DAD
DDD
CCCFFFEEEAAA
CBCFCFEDEAEA
CCCFFFEEEAAA
BBB
BFB
BBB
\sampleEND


\end{document}