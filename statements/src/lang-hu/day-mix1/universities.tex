\input sys/inputs.tex

\begin{document}

\bigheading{Egyetemek}

% \info{task_name}{infile}{outfile}{points}{timelimit}{memlimit}
% leave this values, if you are not interested
\info{universities}{stdin}{stdout}{100}{1000 ms}{1 GB}

Bytelandiában $N$ mágikus egyetem van, mindegyik vagy fekete, vagy fehér mágiát tanít. $N-1$ kétirányú út köti össze az egyetemeket, minden út két különböző egyetemet köt össze. Az úthálózat olyan, hogy benne bármely két egyetem között pontosan egy útvonal létezik.

Szeretnél meglátogatni néhány egyetemet, amelyek egy útvonalra esnek.
Minden meglátogatott egyetemhez egy boldogságfaktor tartozik, amely érték megváltoztatja a boldogságodat. Pozitív boldogság faktor növeli, negatív pedig csökkenti a boldogságodat.

A célod, hogy olyan két különböző egyetem közötti útvonalat válasszál, amely hatására a legboldogabb leszel az út végére. Az útvonalra eső összes egyetemet meg kell látogatnod. Azonban csak olyan útvonalat választhatsz, amelyen ugyanannyi a fehér és a fekete mágiát oktató egyetemek száma (beleértve az útvonal két végpontját is). 


\heading{Feladat}

Írjál programot, amely kiszámítja azt a legnagyobb boldogság értéket, amely egy útvonalra eső egyetemek boldogságfaktorainak összege, és amely útvonalon a fehér és a fekete mágiát oktató egyetemek száma megegyezik.

% Az a feladatod, hogy olyan útvonalat válasszál, amelyen a fehér és a fekete mágiát oktató egyetemek száma megegyezik, és az útba eső egyetemek boldogságfaktorainak összege a lehető legnagyobb. 

\heading{Bemenet}

A bemenet négy sorból áll.

Az első sorban egy egész szám van, az egyetemek $N$ száma ($2 \leq N \leq 10^5$). Az egyetemeket $1, \ldots, N$ számokkal azonosítjuk.

A második sor egy $N$ darab, ``B'' vagy ``W'' karakterből álló szöveget tartalmaz. Az $i$. karakter adja meg, hogy az $i$. egyetem fekete (``B'') vagy fehér (``W'') mágiát oktat. Legalább egy fekete és legalább egy fehér egyetem van.

A harmadik sor $N$ egész számot tartalmaz egy-egy szóközzel elválasztva: $h_1 h_2 \ldots h_N$ ($-10^5 \leq h_i \leq 10^5$). A $h_i$ az $i$. egyetem boldogságfaktora.

A negyedik sor $N-1$ egész számot tartalmaz egy-egy szóközzel elválasztva: $v_1 v_2 \ldots v_{N-1}$. A $v_i$ azt adja meg, hogy a $v_i$ és az $(i+1)$ ($1 \leq v_i \leq i$) azonosítójú egyetemeket közvetlen út köti össze.

\heading{Kimenet}

A kimenet egyetlen egész számot tartalmaz, azt a legnagyobb boldogság értéket, amely egy útvonal során elérhető úgy, hogy azonos számú fehér és fekete mágiát oktató egyetemet tartalmaz az útvonal.

\heading{Példa bemenet és kimenet}


\sampleIN
6
BWBBBW
6 0 3 -2 100 5
1 2 2 4 4
\sampleOUT
9
\sampleCOMMENT
Az optimális útvonalra az 1,2,4,6 azonosítójú egyetemek esnek.
\sampleEND

\bigskip

\sampleIN
3
WBW
1 -10 5
1 2
\sampleOUT
-5
\sampleCOMMENT
Az eredmény negatív is lehet, mert legalább két egyetemet meg kell látogatnod.
\sampleEND

\end{document}
