\input sys/inputs.tex

\begin{document}

\bigheading{Játék}

% \info{task_name}{infile}{outfile}{points}{timelimit}{memlimit}
% leave this values, if you are not interested
\info{game}{stdin}{stdout}{100}{100 ms}{1 GB}

Mirek szeret a számokkal játszani. Barátjával, Kamillal, a következő játékot játsszák. Kezdetben adott két nem negatív egész szám: $A$ és $B$. Felváltva lépnek, minden lépésnél a soron következő játékos a nagyobbik számot cseréli le az alábbi két lehetőség szerint. Ha $A \leqslant B$, akkor

\begin{itemize}
	\item vagy lecseréli $B$-t $B - A^K$-ra, ahol $K$ tetszőleges olyan szám, amelyre teljesül, hogy $K > 0$ és $B - A^K \ge 0$;
	\item vagy lecseréli $B$-t $B\ mod\ A$-ra. 
\end{itemize}

Ha $B \leqslant A$, akkor $A$-t cseréli le úgy, hogy a fenti szabályokban $B$ és $A$ helyet cserél. Az a játékos nyer, akinek a lépése után bármelyik szám 0 lesz. Mindig Mirek kezd. Segíts neki megmondani, hogy adott $A$ és $B$ esetén ki az, aki nyer, feltéve, hogy mindketten optimálisan játszanak.

\heading{Feladat}
Írj programot, amely adott kezdeti játékállásokra kiszámolja, hogy Mirek vagy Kamil nyer-e!

\heading{Bemenet}
A bemenet első sora a játékok $T$ számát tartalmazza ($1 \le T \le 10^4$).
A következő $T$ sor mindegyike egy kezdeti játékállást ad meg, azaz $A$ és $B$ értékét ($1 \le A, B \le 10^{18}$).

\heading{Korlátok}
A tesztesetek $30\%$-ában $A,B \le 1000$.

\heading{Kimenet}
A kimenetre $T$ sort kell kiírni, ahány játékot játszottak. Az $i$. sorba annak a játékosan a nevét (\texttt{Mirek} vagy \texttt{Kamil}) kell írni, aki az $i$. játékban nyer.

\heading{Példa bemenet és kimenet}

\sampleIN
4
1 1
12 4
4 6
15 31
\sampleOUT
\texttt{Mirek}
\texttt{Mirek}
\texttt{Kamil}
\texttt{Mirek}
\sampleEND

\end{document}
