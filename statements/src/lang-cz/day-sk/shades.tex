\input sys/inputs.tex

\begin{document}

\bigheading{Stín města}

% \info{task_name}{infile}{outfile}{points}{timelimit}{memlimit}
% leave this values, if you are not interested
\info{shades}{stdin}{stdout}{100}{1000 ms}{1 GB}

Bylo, nebylo, bylo jednou jedno město.\footnote{Vyskytují se jisté pochybnosti o správnosti překladu.}
Tedy \dots předpokládejme, že to bylo město.
Každopádně, toto město zmizelo a nezanechalo po sobě vůbec nic.
Skoro nic. Jenom stín. Studený, tmavý stín.
Stín, který se nehýbe. Stín osudu.
Stín \dots OK, dost bylo povídání.
Jdeme kódit.
Potřebujeme zjistit výšky budov ve městě.
Stín však může být libovolně roztažený.
Naštěstí známe detaily o architektuře civilizace, která v této oblasti bydlela.

Město se skládalo z budov seřazených v řadě.
Stejně široké, stejně vzdálené. Pouze výšky měly různé.
Dokonce i ty měly svoje vzory.

Tvým úkolem bude najít všechny výskyty vzorů v stínu a vypsat jejich počet.

\heading{Úloha}

Stín je posloupnost celých čísel - naškálovaných výšek.
Vzor je také posloupnost celých čísel - výšek.
Řekneme, že vzor má výskyt ve stínu,
pokud existuje taková podposloupnost stínu,
že po jejím naškálování dostaneme vzor.
(Vzor může mít ve stínu i více výskytů)

\heading{Vstup}

První řádek vstupu obsahuje celé číslo $n$
-- počet vzorů.

Každý z následujících $n$ řádků popisuje jeden vzor.
Začíná jedním celým číslem $l_i$ -- délkou vzoru.
Po něm následuje $l_i$ mezerou odddělených kladných celých čísel.

Poslední řádek popisuje stín.
Začíná jedním celým čáslem $m$ -- délkou stínu.
Po něm následuje $m$ mezerou oddělených kladných celých čísel.

\bigskip

\noindent
Platí:

\begin{itemize}
  \item $1 \leq m \leq 3\cdot10^5$
  \item $1 \leq l_i$
  \item $\sum^{n}_{i=1} l_i \leq m$
\end{itemize}
Všechny členy posloupností jsou nejvíce $10\ 000$.

\smallskip

\noindent
Ve $40\%$ vstupů dokonce platí $m \leq 1000$.

\heading{Výstup}

Vypište jeden řádek s počtem všech výskytů
všech vzorů ve stínu.

\heading{Příklady}


\sampleIN
4
1 47
2 21 42
2 34 17
3 1 2 1
7 3 6 3 6 12 6 3
\sampleOUT
15
\sampleCOMMENT
První vzor může být naškálovaný na jakoukoliv výšku.
Proto se vyskytuje 7 krát. Druhý se třetím se
vyskytují 3 krát a poslední vzor 2 krát.
\sampleEND


\end{document}
