\input sys/inputs.tex

\begin{document}

\bigheading{Hypernice}

% \info{task_name}{infile}{outfile}{points}{timelimit}{memlimit}
% leave this values, if you are not interested
\info{hyperways}{stdin}{stdout}{100}{3000 ms}{1 GB}

Intergalaktická společnost Oods\&co vybuduje
během následujícího století v naší galaxii
sít hypernic.\footnote{hyperprostorové dálnice, takzvané hyhi}
Plány stavby (pořadí stavění dálnic) jsou již hotové.
Zbývá zkontrolovat zda jsou dodržené
směrnice pro bezpečnost výstavby hypernic.
A k tomu je potřeba zjistit změny v počtu
bezpečných hypernic během výstavby.
V rámci projektu rozvoje zaostalých planet
to budete programovat vy.

\heading{Úloha}

Hypernice je bezpečná, pokud neexistují 2 planety,
mezi kterými je jedinou cestou.
(Tedy, pokud je součástí nějakého hypernicového cyklu.)
Když přidáváme hypernice, počet těch bezpečných se může jedině zvýšit.
Dostanete pořadí, ve kterém stavíme hypernice.
Po každé stavbě řekněte, kolik hypernic se stalo bezpečnými.

\heading{Vstup}

První řádek obsahuje 2 celá čísla $n$, $m$.
$n$ je počet planet,
$m$ je počet hypernic.
Planety jsou číslované $1 \dots n$.

Každý z následujících $m$ řádků obsahuje 2
mezerou oddělená celá čísla.
Jsou to indexy planet, které spojuje další hypernice.
Všechny hypernice jsou obousměrné.
Mezi dvěma planetami může vézt více hypernic
a hypernice může vézt i z planety do ní samotné.

\bigskip

\noindent
Platí $n \leq 10^6$, $m \leq 2\cdot10^6$.\\
Ve $40\%$ vstupů dokonce platí
$n \leq 1000$, $m \leq 2000$.

\heading{Výstup}

Pro každou hypernici na vstupu vypište na samostatném řádku
kolik hypernic se jejím vybudováním stalo bezpečnými (včetně ní).

\heading{Příklady}


\sampleIN
5 8
1 2
3 3
4 5
2 3
4 5
3 4
4 1
5 2
\sampleOUT
0
1
0
0
2
0
4
1
\sampleEND


\end{document}
