\input sys/inputs.tex

\begin{document}

\bigheading{Dynamická alokace paměti}

% \info{task_name}{infile}{outfile}{points}{timelimit}{memlimit}
% leave this values, if you are not interested
\info{malloc}{stdin}{stdout}{100}{FIXME}{1 GB}

Kamila se opět jednou snaží vytvořit nový programovací jazyk, který strčí
všechny ostatní jazyky do kapsy. Programování v něm bude natolik jednoduché,
že to bude umět každé mimino dřív, než se vůbec naučí psát komentáře.
Tohle v žádném případě není bug\footnote{Mimochodem, je zcela nemožné napsat program s chybami --
Kamilin geniální kompilátor všechny bugy sám opraví.},
neboť komentáře budou zcela zbytečné: Kamila navrhla syntax tak šikovně, že každému
bude význam kterékoliv časti kódu okamžitě zřejmý. Jo, a ještě jedna věc -
kompilátor vám v den vašich narozenin upeče dort!

Několik detailů je ale potřeba ještě doladit. Například systém pro správu
paměti ještě nefunguje. Kamila již má připravenou specifikaci toho, jak se
paměť alokuje a uvolňuje, ale nestihla ji ještě implementovat. Pomozte ji s tím
a vstupte tak do dějin -- příležitost pracovat na tak významném projektu máte
jenom jednou za život!


\heading{Task}

Dostupná paměť je pole $n$ bajtů, očíslovaných od $0$ do $n-1$.
Na začátku jsou všechny bajty volné (tzn. nealokované).
Následně systém bajty alokuje a uvolňuje na základě příchozích požadavků.

Požadavek na \emph{alokaci} je zadaný jako celé číslo $\ell$.
Systém pak nájde souvislý blok $\ell$ volných bajtů, alokuje je a vrátí pozici
prvního bajtu v tom bloku. Pokud je víc takových bloků, systém zvolí ten,
který začíná na pozici s nejmenším číslem. Naopak, pokud žádný takový blok
neexistuje, systém požadavek zamítne a vrátí $-1$.

Požadavek na \emph{uvolnění} je zadaný jako dvě celá čísla $x$ a $\ell$.
Systém pak označí souvislý blok $\ell$ bajtů začínajíci na pozici $x$ jako
volný a vrátí počet skutečně uvolněných bajtů (tedy počet takových bajtů v
tom bloku, které byly před vykonáním požadavku alokované).


\heading{Input}

První řádek vstupu obsahuje dvě celá čísla $n$ a $q$ -- počet bajtů v dostupné
paměti a počet požadavků na systém ($1 \leq n, q \leq 3\cdot 10^5$).

Každý z nasledujících $q$ řádků obsahuje jeden požadavek.
První číslo na řádku určuje typ požadavku: $1$ znamená alokaci, $2$ znamená uvolnění.
Řádek s požadavkem na alokaci pak obsahuje ještě jedno celé číslo $\ell$ ($1 \leq \ell \leq n$).
Řádek s požadavkem na uvolnění pak obsahuje ještě dvě celá čísla $x$ a $\ell$ ($0 \leq x \leq n-1$, $1 \leq \ell \leq n-x$).


\heading{Output}

Pro každý požadavek ze vstupu vypište jeden řádek s hodnotou vrácenou systémem.


\heading{Samples}

\sampleIN
5 4
1 3
1 3
2 1 3
1 4
\sampleOUT
0
-1
2
1
\sampleEND


\end{document}
