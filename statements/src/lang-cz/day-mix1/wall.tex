\input sys/inputs.tex

\begin{document}

\bigheading{Hradby}

% \info{task_name}{infile}{outfile}{points}{timelimit}{memlimit}
% leave this values, if you are not interested
\info{wall}{stdin}{stdout}{100}{1000 ms}{128 MB}

Mirek se živí jako konzervátor.
Jeho prací je starat se o monumenty a udržovat je v dobrém stavu.
Mirkovým dnešním úkolem je opravit hradby starobylé pevnosti.
Hradby se již skoro rozpadají, takže by měl raději hned začít.
Aby se s tím nemusel pachtit sám, tak si na internetu zakoupil robota
speciálně navrženého k extrémně rychlé opravě takových opevnění.
Ukázalo se, že navrhnout optimální strategii na rekonstrukci nebude vůbec jednoduché.

Hradby si můžete představit jako přímku.
Mirek si již zapsal všechna místa na hradbách, která potřebují opravit.
Pro $i$-té takové místo navíc ví, že jeho oprava bude stát $C_i$
a dále koeficient $D_i$ -- jak moc cena roste, pokud místo není opravené okamžitě.
Pokud $i$-té místo je opravené po $t$ časových jednotkách,
tak celková cena jeho opravy je rovna:
\[ C_i + t \cdot D_i \]

\heading{Úloha}

Jsou zadány souřadnice míst na hradbách, které je třeba opravit,
a počáteční poloha robota.
  Navíc víte, že přesun robota z místa na souřadnici $x_1$ do místa na souřadnici $x_2$
zabere $|x_1 - x_2|$ jednotek času.
Spočítejte nejmenší cenu potřebnou k opravě všech míst, které to potřebují.
Můžete předpokládat, že samotné opravy zaberou zanedbatelně mnoho času.

\heading{Vstup}

Na první řádce vstupu jsou dvě celá čísla $N$ a $P$
($1 \le N \le 2\,000$, $0 \le P \le 10^9$)
-- počet míst na hradbách, která mají bý opravena, a souřadnice,
na které začíná robot
Náseleduje $N$ řádek, $i$-tý z těchto řádků popisuje $i$-té místo
a obsahuje tři celá čásla $X_i$, $C_i$ a $D_i$
($0 \le X_i \le 10^9$, $0 \le C_i, D_i \le 10^6$, $X_i \neq P$)
-- souřadnice $i$-tého místa a jeho výše popsané ceny.
Žádná dvě místa neleží na stejné souřadnici.

\heading{Výstup}

Vypište jeden řádek obsahující nejmenší možnou cenu,
za kterou je možné hradby opravit.

\heading{Příklad}


\sampleIN
3 7
10 32 1
3 5 1
14 0 2
\sampleOUT
72
\sampleCOMMENT
Optimální plán opravy je následující:
\begin{itemize}
	\item Přesuň robota ze souřadnice $7$ na souřadnici $10$ a nechej ho opravit první místo v čase $3$.
	\item Přesuň robota ze souřadnice $10$ na souřadnici $14$ a nechej ho opravit třetí místo v čase $3 + 4 = 7$.
	\item Přesuň robota ze souřadnice $14$ na souřadnici $3$ a nechej ho opravit druhé místo v čase $3 + 4 + 11 = 18$.
\end{itemize}
Celková cena takové opravy hradeb je tedy rovna\\
$32 + 3 \cdot 1 + 0 + 7 \cdot 2 + 5 + 18 \cdot 1 = 72$.
\sampleEND

\end{document}
