\input sys/inputs.tex

\begin{document}

\bigheading{RubiKubismus}

% \info{task_name}{infile}{outfile}{points}{timelimit}{memlimit}
% leave this values, if you are not interested
\info{rubik}{stdin}{stdout}{100}{1000 ms}{1 GB}

Moderní umění je nepředvídatelné.
Své o tom ví i Bob.
Minule si jenom uklízel pokoj
a našel starou Rubikovu kostku.
Zavřel oči, naslouchal svému vnitřnímu hlasu
a udělal pár\footnote{$65\,00$0} rychlých pohybů.
Vzniklo téměř dokonalé dílo.
Potřebovalo už jenom pár\footnote{znovu $65\,000$} úprav.
Přesněji, potřebovalo změnit některé z původních tahů.
Bob by však rád věděl, co jednotlivé změny vlastně dělají.
Ale nechce se mu opakovat celou posloupnost tahů pokaždé znovu.
Rozhodl se, že si napíše program, který mu práci trochu usnadní.

\heading{Úloha}

Dostanete Rubikovu kostku (ne nutně vyřešenou),
seznam Bobových tahů a posloupnost změn v tomto seznamu tahů.
Po každé změně vypište, jak by vypadala výsledná kostka
po použití všech tahů ze seznamu.

\heading{Formát kostky}

Barvy kostky označme $A$,$B$,$C$,$D$,$E$,$F$.
Když si hrajete s kostkou, nemění
se vzájemná poloha prostředních čtverečků stěn kostky.
Můžeme tedy vždy kostku natočit tak,
aby prostřední čtverečky měly fixní pozici.
Potom můžeme rozvinout povrch do roviny kostky a získáme
tak její jednoznačný popis:\\
\texttt{
???\\
?A?\\
???\\
????????????\\
?B??C??D??E?\\
????????????\\
???\\
?F?\\
???\\
}

\heading{Vstup}

Prvních $9$ řádků popisuje počáteční konfiguraci kostky
ve výše zmíněném formátu.

Následující řádek obsahuje $2$ celá čísla $n$ a $m$.
$n$ je počet tahů a $m$ je počet změn.

Dalších $n$ řádků popisuje původní Bobovy tahy.
Mají tvar $C_i$ $d_i$,
kde $C_i$ je barva středu té stěny kostky,
kterou během tahu otáčíme, a $d_i$ je rovno $-1$
pokud točíme po směru chodu hodinových ručiček, případně
rovno $1$ pokud točíme proti směru chodu hodinových ručiček.

Posledních $m$ řádků popisuje změny.
Mají tvar $a_j$ $C_j$ $d_j$,
kde $a_j$ je číslo tahu, který se nahrazuje
a $C_j$ $d_j$ popisují nový tah.
Tahy jsou číslované od $1$ do $n$.

\smallskip

Platí $n,m \leq 65000$
a v $50\%$ vstupů dokonce
$n,m \leq 1000$.

\heading{Výstup}

Po každé z $m$ změn vypište $9$ řádků
popisujících konečnou konfiguraci kostky.

\heading{Příklad}

\sampleIN
AAA
AAA
AAA
BBBCCCDDDEEE
BBBCCCDDDEEE
BBBCCCDDDEEE
FFF
FFF
FFF
8 4
E 1
E -1
B 1
B -1
F 1
F -1
E 1
E -1
2 C -1
4 D -1
6 A -1
8 C -1
\sampleCOMMENT
Původní tahy se navzájem vyřuší.
Postupně je měníme na známý trik
rotace středů.
\sampleOUT
DAD
DAD
DAD
ABACCCFDFEEE
ABACCCFDFEEE
ABACCCFDFEEE
BFB
BFB
BFB
CCC
DAD
CCC
AAABCBFFFDED
BBBFCFDDDAEA
AAABCBFFFDED
EEE
BFB
EEE
CDC
CAC
CDC
BCBFFFDEDAAA
BBBFCFDDDAEA
BCBFFFDEDAAA
EBE
EFE
EBE
DDD
DAD
DDD
CCCFFFEEEAAA
CBCFCFEDEAEA
CCCFFFEEEAAA
BBB
BFB
BBB
\sampleEND


\end{document}
