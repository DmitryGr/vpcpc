\input sys/inputs.tex

\begin{document}

\bigheading{RubiKubismus}

% \info{task_name}{infile}{outfile}{points}{timelimit}{memlimit}
% leave this values, if you are not interested
\info{rubik}{stdin}{stdout}{100}{1000 ms}{1 GB}

Moderní umění je nepředvídatelné.
Když si posledně Bob uklízel pokoj,
našel svojí starou Rubikovu kostku.
Zavřel oči, naslouchal svému vnitřnímu hlasu
a udělal pár rychlých pohybů (nejvýše $65\,000$).
Vzniklo téměř dokonalé dílo.
Bobovi se ale zase tak úplně nelíbilo.
Kdyby se tak jen mohl vrátit zpátky v čase a změnit některé své tahy!

Všechno, co musí teď udělat je pár změn (znovu nejvýše $65\,000$).
Každá změna odpovídá náhradě jednoho tahu za nějaký jiný.
Bob by však rád věděl, co jednotlivé změny vlastně dělají.
Ale nechce se mu opakovat celou posloupnost tahů pokaždé znovu.

\heading{Úloha}

Dostanete Rubikovu kostku (ne nutně vyřešenou).
Dále dostanete seznam původních Bobových tahů.

Na závěr dostanete seznam změn, které Bob plánuje provést.
Každá změna je tvaru ``změň $k$-tý tah na tento nový tah.''
Po každé změně vypište, jak by vypadala \textit{výsledná kostka
po použití všech tahů} ze seznamu.

Všimněte si, že změny přetrvávají -- například druhá změna
se provede na seznam tahů, který je již upravený první změnou.

\heading{Formát kostky}

Barvy kostky označme $A$,$B$,$C$,$D$,$E$,$F$.
Když si hrajete s kostkou, nemění
se vzájemná poloha prostředních čtverečků stěn kostky.
Postavíme kostku tedy vždy tak, aby střed vrchní stěny
kostky měl barvu $A$, středy bočních stěn postupně barvy $B$, $C$, $D$, $E$
a konečně střed spodní stěny barvu $F$.
Potom můžeme rozvinout povrch kostky do roviny a získáme
tak její jednoznačný popis:\\
\texttt{
???\\
?A?\\
???\\
????????????\\
?B??C??D??E?\\
????????????\\
???\\
?F?\\
???\\
}

\heading{Vstup}

Prvních $9$ řádků popisuje počáteční konfiguraci kostky
ve výše zmíněném formátu. Můžete předpokládat, že středy
stěn kostky mají takové barvy, jak je popsáno
výše.\footnote{Navíc můžete předpokládat, že počáteční konfigurace
je validní stav kostky, který můžeme získat z vyřešeného stavu.
Na druhou stranu to pro řešení úlohy není nijak důležité.}

Následující řádek obsahuje dvě celá čísla $n$ a $m$.
$n$ je počet tahů a $m$ je počet změn.

Dalších $n$ řádků popisuje původní Bobovy tahy.
Mají tvar ``$C_i$ $d_i$'',
kde $C_i$ je barva středu té stěny kostky,
kterou během tahu otáčíme, a $d_i$ je rovno $-1$
pokud točíme po směru chodu hodinových ručiček, případně
rovno $1$ pokud točíme proti směru chodu hodinových ručiček.

Posledních $m$ řádků popisuje změny.
Mají tvar ``$a_j$ $C_j$ $d_j$'',
kde $a_j$ je číslo tahu, který se nahrazuje
a $C_j$ $d_j$ popisují nový tah.
Tahy jsou číslované od $1$ do $n$.

\heading{Omezení}

Ve všech vstupech platí $n, m \leq 65\,000$.
V $50\%$ vstupů navíc $n, m \leq 1\,000$.

\pagebreak

\heading{Výstup}

Nechť $S_i$ je seznam tahů po použití prvních $i$ změn.

Pro každé $i$ od $1$ do $m$ vypište $9$ řádků popisujících konečnou konfiguraci kostky, pokud aplikujeme tahy $S_i$ na počáteční konfiguraci kostky. Pužijte stejný formát jako na vstupu.

\heading{Příklad}

\sampleIN
AAA
AAA
AAA
BBBCCCDDDEEE
BBBCCCDDDEEE
BBBCCCDDDEEE
FFF
FFF
FFF
8 4
E 1
E -1
B 1
B -1
F 1
F -1
E 1
E -1
8 C -1
2 C -1
6 A -1
4 D -1
\sampleCOMMENT
Původní tahy se navzájem vyruší. A na konci původního seznamu tahů
bude kostka zase v počátečním stavu.

Po provedení všech čtyř změn, získáme seznam tahů, který
přehodí vše, kromě středů stěn.
\sampleOUT
DAD
DAD
DAD
ABACCCFDFEEE
ABACCCFDFEEE
ABACCCFDFEEE
BFB
BFB
BFB
FAF
FAF
FAF
DBDCCCBDBEEE
DBDCCCBDBEEE
DBDCCCBDBEEE
AFA
AFA
AFA
EDE
FAF
EDE
DCDFCFBEBAEA
ABADCDFDFBEB
DCDFCFBEBAEA
CBC
AFA
CBC
DDD
DAD
DDD
CCCFFFEEEAAA
CBCFCFEDEAEA
CCCFFFEEEAAA
BBB
BFB
BBB
\sampleEND


\end{document}
