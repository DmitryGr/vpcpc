\input sys/inputs.tex

\begin{document}

\bigheading{Nový stromeček}

% \info{task_name}{infile}{outfile}{points}{timelimit}{memlimit}
% leave this values, if you are not interested
\info{newtree}{stdin}{stdout}{100}{100 ms}{32 MiB}

V městském parku Bytelandie byl zasazen nový kouzelný stromeček.
Jedná se o velmi vzácný druh, a proto ho chceme ochránit.
Vybereme tři již existující stromy a natáhneme okolo nich pletivo tak,
že výsledná ohrazená oblast tvoří trojúhelník s vybranými stromy jako vrcholy\footnote{Tloušťka stromu je zanedbatelná.}.
Je nutné aby se nový stromeček nacházel ostře uvnitř ohrazené oblasti a aby se v
ní žádný jiný strom nenacházel.

Jeden z trojice stromů pro stavbu ohrady již byl určen. Pomozte nám vybrat zbylé dva.

\heading{Úloha}

Vaší úlohou je najít další dva stromy pro stavbu ohrady tak, aby ohrazená oblast
obsahovala nový stromeček.

\heading{Vstup}

První řádek vstupu obsahuje dvě celá čísla, $N$ a $A$.
$N$ ($3 \leq N \leq 100\,000$) je počet již existujících stromů a $A$ ($1 \leq A \leq N$)
je číslo již vybraného stromu (stromy jsou číslované od $1$ do $N$).

Druhý řádek obsahuje dvě celá čísla $x$ a $y$, souřadnice nového stromečku.

Dále následuje $N$ řádků. $i$-tý z nich obsahuje dvě celá čísla $x$ a $y$, souřadnice stromu s číslem $i$.

Pro všechny souřadnice platí $-1\,000\,000 \leq x,y \leq 1\,000\,000$.

\smallskip

Ve $40 \%$ případů platí $N \leq 5\,000$.

\heading{Výstup}

Na výstup vypište dvě celá čísla $B$ a $C$ oddělená mezerou.
$B$ a $C$ jsou čísla již existujících stromů, pro která platí, že trojúhelník
s vrcholy na souřadnicích stromů $A$, $B$ a $C$ (proti směru hodinových ručiček) splňuje výše popsaná pravidla.
T.j. žádný jiný strom se nenachází na stranách trojúhelníku $A$, $B$ a $C$, a uvnitř trojúhelníku se nachází pouze nový stromeček.


Pokud žádné řešení neexistuje, tak na výstup vypište 0 0.
Pokud existuje více správných řešení, vypište libovolné z nich.

\heading{Příklady}

\sampleIN
7 1
9 3
3 1
8 7
9 5
11 5
12 4
9 1
13 6
\sampleOUT
6 4
\sampleCOMMENT

\sampleEND

\end{document}
