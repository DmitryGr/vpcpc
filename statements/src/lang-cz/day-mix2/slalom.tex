\input sys/inputs.tex

\begin{document}

\bigheading{Nezkušený lyžař}

% \info{task_name}{infile}{outfile}{points}{timelimit}{memlimit}
% leave this values, if you are not interested
\info{slalom}{stdin}{stdout}{100}{FIXME}{1 GB}

Tohle je Filipův velký den! Jeho první závod v slalomu na lyžích.

\bigskip \noindent
- ``Filipe, jsi nervózní?''\\
- ``Trochu.''
\bigskip \noindent

No, ještě aby nebyl nervózní, vždyť na lyžích ješte nikdy nestál.
Zátačky dělat neumí a jediné co dokáže je spustit se zhora dolů po rovné přímce.

Ale možná nic není straceno.
Je tu určitá šance, že mu jeho technika postačí na projetí celého závodu.

\heading{Task}

Dostanete popis tratě, na které je $n$ branek, kterými musí závodníci projít.
Směr tratě je zleva do prava.
Každá branka je reprezentovaná jako vertikální (při pohledu zhora) úsečka spojující dva body.
Z ptačí perspektivy vypadá Filip jako kruh s průměrem $d$ ($d \geq 0$) a
trajektorie jeho středu se pohybuje po rovné přímce.

Filip si na začátku může vybrat bod, ze kterého bude startovat, kdekoliv nalevo od nejlvější branky,
a bod, ve kterém skončí, kdekoliv napravo od nejprvější branky.
Aby Filip závod úspěšně projel, musí celým tělem projet hraničnými body každé branky.

Najděte největší průměr $d$, pro který je možné závod ůspěšně dokončit.


\heading{Input}

První řádek vstupu obsahuje jedno celé číslo $n$, počet branek ($1 \leq n \leq 10^5$).

Každý z následujících $n$ řádků popisuje jednu branku jako tři mezerou oddělená čísla $x, y_1, y_2$ ($0 \leq x \leq 10^9$,
$0 \leq y_1 \leq y_2 \leq 10^9$).
Branka je reprezentováná jako vertikální úsečka s krajními body na souřadnicích $[x, y_1]$ a $[x, y_2]$.
Žádné dvě branky nemají stejnou x-ovou souřadnici.

\heading{Output}

Vypište jediný řádek obsahující největší $d$ takové, že Filip s průměrem $d$ může závod dokončit.
Akceptujeme každou odpověď, která se od správné liší (absolutně nebo relativně) o méně než $10^{-9}$.

V C/C++, můžete odpověď vypsat pomocí \verb!printf("%.10lf\n", d);!

Pokud žadné takové nezáporné $d$ neexistuje, vypište jediný řádek se slovem \texttt{Impossible}.


\heading{Samples}

\sampleIN
3
4 3 7
6 6 9
1 5 10
\sampleOUT
1.3728129460
\sampleEND

\sampleIN
2
3 7 9
10 4 4
\sampleOUT
0.0000000000
\sampleEND

\sampleIN
3
0 4 7
2 0 3
4 4 7
\sampleOUT
Impossible
\sampleEND


\end{document}
