\input sys/inputs.tex

\begin{document}

\bigheading{Noviny}

% \info{task_name}{infile}{outfile}{points}{timelimit}{memlimit}
% leave this values, if you are not interested
\info{newspapers}{stdin}{stdout}{100}{4000 ms}{1 GB}

V nádherném polském městě Mirkowě je $n$ křižovatek propojených $n - 1$
obousměrnými cestami.
Je možné cestovat mezi každou dvojicí křižovatek použitím těchto cest.

Mirek v Mirkowě pracuje jako pošťák a doručuje noviny domorodcům.
Pro každou ulici je známé, kolik lidí v ní žije
(to je také počet novin, které má do té ulice doručit).
Každý den si Mirek zvolí dvě křižovatky a navštíví všechny cesty na nejkratší
cestě mezi němi. Mirkův denní příjem je úměrný průměrnému počtu novin, které
doručil na nějakou cetstu
(počet doručených novin dělený počtem projetých cest).

Nejprve chtěl Mirek přechytračit nadřízené doručením novin pouze do těch ulic,
kde žije nejvíce lidí, ale šéf brzy odhalil jeho strategii a rozhodl se ho zastavit,
protože spousta lidí takto vůbec nedostávala denní tisk.
Šéf dal tedy Mirkovi dodatečnou podmínku: musí během své cesty navštívit
alespon $k$ různých ulic.
Pomozte Mirkovi nalézt cestu, jejíž použitím vydělá co nejvíce peněz.

\heading{Úloha}

Je zadán popis Mirkowa, nalezněte cestu obsahující alespoň $k$ ulic,
během které Mirek doručí největší průměrný počet novin na ulici.

\heading{Vstup}

Na první řádce vstupu se nacházejí dvě celá čísla $n$ a $k$
($1 \le n \le 50\,000$, $1 \le k \le n - 1$)
-- počet křižovatek v Mirkowě a nejmenší povolenou délku Mirkovy cesty.
Na následujících $n-1$ řádkách je popis ulic.
Každá ulice je reprezentována třemi celými čísly $a$, $b$ a $c$
($1 \le a < b \le n$, $0 \le c \le 10^6$)
-- znamenají, že ve městě se nachází cesta spojující křižovatky $a$ a $b$. V takové ulici žije $c$ rodin.

Můžete předpokládat, že v zadaném měste existuje cesta délky $k$.
\smallskip
Ve $30\%$ testovacích dat platí, že  $n \le 1\,000$.

\heading{Výstup}

Vypište jediné číslo -- průměrné množství doručených novin do ulice při použití
optimální cesty.
Výstup bude považován za správný, pokud jeho vzdálenost od správného řešení
je menší než $10^{-6}$.

\heading{Příklad}

\sampleIN
5 2
1 2 4
2 3 1
3 4 3
3 5 3
\sampleOUT
3.00000000
\sampleEND

\end{document}
