\input sys/inputs.tex

\begin{document}

\bigheading{Vyšetřování}

% \info{task_name}{infile}{outfile}{points}{timelimit}{memlimit}
% leave this values, if you are not interested
\info{investigation}{stdin}{stdout}{100}{1000 ms}{1 GB}

Ve městě Bytelandia se stala loupež!
Zloději se podařilo utéct and ukrýt někde ve městě.
Vy tento zločin vyšetřujete. Vaším ůkolem je nalézt a uvěznit zloděje.

Město se skládá z $N$ domů a $N-1$ cest. Cesty spojují některé dvojice domů
takovým spůsobem, že mezi každou dvojicí domů existuje právě jedna unikátní cesta
(tzn. město má stromovou strukturu).
Zloděj se ukrývá v jednom z domů.

Aby jste zloděje nalezli, můžete si zvolit dům $h$ a celý jej prohledat.
Pokud se v tom domě zloděj nacházel, zatknete ho.
V opačnem případě po drastickém vyslíchání všech obyvatel domu se dozvíte následující informaci:
``Pokud si představíte město jako zakořeněný strom, s domem $h$ jako kořenem a domy $c_1, c_2 \ldots c_m$ jako jeho syny,
pak se zloděj ukrýva v jednom z domů v podstromu zakořeněném v $c_i$ (pro $i$, $1 \leq i \leq m$).''

Domy musíte prohledávat až dokud zloděje nenajdete.
Můžete předpokládat, že se zloděj ukrýva ve stejném domě po celou dobu vyšetřování (tzn. zloděj nemění svou lokaci).

Je zřejmé, že na pořadí, ve kterém domy prohledáváte záleží, protože i když v nějakém
domě zloděje nenajdete, tak informace, kterou po vyslíchání získate může značně
omezit počet domů, ve kterých se zloděj může nacházet.
Potřebujete tedy vymyslet optimální strategii, se kterou minimalizujete počet
prohledaných domů v nejhorším možném případě.

\heading{Úloha}

Na vstupu dostanete popis města.
Vymyslete strategii pro prohledávání domů, tak aby jste minimalizovali
počet prohledaných domů v nejhorším možném případě.

\heading{Vstup}

První řádek obsahuje jedno celé číslo $N$, počet domů ve městě (domy jsou očíslované od $0$ do $N-1$).

Druhý řádek obsahuje $N-1$ mezerou oddělených čísel, $v_1 v_2 \ldots v_{N-1}$.
Číslo $v_i$ ($1 \leq i \leq N-1$) znamená, že existuje cesta, která vede medzi domy s čísly $v_i$ a $i$ ($v_i < i$).

\smallskip
\noindent
Platí $2 \leq N \leq 10^5$.\\
Ve $20\%$ případů $N \leq 10$.\\
Ve $40\%$ případů $N \leq 20$.\\
Ve $60\%$ případů $N \leq 1000$.

\heading{Výstup}

Na výstup vypište právě jedno číslo, počet domů, které je nutné prohledat v
nejhorším možném případě, když budete prohledávat podle optimálni strategie.

\heading{Příklad}

\sampleIN
5
0 1 1 1
\sampleCOMMENT
Struktura města vypadá jako hvězda, s domem č. 1 ve středu.
\sampleOUT
2
\sampleCOMMENT
Jako první prohledejte dům č. 1.
Pokud tam zloděje nenajdete, tak po vyslíchání už budete vědět, ve kterém domě se nachází.
\sampleEND


\bigskip


\sampleIN
8
0 1 2 1 3 5 6
\sampleOUT
3
\sampleEND


\end{document}
