\input sys/inputs.tex

\begin{document}

\bigheading{Autobusové linky}

% \info{task_name}{infile}{outfile}{points}{timelimit}{memlimit}
% leave this values, if you are not interested
\info{buslines}{stdin}{stdout}{100}{1000 ms}{1 GB}

V Kocourkově mají skutečně pozoruhodný systém hromadné dopravy.
Skládá se z $N$ autobusových zastávek a $N-1$ obousměrných cest mezi nimi. Každá cesta spojuje dvě zastávky.
Navíc je možné se dostat z každé autobusové zastávky do všech ostatních za použítí nějaké posloupnosti cest.

Pro každou dvojici různých autobusových zastávek $a$ a $b$ platí, že každé ráno
vyrazí jeden autobus ze zastávky $a$ a pojede do zastávky $b$ (po jediné možné cestě).
Celkově tedy vyrazí $N(N-1)$ autobusů.
Každý autobus zastaví na všech zastávkách, které během své cesty navštíví.

Na zastávkách musí být jízdní řád obsahující všechny autobusy, které na nich zastavují.
Zajímalo by nás, kolik autobusů je na každém z jízdních řádů napsáno.

\heading{Úloha}

Je zadán popis systému hromadné dopravy v Kocourkově.
Pro každou autobusovou zastávku spočítejte kolik autobusů na ní zastavuje.

\heading{Vstup}

První řádek vstupu obsahuje celé číslo $N$, počet autobusových zastávek ve městě (jednotlivé zastávky jsou číslované od $1$ do $N$).
Následujících $N-1$ řádek popisuje cesty ve městě.
Každý takový řádek obsahuje dvě různá celá čísla $1 \le x, y \le N$ znamenající, že cesta spojuje autobusové zastávky $x$ a $y$.

\bigskip
\noindent
Platí, že $1 \leq N \leq 10^6$.\\
Ve $20\%$ vstupů navíc platí, že $N \leq 100$.\\
Ve $40\%$ vstupů navíc platí, že $N \leq 1000$.

\heading{Výstup}

Výstup se skládá z $N$ řádek. $i$-tý řádek obsahuje jedno celé číslo -- počet autobusů, který zastavuje na $i$-té zastávce.

\heading{Příklady}

\sampleIN
6
1 2
2 3
3 4
4 5
5 6
\sampleOUT
10
18
22
22
18
10
\sampleEND

\bigskip

\sampleIN
5
4 5
2 1
3 2
2 5
\sampleOUT
8
18
8
8
14
\sampleEND

\end{document}
