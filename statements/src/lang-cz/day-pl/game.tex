\input sys/inputs.tex

\begin{document}

\bigheading{Hra}

% \info{task_name}{infile}{outfile}{points}{timelimit}{memlimit}
% leave this values, if you are not interested
\info{game}{stdin}{stdout}{100}{1000 ms}{64 MB}

Alík má čísla hrozně rád a ještě raději má hry s čísly.
Spolu se svím kamarádem Bubákem, jednu takovou hru hrají.
Na začátku hry dostanou dvě nezáporná čísla $A$ a $B$.
Pokud $A \leqslant B$, hráči můžou udělat jeden z dvou tahů:

\begin{itemize}
    \item Nahradit $B$ číslem $B - A^K$. Číslo $K$ si může hráč zvolit, přičemž musí platit $K > 0$ a $B - A^K \ge 0$.
    \item Nahradit $B$ číslem $B\ mod\ A$.
\end{itemize}

Pokud $B \leqslant A$, tak lze použít obdobné tahy.
Hráč, který jako první změní jedno z čísel na $0$, vyhrál.

Alík má první tah. I když má tuhle hru hodně rád, tak vyhrávání má ještě raději.
Pomozte mu rozhodnout, který z hráčů vyhraje, pokud budou oba hráči hrát optimálně.

\heading{Task}

Dostanete popis několika her mezi Alíkem a Bubákem.
Pro každou hru rozhodněte, který z hráčů vyhraje.

\heading{Input}

První řádek obsahuje jedno celé číslo $T$ ($1 \le T \le 10^4$), počet her.

Následujících $T$ řádek obsahuje popis her. Každý řádek obsahuje dvě celá čísla $A$ a $B$ ($1 \le A, B \le 10^{18}$).

\smallskip

Ve $30\%$ případů platí $A,B \le 1\,000$.

\heading{Output}

Na výstup vypište $T$ řádek.
Na $i$-tý řádek vypište \texttt{Mirek}, pokud $i$-tou hru vyhraje Alík. V opačnem případě, pokud vyhraje Bubák, vypište \texttt{Kamil}.

\heading{Samples}

\sampleIN
4
1 1
12 4
4 6
15 31
\sampleOUT
\texttt{Mirek}
\texttt{Mirek}
\texttt{Kamil}
\texttt{Mirek}
\sampleEND

\end{document}
