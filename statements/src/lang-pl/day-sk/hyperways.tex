\input sys/inputs.tex

\usepackage[utf8]{inputenc}
\usepackage[T1]{fontenc}
\usepackage[polish]{babel}
\usepackage{polski}

\begin{document}

\bigheading{Hyperways}

% \info{task_name}{infile}{outfile}{points}{timelimit}{memlimit}
% leave this values, if you are not interested
\info{hyperways}{stdin}{stdout}{100}{3000 ms}{1 GB}

Międzygalaktyczna firma Oods\&co ma plan zbudować sieć dróg ekspresowych,
	łączących niektóre planety naszej galaktyki.
Kolejność budowy dróg została już ustalona.
Zostaje tylko sprawdzić, czy plan spełnia międzygalaktyczne warunki bezpieczeństwa.
W tym celu, firma Oods\&co musi obliczyć liczbę bezpiecznych dróg po każdym etapie budowy.
Twoim zadaniem będzie napisać program, który to zrobi.

\heading{Task}

Droga jest bezpieczna, jeśli nie jest jedyną drogą łączącą pewne dwie planety (czyli leży na pewnym cyklu).
Masz daną kolejność budowy dróg ekspresowych.
Po każdej wybudowanej drodze, niektóre drogi mogą stać się bezpieczne.
Wypisz ich liczbę.
Zauważ, że kiedy tylko dodajesz drogi, to niebezpieczne drogi mogą stać się bezpieczne, ale nie na odwrót!

\heading{Input}

W pierwszej linii wejścia znajdują się dwie liczby całkowite $n$ i $m$,
	oznaczające liczbę planet i liczbę dróg w~planie firmy.
Planety są ponumerowane od $1$ do $n$.

Kolejne $m$ linii zawiera dwie liczby całkowite oddzielone spacją
	-- numery planet, które połączy kolejna droga.
Wszystkie drogi są dwukierunkowe.
Może wystąpić droga łącząca pewną planetę z nią samą.
Może być wiele dróg łączących te same dwie planety.

Ograniczenia: $n \le 10^6$, $m \le 2 \cdot 10^6$.
W $40\%$ testów zachodzą dodatkowe ograniczenia: $n \le 1\,000$, $m \le 2\,000$.

\heading{Output}

Dla każdej drogi z wejścia wypisz w oddzielnej linii jedną liczbę
	-- liczbę dróg, które staną się bezpieczne po dodaniu danej drogi (włączając tę drogę).

\heading{Sample}


\sampleIN
5 8
1 2
3 3
4 5
2 3
4 5
3 4
4 1
5 2
\sampleOUT
0
1
0
0
2
0
4
1
\sampleEND


\end{document}
