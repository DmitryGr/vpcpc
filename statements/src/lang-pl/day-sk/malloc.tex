\input sys/inputs.tex

\usepackage[utf8]{inputenc}
\usepackage[T1]{fontenc}
\usepackage[polish]{babel}
\usepackage{polski}

\begin{document}

\bigheading{Dynamic memory allocation}

% \info{task_name}{infile}{outfile}{points}{timelimit}{memlimit}
% leave this values, if you are not interested
\info{malloc}{stdin}{stdout}{100}{1200 ms}{1 GB}

Kamila wzięła się za stworzenie nowego niesamowitego języka programistycznego,
	który będzie miał przewagę nad wszystkimi innymi językami pod każdym względem.
Pisanie kodu stałoby się nawet prostsze niż nauczenie się alfabetu,
	więc małe dzieci mogłyby zacząć programować zanim nauczyłyby się pisać komentarze do kodu.
To nie jest bug\footnote{Swoją drogą, robienie bugów byłoby i tak niemożliwe,
	bo genialny kompilator Kamili by naprawiał każdy błąd.},
	bo pisanie komentarzy byłoby kompletnie niepotrzebne
	(dzięki bardzo intuicyjnej składni języka, przeznaczenie każdego bloku kodu byłoby od razu jasne).
Kompilator byłby również w stanie upiec Ci ciasto na Twoje urodziny.

Jednak ciągle jest dużo pracy do zrobienia.
Na przykład, składnia języka odpowiedzialna za zarządzanie pamięcią jeszcze nie działa.
Kamila przygotowała dokładną specyfikację alokacji i dealokacji pamięci,
	ale nie miała wystarczająco czasu, aby je zaimplementować.
Pomóż jej i zdobądź wieczną chwałę za pomoc w tym wspaniałym projekcie!

\heading{Task}

Dostępna pamięć jest tablicą $n$ bajtów, ponumerowanych od $0$ do $n-1$.
Na początku wszystkie bajty są wolne (czyli niezaalokowane).
Potem następują zapytania, polegające na alokacji i zwalnianiu pamięci.

Alokacja jest opisana przez liczbę $l$.
System szuka wtedy bloku $l$ kolejnych wolnych bajtów, alokuje je i~zwraca pozycję pierwszego bajtu w tym bloku.
Jeśli jest wiele możliwych bloków do wybrania, wybierany jest blok zaczynający się na najmniejszej pozycji.
Jeśli nie istnieje taki blok, to zapytanie jest odrzucane i system zwraca $-1$.

Zwalnianie jest opisane przez dwie liczby $x$ i $l$.
System zwalnia (oznacza jako wolne) blok $l$ kolejnych bajtów zaczynających się od pozycji $x$
	i zwraca liczbę bajtów, które naprawdę zostały zwolnione
	(czyli liczbę bajtów w~tym bloku, które były zaalokowane przed tym zapytaniem).

\heading{Input}

W pierwszej linii wejścia znajdują się dwie liczby całkowite $n$ i $q$ ($1 \le n, q \le 3 \cdot 10^5$),
	oznaczające liczbę bajtów, z których składa się pamięć oraz liczbę zapytań.
Każda z kolejnych $q$ linii opisuje jedno zapytanie.
Alokacja ma postać ,,\texttt{1 $l$}'' ($1 \le l \le n$), a dealokacja postać ,,\texttt{2 $x$ $l$}''
	($0 \le x \le n -1$, $1 \le l \le n - x$).

\heading{Output}

Dla każdego zapytania wypisz jedną linię z wartością zwracaną przez system.

\heading{Sample}


\sampleIN
5 4
1 3
1 3
2 1 3
1 4
\sampleOUT
0
-1
2
1
\sampleEND


\end{document}
