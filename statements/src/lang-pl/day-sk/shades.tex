\input sys/inputs.tex

\usepackage[utf8]{inputenc}
\usepackage[T1]{fontenc}
\usepackage[polish]{babel}
\usepackage{polski}

\begin{document}

\bigheading{Shades of the town}

% \info{task_name}{infile}{outfile}{points}{timelimit}{memlimit}
% leave this values, if you are not interested
\info{shades}{stdin}{stdout}{100}{1000 ms}{1 GB}

Dawno dawno temu, w odległej galaktyce było sobie miasto.
W sumie, to tylko wydaje nam się, że to było miasto,
	ponieważ już dawno zniknęło i praktycznie nic po nim nie zostało -- tylko cień.
Chcielibyśmy znać wysokości budynków w mieście, aby policzyć mieszkańców.
Jednak znamy jedynie wysokości cieni\footnote{bo po budynkach został tylko cień -- przyp. red.},
	które są przeskalowanymi wysokościami budynków.

\heading{Task}

Wszystkie budynki w mieście mają jednakową szerokość i stoją w jednej linii, w równych odległościach od siebie.
Budynki mogą się różnić tylko wysokościami.
Znamy pewne architektoniczne schematy, używane przez mieszkańców tej planety.

Schemat to ciąg dodatnich liczb całkowitych, oznaczających wysokości budynków.
Mówimy, że schemat występuje w ciągu cieni budynków, jeśli istnieje spójny podciąg cieni budynków,
	który po przeskalowaniu byłby taki sam jak schemat.

Dla każdego schematu, znajdź liczbę jego wystąpień.

\heading{Input}

W pierwszej linii wejścia znajduje się jedna liczba całkowita $n$ -- liczba schematów do rozważenia.
Każda z następnych $n$ linii opisuje jeden schemat.
Opis schematu zaczyna się liczbą $l_i$ -- długością ciągu wysokości budynków.
Po niej następuje $l_i$ dodatnich liczb całkowitych, oznaczających wysokości poszczególnych budynków w ciągu.

Ostatnia linia wejścia opisuje ciąg cieni budynków.
Zaczyna się jedną liczbą $m$ -- liczbą cieni.
Po niej następuje $m$ dodatnich liczb całkowitych, oznaczających wysokości poszczególnych cieni.

Ograniczenia:
\begin{itemize}
	\item $1 \le m \le 3 \cdot 10^5$
	\item $1 \le l_i$
	\item $\sum_{i = 1}^n l_i \le m$
	\item wszystkie wysokości są nie większe niż $10\,000$
\end{itemize}
W $40\%$ testów zachodzi $m \le 1\,000$.

\heading{Output}

Na wyjście wypisz jedną liczbę -- liczbę wystąpień wszystkich schematów w ciągu cieni.

\heading{Sample}


\sampleIN
4
1 47
2 21 42
2 34 17
3 1 2 1
7 3 6 3 6 12 6 3
\sampleOUT
15
\sampleCOMMENT
Pierwszy schemat może zostać przeskalowany do dowolnej wysokości, więc występuje $7$ razy.
Drugi schemat występuje $3$ razy.
Trzeci schemat występuje $3$ razy.
Czwarty schemat występuje $2$ razy.
\sampleEND


\end{document}
