\input sys/inputs.tex

\usepackage[utf8]{inputenc}
\usepackage[T1]{fontenc}
\usepackage[polish]{babel}
\usepackage{polski}

\begin{document}

\bigheading{Sorting}

% \info{task_name}{infile}{outfile}{points}{timelimit}{memlimit}
% leave this values, if you are not interested
\info{sorting}{stdin}{stdout}{100}{100 ms}{32 MiB}

Mirek miał trochę plików zawierających liczby całkowite.
Jego zadaniem było posortować liczby w każdym z~plików, w kolejności rosnącej.
Mirek jest informatykiem, więc oczywiście próbował zrobić to jakimś narzędziem z wiersza poleceń.
Nazwa programu, który sortuje nie była trudna do zgadnięcia, ale program nie działał do końca tak, jak Mirek by chciał
	-- po posortowaniu plików Mirek zorientował się, że to narzędzie traktowało każdą liczbę jako napis
	i posortowała je w porządku leksykograficznym.
To było do przewidzenia, ale i tak Mirek był zaskoczony -- pomimo sortowania leksykograficznego,
	liczby we wszystkich plikach były posortowane rosnąco.

Teraz Mirek zastanawia się jak duże szczęście go spotkało i jak to w ogóle jest możliwe,
	że liczby z tych plików mają ten sam porządek leksykograficzny, co numeryczny.
Pomóż mu zaspokoić jego ciekawość.

\heading{Task}

Mając dany przedział liczb całkowitych $[A, B]$, znajdź liczbę podzbiorów tych liczb,
	takich, że ich porządek leksykograficzny jest taki sam jak porządek numeryczny.

\heading{Input}

W pierwszej linii wejścia znajdują się dwie liczby całkowite $A$ i $B$
	($1 \le A \le B \le 10^{18}$, $B - A \le 10^5$).

\heading{Output}

Na wyjście należy wypisać jedną liczbę całkowitą -- liczbę podzbiorów zbioru $\{A, A+1, \ldots, B\}$,
	które mają własność opisaną w treści zadania.
Odpowiedź może być bardzo duża, więc wypisz ją modulo $10^9+7$.

\heading{Sample}

\sampleIN
98 101
\sampleOUT
7
\sampleCOMMENT
Tymi podzbiorami są: $\emptyset$, $\{98\}$, $\{99\}$, $\{100\}$, $\{101\}$,	$\{98, 99\}$, $\{100, 101\}$.
\sampleEND

\end{document}
