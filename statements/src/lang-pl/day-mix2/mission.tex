\input sys/inputs.tex

\usepackage[utf8]{inputenc}
\usepackage[T1]{fontenc}
\usepackage[polish]{babel}
\usepackage{polski}

\begin{document}

\bigheading{Mission}

% \info{task_name}{infile}{outfile}{points}{timelimit}{memlimit}
% leave this values, if you are not interested
\info{mission}{stdin}{stdout}{100}{1000 ms}{1 GB}

\noindent
Uwaga żołnierzu!

\bigskip \noindent
Mam dla Was specjalne zadanie.
Wykryliśmy bazę wroga i musi ona zostać zniszczona.
Dostaniecie mapę i~wystarczająco dużo bomb by zniszczyć wszystko.
Po udanej akcji będzie czekał na Was helikopter w~pobliskim lesie.

\bigskip \noindent
Brzmi łatwo, co nie?
Znajdźcie najszybszą drogę by osiągnąć cel i upewnijcie się, że nie odwiedzicie tego samego miejsca dwa razy.
W~przeciwnym wypadku zostaniecie wykryci.

\bigskip \noindent
Czy wszystko jasne?
Barzo dobrze... przygotujcie się, ponieważ wyruszacie za 10 minut!

\bigskip \noindent
Życzę Wam powodzenia, nie dajcie się zabić i do zobaczenia na obiedzie.

\heading{Task}

Otrzymujecie nieskierowany graf ważony oraz trzy różne wierzchołki:
	Waszą bazę, bazę wroga i miejsce, w~którym czeka helikopter.
Znajdźcie najkrótszą ścieżkę w~grafie między Waszą bazą i miejscem, w~którym czeka helikopter.
Ścieżka musi przechodzić przez bazę wroga i nie może odwiedzić żadnego wierzchołka więcej niż raz.

\heading{Input}

W~pierwszej linii wejścia znajduje się pięć liczb całkowitych $N$, $M$, $B$, $W$, $H$
	($3 \le N \le 1000, 0 \le M \le 1000, 1 \le B,W,H \le N, B \neq W, W \neq H, B \neq H$),
	oznaczające odpowiednio: liczbę wierzchołków grafu,
	liczbę krawędzi grafu,
	numer wierzchołka, w~którym znajduje się Wasza baza,
	numer wierzchołka, w~którym znajduje się baza wroga
	i numer wierzchołka, w~którym znajduje się helikopter.
Wierzchołki grafu są ponumerowane od $1$ do $N$.
Kolejne $M$ linii opisuje krawędzie grafu.
Każda linia zawiera trzy liczby całkowite $v$, $w$ i $t$
	($1 \le v,w \le N, v \neq w, 1 \le t \le 1\,000\,000$).
Oznaczają one, że istnieje nieskierowana krawędź między wierzchołkami
	o~numerach $v$ oraz $w$ i przejście nią zajmuje $t$ jednostek czasu.
Żadne dwa wierzchołki nie są połączone więcej niż jedną krawędzią.

\smallskip
W $40\%$ testów zachodzi $N \le 20$.

\heading{Output}
Wypiszcie jedną linię z~jedną liczbą całkowitą, oznaczającą minimalny czas potrzebny do wykonania misji.
Jeżeli wykonanie misji jest niemożliwe to poddajcie się i nie wracajcie na obiad, lub wypiszcie $-1$.

\heading{Samples}


\sampleIN
3 2 1 2 3
1 2 10
2 3 20
\sampleOUT
30
\sampleEND


\bigskip


\sampleIN
3 0 2 1 3
\sampleOUT
-1
\sampleEND


\bigskip


\sampleIN
4 4 3 2 4
2 3 5
3 1 1
1 4 1
2 4 100
\sampleOUT
105
\sampleEND

\end{document}
