\input sys/inputs.tex

\usepackage[utf8]{inputenc}
\usepackage[T1]{fontenc}
\usepackage[polish]{babel}
\usepackage{polski}

\begin{document}

\bigheading{An inexperienced slalomer}

% \info{task_name}{infile}{outfile}{points}{timelimit}{memlimit}
% leave this values, if you are not interested
\info{slalom}{stdin}{stdout}{100}{300 ms}{1 GB}

To jest pierwszy slalom Mirka, więc nic dziwnego, że jest barzo zdenerwowany.
Dodatkowo, jest to jego pierwszy raz na nartach
	i jeszcze się nie nauczył robić skrętów -- potrafi tylko ślizgać się w~linii prostej.
Ale nic straconego.
Być może twórca trasy nie był zbyt uważny i da się przejechać przez każdą bramkę bez robienia pojedynczego skrętu.

\heading{Task}

Masz dany opis slalomu z $n$ bramkami.
Trasa biegnie od lewej do prawej.
Każda bramka jest reprezentowana przez pionowy odcinek między dwoma słupkami.
Z~lotu ptaka Mirek wygląda jak dysk o średnicy $d$ ($d \geq 0$), którego środek porusza się po linii prostej.
Mirek może wybrać jako punkt startowy dowolny punkt położony na lewo od wszystkich bramek,
	a jako punkt końcowy dowolny punkt położony na prawo od wszystkich bramek.
Żeby ukończyć trasę, Mirek musi przejechać całym swoim ciałem pomiędzy wszystkimi bramkami.
Dotykanie słupków jest dozwolone. 

Znajdź największą średnicę $d$, dla której Mirek jest w~stanie przejechać slalom.

\heading{Input}

W~pierwszej linii wejścia znajduje się jedna liczba całkowita $n$ -- liczba bramek
	($1 \leq n \leq 10^5$).
Każda z~kolejnych linii opisuje bramkę i zawiera trzy liczby całkowite $x, y_1, y_2$
	($0 \leq x \leq 10^9$, $0 \leq y_1 \leq y_2 \leq 10^9$).
Opisywana bramką jest pionowym odcinkiem o końcach w~punktach $[x, y_1]$ i $[x, y_2]$.
Żadne dwie bramki nie mają takiej samej współrzędnej $x$.

\heading{Output}

Twój program powinien wypisać na wyjście największe takie $d$, że Mirek może przejechać slalom.
Odpowiedź zostanie uznana za poprawną jeśli jej bezwzględny bądź względny błąd będzie mniejszy niż $10^{-9}$.
W C++, możesz wypisać odpowiedź używając \verb!printf("%.10lf\n", d);!

Jeśli nie istnieje żadne nieujemne $d$ spełniające warunki zadania, wypisz \texttt{Impossible}.

\heading{Samples}

\sampleIN
3
4 3 7
6 6 9
1 5 10
\sampleOUT
1.3728129460
\sampleEND

\sampleIN
2
3 7 9
10 4 4
\sampleOUT
0.0000000000
\sampleEND

\sampleIN
3
0 4 7
2 0 3
4 4 7
\sampleOUT
Impossible
\sampleEND


\end{document}
