\input sys/inputs.tex

\usepackage[utf8]{inputenc}
\usepackage[T1]{fontenc}
\usepackage[polish]{babel}
\usepackage{polski}

\begin{document}

\bigheading{Tickets}

% \info{task_name}{infile}{outfile}{points}{timelimit}{memlimit}
% leave this values, if you are not interested
\info{tickets}{stdin}{stdout}{100}{800 ms}{32 MiB}

Zbliża się ważny mecz, na który przyjdzie wielu kibiców.
Na stadionie jest $N$ miejsc siedzących ponumerowanych liczbami całkowitymi od $1$ do $N$.
Każdy kibic, kupując bilet, może wybrać przedział siedzeń, na których mógłby się zgodzić siedzieć.
Ten przedział jest opisany przez dwie liczby całkowite $L$ i $P$, oznaczające pierwsze i~ostanie miejsce w przedziale.
To oznacza, że kibic zgodzi się jedynie na miejsca $S$ takie, że zachodzi $L \le S \le P$.
Dostano od kibiców $M$ pytań o kupno biletów.
Jako, że umiesz co nieco programować, zlecono Ci napisanie programu, który przydzieli kibicom miejsca,
	tak, aby jak najwięcej kibiców usiadło na właściwych miejscach.

\heading{Task}

Napisz program, który obliczy maksymalną liczbę kibiców, którzy mogą otrzymać bilety z ich upragnionym miejscem,
	a następnie wypisze odpowiedni przydział miejsc.
Żadnych dwóch kibiców nie może dostać tego samego miejsca.

\heading{Input}

W pierwszym wierszu wejścia znajdują się dwie liczby całkowite $N$ i $M$ ($1 \le N \le 100\,000$, $1 \le M \le 1\,000\,000$),
	oznaczające odpowiednio liczbę miejsc siedzących i liczbę kibiców, którzy chcą kupić bilet.

Każdy z następnych $M$ wierszy zawiera dwie liczby całkowite $L$ i $P$ ($1 \le L \le P \le N$),
	oznaczające przedział siedzeń, wybranych przez kibica.
Kibice są ponumerowani kolejnymi liczbami od $1$ do $M$ w tej samej kolejności, w jakiej pojawiają się na wejściu.

\heading{Output}

W pierwszym wierszu wyjścia wypisz jedną liczbę całkowitą $K$, oznaczającą największą możliwą liczbę kibiców,
	których można usatysfakcjonować.
W każdym z następnych $K$ wierszy wypisz dwie liczby całkowite $S$ i~$R$, oznaczające, że na siedzeniu
	nr $S$ powinien usiąść kibic $R$.
Przypisania kibiców do siedzeń mogą być wypisane w dowolnej kolejności.
Jeśli jest wiele poprawnych odpowiedzi, Twój program może wypisać dowolną z nich.

\heading{Sample}

\sampleIN
10 9
1 3
2 4
5 7
2 6
1 5
3 7
4 8
7 9
3 8
\sampleOUT
9
1 1
2 5
3 2
4 4
5 6
6 9
7 3
8 7
9 8
\sampleEND

\end{document}
