\input sys/inputs.tex

\usepackage[utf8]{inputenc}
\usepackage[T1]{fontenc}
\usepackage[polish]{babel}
\usepackage{polski}

\begin{document}

\bigheading{Newspapers}

% \info{task_name}{infile}{outfile}{points}{timelimit}{memlimit}
% leave this values, if you are not interested
\info{newspapers}{stdin}{stdout}{100}{4000 ms}{1 GB}

W pięknej polskiej wsi -- w Mirkowie -- znajduje się $N$ skrzyżowań
	połączonych $N-1$ dwukierunkowymi drogami.
Można się dostać z każdego skrzyżowania do każdego innego, korzystając jedynie z tych dróg.

Mirek pracuje w Mirkowie jako doręczyciel gazet.
Dla każdej drogi wiadomo, ilu mieszkańców przy niej mieszka
	(i jest to również liczba gazet, które Mirek musi tam dostarczyć).
Codziennie Mirek wybiera dwa skrzyżowania i odwiedza każdy dom na najkrótszej ścieżce między nimi.
Jego dzienny zarobek jest proporcjonalny do średniej liczby gazet, dostarczonej na pojedynczej drodze
	(czyli jest proporcjonalny do liczby dostarczonych gazet, podzielonej przez liczbę przebytych dróg).

Na początku, Mirek chciał wykorzystać ten wspaniały system wynagrodzeń,
	za każdym razem dostarczając gazety tylko ludziom, którzy mieszkają przy drodze
	z największą liczbą mieszkańców.
Jednak jego szef szybko wykrył jego sposób i był zmuszony zakończyć ten proceder,
	ponieważ zbyt dużo osób nie dostawało gazety.
Aby zmusić Mirka do dostarczania większej liczby gazet,
	dał mu dodatkowe ograniczenie: Mirek musi za każdym razem wybrać ścieżkę,
	przechodzącą przez co najmniej $K$ dróg.

Pomóż Mirkowi i znajdź dla niego optymalną trasę.

\heading{Task}

Mając dany opis Mirkowa, znajdź ścieżkę, przechodzącą przez co najmniej $K$ dróg,
	z największą średnią dostarczonych gazet na jednej drodze.

\heading{Input}

W pierwszym wierszu wejścia znajdują się dwie liczby całkowite $N$ i $K$ ($1 \le N \le 50\,000$, $1 \le K \le N-1$),
	oznaczające liczbę skrzyżowań w Mirkowie i minimalną długość trasy Mirka.
W następnych $N-1$ wierszach znajdują się opisy poszczególnych dróg.
Opis jednej drogi składa się z trzech liczb całkowitych $A$, $B$ i $C$ ($1 \le A < B \le N$, $0 \le C \le 10^6$),
	oznaczających drogę łączącą skrzyżowania $A$ i $B$, przy której mieszka $C$ osób.

Możesz założyć, że istnieje co najmniej jedna trasa o długości $K$.

W testach wartych $30$ punktów: $N \le 1000$.

\heading{Output}

Wypisz jedną liczbę -- średnią liczbę dostarczonych gazet na jednej drodze, w optymalnej trasie Mirka.
Odpowiedź będzie uznana za poprawną, jeśli różnica między nią, a prawidłową odpowiedzią będzie mniejsza niż $10^{-6}$.

\heading{Sample}

\sampleIN
5 2
1 2 4
2 3 1
3 4 3
3 5 3
\sampleOUT
3.00000000
\sampleEND

\end{document}
