\input sys/inputs.tex

\usepackage[utf8]{inputenc}
\usepackage[T1]{fontenc}
\usepackage[polish]{babel}
\usepackage{polski}

\begin{document}

\bigheading{Bus lines}

% \info{task_name}{infile}{outfile}{points}{timelimit}{memlimit}
% leave this values, if you are not interested
\info{buslines}{stdin}{stdout}{100}{3000 ms}{1 GB}

W czeskim mieście Kocourkov organizacja komunikacji miejskiej jest dosyć dziwna.
W mieście znajduje się $N$ przystanków i $N-1$ dróg.
Każda droga łączy dwa przystanki.
Korzystając z tych dróg, możliwa jest podróż między dowolnymi dwoma przystankami.

Każdego ranka, z każdego przystanku wyrusza $N-1$ autobusów -- każdy z nich kieruje się do innego przystanku.
Sumarycznie busów jest $N \cdot (N - 1)$.
Podczas trasy, autobus zatrzymuje się na wszystkich przystankach, które napotka na drodze.
Na przystankach należy wywiesić rozkłady jazdy autobusów.
Zastanawiasz się jak długi jest rozkład jazdy na każdym przystanku.

\heading{Task}

Mając dany opis przystanków i dróg, które je łączą,
	oblicz dla każdego przystanku, ile autobusów się na nim zatrzymuje.

\heading{Input}

W pierwszej linii wejścia znajduje się jedna liczba $N$ ($1 \le N \le 10^6$),
	oznaczająca liczbę przystanków autobusowych w Kocourkovie
	(przystanki są ponumerowane od $1$ do $N$).
Następne $N-1$ linii zawiera opisy dróg.
Każda linia składa się z dwóch różnych liczb całkowitych $x$ i $y$ ($1 \le x, y \le N$),
	które oznaczają, że istnieje droga łącząca przystanki nr $x$ i nr $y$.

W $20\%$ testów zachodzi dodatkowy warunek $N \le 100$.

W $40\%$ testów zachodzi dodatkowy warunek $N \le 1\,000$.

\heading{Output}

Wyjście powinno składać się z $N$ wierszy, a $i$-ty z nich powinien zawierać jedną liczbę całkowitą
	-- liczbę autobusów zatrzymujących się na $i$-tym przystanku.

\heading{Samples}


\sampleIN
6
1 2
2 3
3 4
4 5
5 6
\sampleOUT
10
18
22
22
18
10
\sampleEND

\sampleIN
5
4 5
2 1
3 2
2 5
\sampleOUT
8
18
8
8
14
\sampleEND


\end{document}
