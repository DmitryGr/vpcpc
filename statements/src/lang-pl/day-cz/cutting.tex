\input sys/inputs.tex

\usepackage[utf8]{inputenc}
\usepackage[T1]{fontenc}
\usepackage[polish]{babel}
\usepackage{polski}

\begin{document}

\bigheading{Cutting of a birthday cake}

% \info{task_name}{infile}{outfile}{points}{timelimit}{memlimit}
% leave this values, if you are not interested
\info{cutting}{plik tekstowy}{plik tekstowy}{100}{nie dotyczy}{nie dotyczy}

Mimino obchodzi swoje urodziny.
Jest świetnym programistą, chociaż ma tylko $N$ latek.
Kompilator Kamili (tej z wczoraj \texttt{:D}) upiekł ogromne
	ciasto czekoladowe\footnote{Tak duże, że można je utożsamiać z nieskończoną płaszczyzną}
	na urodziny Mimina i umieścił na nim $N$ świeczek.
Mimino i jego przyjaciele chcieliby zjeść to ciasto, ale przedtem muszą je odpowiednio podzielić.
Ciasto można kroić wzdłuż wybranej prostej, która jest pionowa, pozioma, albo skośna.
Po pokrojeniu ciasta, każdy kawałek może zawierać co najwyżej jedną świeczkę.

Mimino zastanawia się ile minimalnie cięć musi wykonać, aby spełnione zostały wszystkie te wymagania.

\heading{Task}

Mając daną płaszczyznę z wyróżnionymi $N$ punktami, znajdź zbiór prostych,
	które dzielą tę płaszczyznę tak, że każda część zawiera co najwyżej jeden wyróżniony punkt.
Ten zbiór powinien być jak najmniejszy.

Współrzędne wszystkich punktów są liczbami całkowitymi.
Proste są zadane poprzez dwa różne punkty (o~całkowitych współrzędnych)
	i są skierowane od pierwszego punktu do drugiego.
To skierowanie jest istotne przy rozstrzyganiu remisów, gdy jakiś punkt leży na linii cięcia.
	-- wówczas uznaje się, że taki punkt leży po prawej stronie prostej.
Linie mogą być tylko pionowe, poziome, albo nachylone pod kątem $45^\circ$ do osi układu współrzędnych.

\heading{Input}

W pierwszej linii wejścia znajduje się jedna liczba całkowita $N$, oznaczająca liczbę punktów.
Następne $N$ linii opisuje te punkty.
W $(i+1)$-wszej linii znajdują się dwie liczby całkowite -- współrzędne $i$-tego punktu.

\heading{Output}

W pierwszym wierszu wyjścia należy wypisać jedną liczbę całkowitą $L$ -- liczbę cięć, które chcesz wykonać.
Następne $L$ wierszy powinno zawierać opisy poszczególnych cięć.
Opis jednego cięcia składa się z czterech liczb całkowitych oddzielonych spacją: $X_1$, $Y_1$, $X_2$, $Y_2$.
Punkty $(X_1, Y_1)$ i $(X_2, Y_2)$ muszą być różne -- są to punkty, przez które ma przechodzić dane cięcie.
Co najmniej jeden z poniższych warunków musi zachodzić:
\begin{itemize}
	\item $X_1 = X_2$ (pionowa linia)
	\item $Y_1 = Y_2$ (pozioma linia)
	\item $X_1 - X_2 = Y_1 - Y_2$ (skośna linia, typ I)
	\item $X_1 - X_2 = Y_2 - Y_1$ (skośna linia, typ II)
\end{itemize}
Twoje rozwiązanie nie może wypisać więcej niż $10\,000$ cięć,
	a wartość bezwzględna współrzędnych nie może być większa niż $1\,000\,000$.

\heading{Scoring}

To jest zadanie z jawnymi testami.
Możesz pobrać wszystkie $10$ testów z systemu.
Rozwiązaniem zadania są pliki wyjściowe (nie należy dołączać kodu).

Jeśli Twoja odpowiedź nie będzie dokładnie przestrzegała specyfikacji wyjścia,
	Twój wynik za dany test będzie równy $0$.

W przeciwnym przypadku, Twoim wynikiem będzie $10 \cdot \left(1 - \sqrt{1 - L_{min} / L} \right)$,
	gdzie $L$ jest liczbą cięć w Twoim wyjściu, a $L_{min}$ jest najlepszym wynikiem spośród wszystkich zawodników.
Wyniki zostaną obliczone dopiero po zakończeniu zawodów.

\newpage

\heading{Sample}

\sampleIN
4
3 1
4 5
6 6
8 4
\sampleOUT
2
3 2 8 7
2 8 8 2
\sampleEND

\end{document}
