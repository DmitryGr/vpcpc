\input sys/inputs.tex

\usepackage[utf8]{inputenc}
\usepackage[T1]{fontenc}
\usepackage[polish]{babel}
\usepackage{polski}

\begin{document}

\bigheading{Wall}

% \info{task_name}{infile}{outfile}{points}{timelimit}{memlimit}
% leave this values, if you are not interested
\info{wall}{stdin}{stdout}{100}{1000 ms}{128 MB}

Mirek jest konserwatorem zabytków.
Jego dzisiejszym zadaniem jest naprawa muru obronnego pewnej starej twierdzy.
Mirek musi się spieszyć, bo mur jest w bardzo złym stanie i mało brakuje, żeby runął.
Przeszukał Internet i znalazł robota, który został specjalnie zaprojektowany do naprawiania takich murów.
Po zakupie robota, postanowił zaplanować optymalny sposób naprawy muru, jednak potrzebuje do tego Twojej pomocy.

Można uznać, że mur jest nieskończoną linią prostą.
Mirek zapisał sobie wszystkie punkty na tym murze, które wymagają naprawy.
Dla każdego takiego punktu Mirek zna kosz naprawy $C_i$ i współczynnik $D_i$,
	mówiący jak koszt naprawy wzrasta, jeśli punkt nie zostałby natychmiast naprawiony.
Jeśli $i$-ty punkt zostałby naprawiony po czasie $t$, to koszt naprawy tego punktu byłby równy:
\[ C_i + t \cdot D_i \]

\heading{Task}

Mając dane współrzędne punktów na murze i początkową pozycję robota,
	i wiedząc, że transport robota z~punktu $x_1$ do punktu $x_2$ zajmuje $|x_1 - x_2|$
	czasu\footnote{Tutaj $x_1$ i $x_2$ oznaczają współrzędne tych dwóch punktów (nie ich indeksy).},
	oblicz minimalny koszt naprawy wszystkich zepsutych punktów.
Możesz założyć, że czas naprawy jednego punktu jest pomijalnie mały.

\heading{Input}

W pierwszym wierszu wejścia znajdują się dwie liczby $N$ i $P$ ($1 \le N \le 2000$, $0 \le P \le 10^9$)
	-- liczba punktów na murze, które wymagają naprawy i początkowa pozycja robota.
W kolejnych $N$ liniach znajdują się opisy poszczególnych punktów.
Opis jednego punktu składa się z trzech liczb $X_i$, $C_i$ i $D_i$
	($0 \le X_i \le 10^9$, $0 \le C_i, D_i \le 10^6$, $X_i \neq P$)
	-- są to: pozycja punktu i współczynniki kosztu naprawy.
Żadne dwa punkty nie leżą na tych samych pozycjach.

\heading{Output}

Na wyjście wypisz jedną liczbę całkowitą -- minimalny koszt naprawy wszystkich punktów na murze.

\heading{Samples}

\sampleIN
3 7
10 32 1
3 5 1
14 0 2
\sampleOUT
72
\sampleCOMMENT
Optymalny plan jest następujący:
\begin{itemize}
	\item Przetransportować robota z punktu $7$ do punktu $10$
		i naprawić pierwszy punkt po czasie $3$.
	\item Przetransportować robota z punktu $10$ do punktu $14$
		i naprawić trzeci punkt po czasie $3 + 4 = 7$.
	\item Przetransportować robota z punktu $14$ do punktu $3$
		i naprawić drugi punkt po czasie $3 + 4 + 11 = 18$.
\end{itemize}
Wówczas, sumaryczny koszt naprawienia ściany jest równy $5 + 18 \cdot 1 + 32 + 3 \cdot 1 + 0 + 7 \cdot 2 = 72$.
\sampleEND

\end{document}
