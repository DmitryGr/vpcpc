\input sys/inputs.tex

\usepackage[utf8]{inputenc}
\usepackage[T1]{fontenc}
\usepackage[polish]{babel}
\usepackage{polski}

\begin{document}

\bigheading{Universities}

% \info{task_name}{infile}{outfile}{points}{timelimit}{memlimit}
% leave this values, if you are not interested
\info{universities}{stdin}{stdout}{100}{1000 ms}{1 GB}

W Bajtocji znajduje się $N$ uniwersytetów.
Każdy z nich uczy albo czarnej, albo białej magii.
Uniwersytety są połączone przy pomocy $N-1$ dróg,
	tak, że między każdą parą z nich można przejść na dokładnie jeden sposób.

Planujesz pozwiedzać niektóre uczelnie.
Każdej z nich przypisałeś współczynnik fajności
	(jeśli uniwersytet jest wyjątkowo słaby, to współczynnik fajności może być ujemny).
Fajność wycieczki jest równa sumie fajności wszystkich odwiedzonych uniwersytetów.

Aby zaplanować wycieczkę, musisz wybrać dwa różne uniwersytety i odwiedzić wszystkie uniwersytety
	na ścieżce pomiędzy nimi (włącznie z tymi wybranymi).
Aby zachować równowagę magii, musisz odwiedzić taką samą liczbę ,,białych'' uniwersytetów,
	co ,,czarnych'' uniwersytetów.

Zastanawiasz się teraz, jaka jest najfajniejsza wycieczka, którą możesz odbyć.

\heading{Task}

Masz dane opisy uniwersytetów i dróg między nimi.
Znajdź najfajniejszą\footnote{taką, której współczynnik fajności jest największy}
	wycieczkę, w której odwiedzisz taką samą liczbę ,,białych'', co ,,czarnych'' uniwersytetów.

\heading{Input}

Wejście składa się z czterech wierszy.

W pierwszym wierszu znajduje się jedna liczba całkowita $N$ ($2 \le N \le 10^5$),
	oznaczająca liczbę uniwersytetów w Bajtocji (ponumerowanych od $1$ do $N$).

W drugim wierszu znajduje się napis złożony z liter \texttt{B} i \texttt{W}.
Jeśli $i$-tą literą jest \texttt{B}, to $i$-ty uniwersytet uczy czarnej magii,
	a jeśli $i$-tą literą jest \texttt{W}, to $i$-ty uniwersytet uczy białej magii.
Możesz założyć, że wystąpi co najmniej jedna literka \texttt{B} i co najmniej jedna literka \texttt{W}.

W trzecim wierszu znajduje się $N$ liczb całkowitych oddzielonych spacjami $h_1$, $h_2$, $\ldots$, $h_N$
	($-10^5 \le h_i \le 10^5$).
Liczba $h_i$ oznacza fajność $i$-tego uniwersytetu.

W czwartym wierszu znajduje się $N - 1$ liczb całkowitych oddzielonych spacjami $v_1$, $v_2$, $\ldots$, $v_{N-1}$.
Liczba $v_i$ oznacza, że istnieje droga łącząca uniwersytety nr $v_i$ i $(i + 1)$ ($1 \le v_i \le i$).

\heading{Output}

Wypisz dokładnie jedną liczbę -- fajność najfajniejszej wycieczki, spełniającej warunki zadania.

\heading{Samples}

\sampleIN
6
BWBBBW
6 0 3 -2 100 5
1 2 2 4 4
\sampleOUT
9
\sampleCOMMENT
W optymalnej wycieczce odwiedzasz uniwersytety $1$, $2$, $4$, $6$.
\sampleEND

\sampleIN
3
WBW
1 -10 5
1 2
\sampleOUT
-5
\sampleCOMMENT
Zawsze musisz odwiedzić jakieś dwa uniwersytety,
	więc czasem wynik jest ujemny.
\sampleEND

\end{document}
