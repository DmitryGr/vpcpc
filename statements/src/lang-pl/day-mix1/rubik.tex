\input sys/inputs.tex

\usepackage[utf8]{inputenc}
\usepackage[T1]{fontenc}
\usepackage[polish]{babel}
\usepackage{polski}

\begin{document}

\bigheading{Cubic Art}

% \info{task_name}{infile}{outfile}{points}{timelimit}{memlimit}
% leave this values, if you are not interested
\info{rubik}{stdin}{stdout}{100}{1000 ms}{1 GB}

Sztuka współczesna jest nieprzewidywalna.
Kiedy Bob sprzątał swój pokój, znalazł swoją starą kostkę Rubika.
Zaciekawiony znaleziskiem zaczął się nim bawić i wykonał kilka ruchów (nie więcej niż $65\,000$).
Chciał stworzyć prześliczne arcydzieło, ale coś mu nie wyszło i nie był zbyt zadowolony z układu, który otrzymał.
Dopiero później się zorientował, że kilka ruchów zrobił niepoprawnie.
Gdyby tylko mógł się cofnąć w czasie i~naprawić swoje błędy...

Bob pragnie dokonać kilku zmian w swoich ruchach (ale nie więcej niż $65\,000$).
Każda zmiana polega na zastąpieniu jednego ruchu jakimś innym ruchem.
Bob chciałby zobaczyć, jak będzie wyglądała kostka, po wprowadzeniu każdej zmiany.
Nie chciałby jednak wykonywać za każdym razem tych ruchów na kostce.
Pomóż mu i napisz program, który będzie w stanie zrobić to za niego.

\heading{Task}

Masz dany początkowy stan kostki Rubika (na początku kostka nie musi być ułożona).
Masz również dany początkowy układ ruchów, wykonywanych przez Boba.

Dane są również wszystkie zmiany, które Bob chciałby zastosować.
Każda zmiana ma postać: ,,zastąp $k$-ty ruch nowym, podanym ruchem''.
Po każdej zmianie, wypisz stan kostki po wykonaniu wszystkich ruchów.

Wszystkie zmiany zachowują się pomiędzy kolejnymi poprawkami
	-- czyli, na przykład, druga zmiana zostanie zastosowana do sekwencji ruchów
	z pierwszą zmianą, a nie do początkowej sekwencji ruchów.

\heading{Cube's I/O}

Nazwijmy kolory na kostce: \texttt{A}, \texttt{B}, \texttt{C}, \texttt{D}, \texttt{E}, \texttt{F}.
Kiedy obracasz ścianami kostki, środkowe kwadraciki się nie poruszają.
Zatem zawsze \texttt{A} będzie kolorem środka górnej ściany, \texttt{B}, \texttt{C}, \texttt{D}, \texttt{E}
	będą środkami bocznych ścian (w tej kolejności), a \texttt{F} będzie środkiem dolnej ściany.
Kostkę można by rozciąć do poniższej siatki:

\noindent
\texttt{???} \\
\texttt{?A?} \\
\texttt{???} \\
\texttt{????????????} \\
\texttt{?B??C??D??E?} \\
\texttt{????????????} \\
\texttt{???} \\
\texttt{?F?} \\
\texttt{???} \\

\heading{Input}

Pierwsze dziewięć linii wejścia zwiera opis początkowego stanu kostki.
Stan kostki jest zapisany w powyższej formie.
Możesz założyć, że środki sześciu ścian mają takie kolory jakie są podane
	powyżej.\footnote{Możesz również założyć, że początkowy stan jest poprawnym układem kostki Rubika,
		czyli może zostać uzyskany z ułożonego układu.
		To w ogóle nie ma znaczenia dla tego zadania, ale tak już jest i nie należy z tym dyskutować.}

Dalej jest podana linia z dwiema liczbami $n$ i $m$, oznaczającymi liczbę ruchów i liczbę zmian wykonywanych przez Boba.

W kolejnych $n$ liniach znajdują się opisy ruchów.
Mają formę ,,\texttt{$C_i$ $d_i$}'', gdzie $C_i$ jest kolorem środka obracanej ściany, a $d_i$ jest równe $-1$
	jeśli obraca się tę ścianę zgodnie ze wskazówkami zegara i $1$ jeśli przeciwnie do wskazówek zegara.

W ostatnich $m$ liniach opisane są poszczególne zmiany.
Każda linia jest postaci ,,\texttt{$a_j$ $C_j$ $d_j$}'', gdzie $a_j$ jest numerem\footnote{Ruchy są numerowane od $1$.}
	ruchu, który należy zastąpić, a $C_j$ i $d_j$ opisują nowy ruch.

We wszystkich testach zachodzi $n, m \le 65\,000$.
W $50\%$ testów zachodzi $n, m \le 1000$.

\newpage

\heading{Output}

Niech $S_i$ będzie sekwencją ruchów, uzyskaną z początkowej sekwencji, poprzez zastosowanie pierwszych $i$ zmian.
Dla każdego $i$ od $1$ do $m$ włącznie, wypisz dziewięć linii, opisujących stan kostki po zastosowaniu
	ruchów $S_i$ do początkowego układu kostki.
Kostkę należy wypisać w takim samym formacie, co był w przypadku wejścia.

\heading{Sample}

\sampleIN
AAA
AAA
AAA
BBBCCCDDDEEE
BBBCCCDDDEEE
BBBCCCDDDEEE
FFF
FFF
FFF
8 4
E 1
E -1
B 1
B -1
F 1
F -1
E 1
E -1
2 C -1
4 D -1
6 A -1
8 C -1
\sampleCOMMENT
Oryginalne ruchy się nawzajem redukują, więc na końcu kostka pozostaje w początkowym układzie.

\vspace{0.2cm}

Po zastosowaniu czterech zmian opisanych na wejściu,
	otrzymamy sekwencję ruchów, która odwraca wszystkie kolory, poza środkami ścian.
\sampleOUT
DAD
DAD
DAD
ABACCCFDFEEE
ABACCCFDFEEE
ABACCCFDFEEE
BFB
BFB
BFB
CCC
DAD
CCC
AAABCBFFFDED
BBBFCFDDDAEA
AAABCBFFFDED
EEE
BFB
EEE
CDC
CAC
CDC
BCBFFFDEDAAA
BBBFCFDDDAEA
BCBFFFDEDAAA
EBE
EFE
EBE
DDD
DAD
DDD
CCCFFFEEEAAA
CBCFCFEDEAEA
CCCFFFEEEAAA
BBB
BFB
BBB
\sampleEND

\end{document}
