\input sys/inputs.tex

\begin{document}

\bigheading{Oprava cesty}

% \info{task_name}{infile}{outfile}{points}{timelimit}{memlimit}
% leave this values, if you are not interested
\info{wall}{stdin}{stdout}{100}{1000 ms}{1 GB}

Slovenské cesty sú väčšinou v katastrofálnom stave. Obrovské jamy a výmole sú úplne bežné a človek
musí vedieť šoférovať naozaj obratne, aby sa im všetkým vyhol. Samozrejme, nikomu z kompetentných to
neprekáža až do okamihu, kým sa na návštevu neprihlási anglická kráľovná a nechce nás prísť
navštíviť na VPCPC.

V tom okamihu sa zalarumuje špeciálna vládna cestárska spoločnosť, ktorej jedinou úlohou je
zaplátať všetky výmole na ceste z bratislavského letiska po ubytovňu v Danišovciach. Na túto prácu
je vyhradený špeciálny stroj, ktorý sa teraz nachádza niekde medzi letiskom a Danišovcami. Ten
dokáže opraviť ľubovoľnú dieru, na ktorej stojí, za nulový čas, bez ohľadu na to, ako je diera veľká. Avšak veľkosť
diery ovplyvňuje množstvo asfaltu, ktorý musí využiť. A diery sa postupne zväčšujú, až kým nebudú
opravené.

Otázkou teda je, koľko najmenej asfaltu je potrebné na zaplátanie každého výmoľu na ceste, aby sa
podľa toho dala vystaviť faktúra pre štát\footnote{Samozrejme, táto hodnota nebude korelovať s
cenou, ktorá bude vo faktúre.}.

\heading{Úloha}

Cestu z letiska do Danišoviec si môžeme predstaviť ako rovnú čiaru, na ktorej sa nachádzajú body
predstavujúce výmole. Pre každý výmoľ poznáme začiatočné množstvo asfaltu $C_i$, ktoré je potrebné
na zaplátanie $i$-teho výmoľu a hodnotu $D_i$, ktorá vyjadruje, o koľko viac asfaltu bude treba na
zaplátanie $i$-tej diery za každú sekundu, čo diera nebude opravená. Ak teda začneme opravovať výmoľ
$i$ v sekunde $t$, oprava nás bude stáť $C_i + t \cdot D_i$.

Na začiatku poznáme pozície jednotlivých dier $x_i$ a počiatočnú pozíciu stroja $P$. Vieme, že
prejsť z bodu s pozíciou $x$ na pozíciu $y$ trvá stroju $|x-y|$ sekúnd. Zistite najmenšie množstvo
asfaltu, ktoré je potrebné na zaplátanie všetkých dier na ceste.

\heading{Vstup}

Na prvom riadku vstupu sú dve čísla $N$ a $P$ ($1 \le N \le 2\,000$, $0 \le P \le 10^9$) -- počet
výmoľov a počiatočnú pozíciu stroja.

Nasledujúcich $N$ riadkov popisuje výmole pomocou trojice čísiel $x_i$, $C_i$ a $D_i$ ($0 \leq x_i
\leq 10^9$, $0 \leq C_i,D_i \leq 10^6$, $x_i \neq P$) -- pozíciu výmoľu a koeficienty množstva
asfaltu. Žiadne dva body nezdieľajú pozíciu.

\heading{Výstup}

Na výstup vypíšte jedno číslo -- najmenšie množstvo asfaltu potrebné na opravenie všetkých výmoľov.

\heading{Príklad}


\sampleIN
3 7
10 32 1
3 5 1
14 0 2
\sampleOUT
72
\sampleCOMMENT
Optimálny spôsob ako plátať diery je nasledovný:
\begin{itemize}
	\item Presuň stroj z pozície $7$ na pozíciu $10$ a oprav prvú dieru po tretej sekunde.
	\item Presuň stroj z pozície $10$ na pozíciu $14$ a oprav druhý bod po čase $7$ sekúnd.
	\item Presuň stroj z pozície $14$ na pozíciu $3$ a oprav tretí bod po čase $18$ sekúnd.
\end{itemize}
Celkové množstvo asfaltu, ktoré je potrebné na opravenie všetkých dier je teda $5 + 18 \cdot 1 + 32 + 3 \cdot 1 + 0 + 7 \cdot 2 = 72$.
\sampleEND


\end{document}
