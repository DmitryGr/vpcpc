\input sys/inputs.tex

\begin{document}

\bigheading{Cubic Art}

% \info{task_name}{infile}{outfile}{points}{timelimit}{memlimit}
% leave this values, if you are not interested
\info{rubik}{stdin}{stdout}{100}{1200 ms}{1 GB}

Moderné umenie je nepredvídateľné.
Svoje o tom vie aj Bob.
Minule si len upratoval izbu
a našiel starú Rubikovu kocku.
Zatvoril oči, vnímal svoj vnútorný hlas
a spravil zopár\footnote{$65000$} ťahov.
Vzniklo takmer dokonalé dielo.
Potrebovalo už len zopár\footnote{opäť $65000$} úprav.
Presnejšie, potrebovalo zmeniť niektoré z pôvodných ťahov.
Bob by však rád vedel čo tieto zmeny robia.
A opakovať celú postupnosť zakaždým odznova je pracné.
Preto to treba skódiť.
Ale kto by to už len kódil?

\heading{Úloha}

Dostanete Rubikovu kocku, postupnosť
Bobových ťahov a postupnosť zmien týchto ťahov.
Po každej zmene vypíšte, ako vyzerá výsledná kocka.

\heading{Formát kocky}

Farby kocky označme $A$,$B$,$C$,$D$,$E$,$F$.
Keď sa hráte s kockou nemení
sa vzájomná poloha stredných štvorčekov.
Teda vieme kocku vždy zrotovať tak,
aby stredné štovrčeky mali fixné pozície.
Potom môžeme rozbaliť povrch kocky a získame
jednoznačný popis kocky:\\
\texttt{
???\\
?A?\\
???\\
????????????\\
?B??C??D??E?\\
????????????\\
???\\
?F?\\
???\\
}

\heading{Vstup}

Prvých 9 riadkov popisuje kocku na začiatku.

Nasledujúci riadok obsahuje 2 celé čísla $n$ a $m$.
$n$ je počet ťahov a $m$ je počet zmien.

Ďalších $n$ riadkov popisuje pôvodné Bobove ťahy.
Majú tvar $C_i$ $d_i$,
pričom $C_i$ je stred strany kocky ktorou točíme
a $d_i$ je $-1$ ak točíme v smere a 
$1$ ak točíme proti smeru hodinových ručiek.

Posledných $m$ riadkov popisuje zmeny.
Majú tvar $a_j$ $C_j$ $d_j$,
kde $a_j$ je číslo ťahu, ktorý sa nahradzuje
a $C_j$ $d_j$ popisujú nový ťah
Ťahy sú číslované $1$ až $n$.

\smallskip

Platí $n,m \leq 65000$
a v $50\%$ vstupov dokonca
$n,m \leq 1000$.

\heading{Výstup}

Vypíšte 9 riadkov popisujúce konečný tvar kocky
postupne po každej z $m$ zmien.

\heading{Príklad}

\sampleIN
AAA
AAA
AAA
BBBCCCDDDEEE
BBBCCCDDDEEE
BBBCCCDDDEEE
FFF
FFF
FFF
8 4
E 1
E -1
F 1
F -1
B 1
B -1
E 1
E -1
8 C -1
2 C -1
6 D -1
4 A -1
\sampleCOMMENT
Pôvodné ťahy sa navzájom vyrušia.
Postupne ich meníme na známy trik
rotácie stredov.
\sampleOUT
BAB
BAB
BAB
FBFCCCADAEEE
FBFCCCADAEEE
FBFCCCADAEEE
DFD
DFD
DFD
FAF
FAF
FAF
DBDCCCBDBEEE
DBDCCCBDBEEE
DBDCCCBDBEEE
AFA
AFA
AFA
FCF
BAB
FCF
EFEDFDCACBAB
DBDCCCBDBEEE
EFEDFDCACBAB
AEA
DFD
AEA
CCC
CAC
CCC
FFFDDDAAABBB
FBFDCDADABEB
FFFDDDAAABBB
EEE
EFE
EEE
\sampleEND


\end{document}
