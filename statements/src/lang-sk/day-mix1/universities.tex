\input sys/inputs.tex

\begin{document}

\bigheading{Univerzity}

% \info{task_name}{infile}{outfile}{points}{timelimit}{memlimit}
% leave this values, if you are not interested
\info{universities}{stdin}{stdout}{100}{1000 ms}{1 GB}

Keď Luxusko úspešne zmaturoval, čakalo ho neľahké rozhodnutie -- výber vhodnej vysokej školy.
Povedal si teda, že sa poobzerá po všetkých zaujímavých vysokých školách. Takých je na Slovensku
práve $N$ a rozlišujú sa podľa toho, či ich absolventi vychádzajú zo školy chlpatejší alebo
plešatejší. O každej univerzite teda povieme, že je chlpatá alebo plešatá. Naviac, pre zvláštnu
topológiu slovenských miest, sú tieto univerzity pospájané pomocou $N-1$ obojsmerných ciest. Dokonca
je zaručené, že medzi každou dvojicou univerzít sa dá dostať postupnosťou ciest (všimnite si, že
táto postupnosť je jednoznačná).

Rozhodol sa, že si naplánuje výlet naprieč niektorými univerzitami. Každá univerzita má udanú
konštantu šťastia, ktorá hovorí, o koľko sa zvýši Luxuskove šťastie, keď navštívi danú univerzitu. V
prípade, že je táto hodnota záporná, Luxusko zosmutnie.

Plán výletu vyzerá nasledovne. Vyberie si dve univerzity $x$ a $y$ a následne navštívi všetky
univerzity na ceste z univerzity $x$ do univerzity $y$ vrátane $x$ a $y$. A aby si zachoval svoje
krásne vlasy aj slizkú hruď, musí navštíviť rovnako veľa chlpatých ako plešatých univerzít, aby
zostal vyvážený.

Posledná otázka je, ako naplánovať výlet aby maximalizoval získané šťastie, ktoré z neho bude mať.
To sa vypočíta ako súčet konštát šťastia pre navštívené univerzity.

\heading{Úloha}

Na vstupe dostanete popis univerzít a ciest, ktoré ich spájajú. Nájdite optimálny plán výletu, pri
ktorom navštívi rovnaký počet chlpatých ako plešatých univerzít a celkové šťastie z výletu bude
maximálne.

\heading{Vstup}

Na vstupe sú štyri riadky.

Prvý riadok obsahuje celé číslo $N$ ($2 \leq N \leq 10^5$) -- počet univerzít. Tieto uviverzity
číslujeme od $1$ po $N$.

Druhý riadok obsahuje reťazec písmen dĺžky $N$, ktorý pozostáva z písmen \texttt{B} a \texttt{W}.
$i$-ty znak je \texttt{B}, ak je $i$-ta univerzita plešatá a \texttt{W} ak je chlpatá. Môžete
predpokladať, že na vstupe je aspoň jedna univerzita každého typu.

Tretí riadok obsahuje $N$, medzerou oddelených čísiel $h_i$ ($-10^5 \leq h_i \leq 10^5$) --
konštanta šťastia $i$-tej univerzity.

Na poslednom riadku je $N-1$ medzerou oddelených čísiel $v_i$ ($1 \leq v_i \leq i$), kde číslo $v_i$ znamená, že medzi
univerzitami $v_i$ a $i+1$ vedie cesta.

\heading{Výstup}

Vypíšte jedno číslo, maximálne možné šťastie, ktoré vie Luxusko získať na výlete, ak navštívi
rovnako veľa chlpatých a plešatých univerzíť.

\heading{Príklady}


\sampleIN
6
BWBBBW
6 0 3 -2 100 5
1 2 2 4 4
\sampleOUT
9
\sampleCOMMENT
Optimálna cesta vyzerá: $1,2,4,6$.
\sampleEND

\bigskip

\sampleIN
3
WBW
1 -10 5
1 2
\sampleOUT
-5
\sampleCOMMENT
Luxusko musí navštíviť aspoň dve univerzity a odpoveď teda môže byť aj záporná.
\sampleEND

\end{document}
