\input sys/inputs.tex

\begin{document}

\bigheading{Dynamická alokácia pamäte}

% \info{task_name}{infile}{outfile}{points}{timelimit}{memlimit}
% leave this values, if you are not interested
\info{malloc}{stdin}{stdout}{100}{FIXME}{1 GB}

Kamila sa zase raz snaží vytvoriť úžasný programovací jazyk, ktorý pošle všetky
predchádzajúce jazyky do zabudnutia. Programovanie sa stane tak jednoduchým, že
malé deti sa naučia kódiť skôr než budú schopné písať komentáre. Toto v žiadnom
prípade nie je bug,%
\footnote{Mimochodom, programy v novom jazyku budú zaručene bez chýb -- Kamilin
brilantný kompilátor všetky bugy odoptimalizuje.}
keďže komentáre už nikto potrebovať nebude: Kamila navrhla syntax svojho jazyka
tak šikovne, že každému bude význam ktorejkoľvek časti kódu už na prvý pohľad
zrejmá. Navyše, v deň vašich narodenín vám kompilátor napečie tortu.

Treba ale priznať, že niekoľko detailov momentálne nie je úplne dotiahnutých do
konca. Napríklad systém na správu pamäte stále nefunguje. Kamila síce pripravila
úplne skvelú špecifikáciu, ako sa bude pamäť alokovať a uvoľňovať, no nestihla
ju ešte implementovať. Pomôžte jej a získajte tak večnú slávu -- príležitosť
pracovať na takomto významnom projekte sa vám núka len raz za život!


\heading{Úloha}

Dostupnú pamäť si môžete predstaviť ako pole $n$ bajtov, ktoré sú očíslované od
$0$ do $n-1$. Na začiatku sú všetky bajty voľné (čiže nealokované). Následne
systém na správu pamäte spracuje niekoľko požiadaviek a príslušne alokuje a
uvoľní niektoré bajty.

Požiadavka na \emph{alokáciu} obsahuje celé číslo $\ell$. Systém nájde súvislý
blok $\ell$ voľných bajtov, alokuje ich a vráti pozíciu prvého bajtu v tomto
bloku. Ak by takýchto blokov bolo viacero, systém zvolí ten, ktorý začína na
pozícii s najmenším číslom. Naopak, ak žiadny vhodný blok neexistuje, systém
požiadavku odmietne a vráti $-1$.

Požiadavka na \emph{uvoľnenie} pozostáva z dvoch celých čísel $x$ a $\ell$.
Systém označí súvislý blok $\ell$ bajtov začínajúci na pozícii $x$ ako voľný a
vráti počet skutočne uvoľnených bajtov (čiže počet takých bajtov v tomto bloku,
ktoré pred vykonaním požiadavky neboli voľné).


\heading{Vstup}

Prvý riadok vstupu pozostáva z dvoch celých čísel $n$ a $q$ -- počtu bajtov v
dostupnej pamäti a počtu požiadaviek na systém ($1 \leq n, q \leq 3\cdot 10^5$).

Každý z nasledujúcich $q$ riadkov popisuje jednu požiadavku. Prvé číslo v riadku
určuje typ požiadavky: $1$ znamená alokáciu a $2$ uvoľnenie. Riadok s
požiadavkou na alokáciu pokračuje jedným celým číslom $\ell$ ($1 \leq \ell \leq
n$). Riadok s požiadavkou na uvoľnenie pokračuje dvoma celými číslami $x$ a
$\ell$ ($0 \leq x \leq n-1$, $1 \leq \ell \leq n-x$).


\heading{Výstup}

Pre každú požiadavku zo vstupu vypíšte jeden riadok s hodnotou vrátenou
systémom.


\heading{Príklad}

\sampleIN
5 4
1 3
1 3
2 1 3
1 4
\sampleOUT
0
-1
2
1
\sampleEND


\end{document}
