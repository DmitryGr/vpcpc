\input sys/inputs.tex

\begin{document}

\bigheading{Hra}

% \info{task_name}{infile}{outfile}{points}{timelimit}{memlimit}
% leave this values, if you are not interested
\info{game}{stdin}{stdout}{100}{100 ms}{1 GB}

% Mirek = Syslík
% Kamil = Žabka

Syslík a Žabka sa radi hrajú s číslami. Práve teraz frčia na takejto hre: Na
začiatku si určia dve kladné celé čísla $A$ a $B$. Potom sa striedajú v ťahoch,
začína Syslík.

Povedzme, že $A \leq B$. Potom má hráč na ťahu na výber z nasledujúcich možných
ťahov:
\begin{itemize}
\item
Nahraď $B$ hodnotou $B - A^K$ pre nejaké celé číslo $K$. Číslo $K$ musí spĺňať
$K > 0$ a $B - A^K \geq 0$.
\item
Nahraď $B$ hodnotou $B \bmod A$.
\end{itemize}
Ak platí $A \geq B$, potom má hráč na výber symetrické ťahy. Hru vyhrá ten hráč,
ktorý svojím ťahom zmení niektoré z čísel $A$, $B$ na $0$.


\heading{Úloha}

Dostanete zadané počiatočné hodnoty $A$, $B$ pre niekoľko hier. Pre každú hru
rozhodnite, kto ju vyhrá (Syslík alebo Žabka), ak obaja hráči hrajú optimálne s
cieľom vyhrať.


\heading{Vstup}

Prvý riadok vstupu obsahuje celé číslo $T$ ($1 \leq T \leq 10^4$) -- počet hier.
Nasledujúcich $T$ riadkov obsahuje popisy počiatočných stavov týchto hier. Každý
riadok pozostáva z dvoch celých čísel $A$, $B$ ($1 \leq A, B \leq 10^{18}$).

V $30\%$ vstupných sád bude navyše platiť $A, B \leq 1\,000$.


\heading{Výstup}

Vypíšte $T$ riadkov; $i$-ty riadok má obsahovať slovo \texttt{Mirek} ak $i$-tu
hru vyhrá Syslík, resp.\ \texttt{Kamil} ak ju vyhrá Žabka.


\heading{Príklad}

\sampleIN
4
1 1
12 4
4 6
15 31
\sampleOUT
Mirek
Mirek
Kamil
Mirek
\sampleEND

\end{document}
