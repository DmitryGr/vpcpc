\input sys/inputs.tex

\begin{document}

\bigheading{Roznášač}

% \info{task_name}{infile}{outfile}{points}{timelimit}{memlimit}
% leave this values, if you are not interested
\info{newspapers}{stdin}{stdout}{100}{4000 ms}{1 GB}

Hoci je Lamač slovenskou metropolou, jeho $n$ križovatiek je pospájaných len
$n-1$ obojsmernými ulicami. Mesto za to vďačí svojmu predvídavému starostovi,
ktorý takýmto opatrením podporuje turizmus: Medzi každými dvoma križovatkami
existuje práve jedna cesta, takže v meste sa nedá stratiť!

Hermi roznáša v Lamači noviny. Pre každú ulicu pozná počet jej obyvateľov a teda
aj počet výtlačkov, ktoré na nej roznesie. Na začiatku dňa si Hermi vyberie dve
križovatky a potom prejde všetkými ulicami na ceste medzi nimi. Noviny roznesie
práve obyvateľom ulíc, cez ktoré prešiel. Jeho zárobok za tento deň sa rovná
priemernému počtu výtlačkov doručených na jednej ulici (t.j.\ celkový počet
roznesených výtlačkov vydelený počtom prejdených ulíc).

Hermi hneď pochopil, že najvýhodnejšie pre neho je roznášať noviny len na ulici
s najväčším počtom obyvateľov. Jeho šéfovi sa ale tento prístup nepozdával --
príliš veľa ľudí sa sťažovalo, že im noviny nechodia. Preto šéf zakázal Hermimu
vyberať si také cesty, ktoré majú menej ako $k$ ulíc.

Pomôžte Hermimu a nájdite cestu, ktorá maximalizuje jeho zárobok.


\heading{Úloha}

Dostanete zadaný popis ulíc v Lamači a číslo $k$. Nájdite cestu s aspoň $k$
ulicami, ktorá má najväčší počet doručených výtlačkov na ulicu.


\heading{Vstup}

V prvom riadku vstupu sú dve celé čísla $n$ a $k$ -- počet križovatiek v Lamači
a minimálna dĺžka Hermiho cesty ($1 \leq n \leq 50\,000$, $1 \leq k \leq n -
1$). Nasledujúcich $n-1$ riadkov obsahuje popisy ulíc. Každý popis pozostáva z
troch celých čísel $a$, $b$ a $c$ ($1 \leq a < b \leq n$, $0 \leq c \leq 10^6$)
s nasledovným významom: Existuje ulica spájajúca križovatky $a$ a $b$, pričom na
tejto ulici býva $c$ ľudí.

Môžete predpokladať, že v Lamači existuje nejaká cesta dĺžky aspoň $k$.

V testovacích sadách ohodnotených $30$ bodmi navyše platí $n \leq 1\,000$.


\heading{Výstup}

Vypíšte jedno číslo -- priemerný počet doručených výtlačkov na jednej ulici v
optimálnej ceste. Výstup sa bude považovať za správny, ak rozdiel medzi ním a
správnou odpoveďou bude menší ako $10^{-6}$.


\heading{Príklad}

\sampleIN
5 2
1 2 4
2 3 1
3 4 3
3 5 3
\sampleOUT
3.00000000
\sampleEND

\end{document}
