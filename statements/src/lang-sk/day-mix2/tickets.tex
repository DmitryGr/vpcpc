\input sys/inputs.tex

\begin{document}

\bigheading{The International 4}

% \info{task_name}{infile}{outfile}{points}{timelimit}{memlimit}
% leave this values, if you are not interested
\info{tickets}{stdin}{stdout}{100}{800 ms}{1 GB}

Tento rok sa už po štvrtýkrát koná obrovský celosvetový turnaj v DOTA 2, ktorý sa volá The
International (TI4). Uskutoční sa 18.-21. júla v Seattle v Amerike. A pre veľký záujem fanúšikov o
miesta do hľadiska, rieši teraz Valve nasledovný problém.

Veľká hala, kde sa bude turnaj konať, má $n$ miest na sedenie. Tie sú očíslované od $1$ po $n$. Každý
fanúšik, ktorý si chcel kúpiť lístok, zadal do systému dve čísla $f$ a $l$, ktoré hovoria o
prvom a poslednom mieste, kde je ochotný sedieť. To znamená, že ak mu bude pridelené sedadlo $s$,
pre ktoré platí, že $f \leq s \leq l$, tak bude spokojný, v opačnom prípade sa na to vykašle a začne
hrať LoL. Valve sa teraz snaží vyhovieť čo najviac fanúšikom a usadiť ich na miesta, ktoré im
vyhovujú. Pokúste sa im pomôcť.

\heading{Úloha}

Na vstupe dostanete požiadavky všetkých fanúšikov. Zistite maximálny počet fanúšikov, ktorý môžu
dostať im vyhovujúce sedadlo a takisto nájdite jedno dobré priradenie miest fanúšikom. Samozrejme,
dvaja fanúšikovia nemôžu sedieť na jednom sedadle\footnote{Ani keby veľmi chceli.}.

\heading{Vstup}

Prvý riadok obsahuje dve čísla $n$ ($1 \leq n \leq 100\,000$) a $m$ ($1 \leq m \leq 1\,000\,000$)
-- počet miest na sedenie a počet fanúšikov, ktorý si chcú kúpiť lístok\footnote{Nebuďte prekvapení.
Záujem je naozaj väčší ako kapacita haly.}. Každý z nasledujúcich $m$ riadkov obsahuje dve čísla $f$
a $l$ ($1 \leq f \leq l \leq n$), ktoré popisujú požiadavky jednotlivých zákazníkov. Fanúšikovia sú
číslovaný od $1$ po $m$ v poradí, v akom sa ich požiadavky vyskytli na vstupe.

\heading{Výstup}

Na prvý riadok výstupu vypíšte jedno číslo $k$ -- maximálny počet fanúšikov, ktorým vieme priradiť
sedadlo.

Nasleduje $k$ riadkov, každý obsahuje dve čísla $s$ a $r$, kde $s$ je číslo sedadla a $r$ je číslo
fanúšika, ktorému bude toto sedadlo priradené. Táto časť výstupu môže byť v ľubovoľnom poradí a ak
existuje viacero priradení, stačí vypísať ľubovoľné z nich.

\heading{Príklad}
\sampleIN
10 9
1 3
2 4
5 7
2 6
1 5
3 7
4 8
7 9
3 8
\sampleOUT
9
1 1
2 5
3 2
4 4
5 6
6 9
7 3
8 7
9 8
\sampleCOMMENT

\sampleEND
\bigskip

\end{document}
