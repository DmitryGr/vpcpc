\input sys/inputs.tex

\begin{document}

\bigheading{Slalomár-začiatočník}

% \info{task_name}{infile}{outfile}{points}{timelimit}{memlimit}
% leave this values, if you are not interested
\info{slalom}{stdin}{stdout}{100}{300 ms}{1 GB}

Držte všetci Hubertovi prsty -- dnes po prvýkrát štartuje na pretekoch v
slalome! Horšie je, že dnes taktiež po prvýkrát stojí na lyžiach a ešte sa
nenaučil zatáčať. Keď sa spustí dole kopcom, pokračuje po priamke, až kým
neskončí v plote. V slalome to zvyčajne predstavuje neprekonateľnú nevýhodu, no
Hubert sa nevzdáva: Možno bola táto trať navrhnutá neporiadne a v skutočnosti sa
dá prejsť bez jedinej zatáčky.


\heading{Úloha}

Dostanete zadaný popis trate slalomu s $n$ bránkami. Trať prebieha zľava
doprava. Každá bránka je reprezentovaná zvislou úsečkou medzi jej dvoma žŕdkami.
Hubert vyzerá pri pohľade zhora ako kruh s priemerom $d$ ($d \geq 0$) a
trajektória jeho stredu kopíruje priamku. Svoj začiatočný bod si môže Hubert
zvoliť kdekoľvek naľavo od najľavejšej bránky a svoj konečný bod kdekoľvek
napravo od najpravejšej bránky. Na úspešné zvládnutie trate potrebuje prejsť
celým svojím telom pomedzi žŕdky každej bránky. Dotknúť sa žŕdky je povolené.

Nájdite najväčší priemer $d$ taký, že existuje trajektória, po ktorej Hubert
úspešne zvládne trať.


\heading{Vstup}

Prvý riadok vstupu obsahuje jedno celé číslo $n$ -- počet bránok ($1 \leq n
\leq 10^5$). Každý z nasledujúcich $n$ riadkov popisuje jednu bránku a pozostáva
z troch celých čísel $x, y_1, y_2$ oddelených medzerami ($0 \leq x \leq 10^9$,
$0 \leq y_1 \leq y_2 \leq 10^9$). Popísaná bránka je zvislá úsečka s koncovými
bodmi $[x, y_1]$ a $[x, y_2]$. Žiadne dve bránky nemajú rovnakú $x$-ovú
súradnicu.


\heading{Výstup}

Vypíšte jediný riadok obsahujúci najväčšiu hodnotu $d$, pre ktorú Hubert môže
úspešne zvládnuť trať. Budeme akceptovať odpovede s absolútnou alebo relatívnou
chybou menšou ako $10^{-9}$. V C++, svoju odpoveď môžete vypísať príkazom
\verb!printf("%.10lf\n", d);!

Ak neexistuje žiadna vyhovujúca nezáporná hodnota $d$, vypíšte jediný riadok so
slovom \texttt{Impossible}.


\heading{Príklady}

\sampleIN
3
4 3 7
6 6 9
1 5 10
\sampleOUT
1.3728129460
\sampleEND

\sampleIN
2
3 7 9
10 4 4
\sampleOUT
0.0000000000
\sampleEND

\sampleIN
3
0 4 7
2 0 3
4 4 7
\sampleOUT
Impossible
\sampleEND


\end{document}
