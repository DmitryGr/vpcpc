\input sys/inputs.tex

\begin{document}

\bigheading{Olíviina misia}

% \info{task_name}{infile}{outfile}{points}{timelimit}{memlimit}
% leave this values, if you are not interested
\info{mission}{stdin}{stdout}{100}{1000 ms}{1 GB}

Olívia sa naďalej snaží dosiahnuť svoj cieľ -- stať sa všeumelcom schopným vykonávať ľubovoľné
povolanie. Po postúpení na vyššiu hodnosť v Kulte Skrytého Poznania pracuje ako terénny agent.
Väčšinou je v zahraničí, kam jej posielajú samoničiace sa správy popisujúce misiu.

A práve dnes jej prišla takáto správa:

\bigskip

``Agent Delfínik!''

``V novej budove Federálnej Inštitúcie Idealizovaných Tubasov sa nachádza dokument, ktorý by mohol ohroziť
celú našu organizáciu. Vašou úlohou bude preniknúť do vnútra budovy, lokalizovať daný dokument a
zničiť ho.''

``Ak sa vám naviac podarí úspešne uniknúť, pri neďalekej rieke vás bude čakať vrtuľník, ktorý vás
odnesie na našu základňu.''

``Veľa šťastia.''

``Táto správa sa zničí o $5$ sekúnd.''

\bigskip

Olívia rýchlo odhodila diktafón s nahrávkou, vybrala mapu a začala plánovať trasu, ktorou vnikne do
budovy a taktiež trasu, ktorou z nej unikne. Samozrejme chce, aby táto trasa bola dokopy čo
najkratšia. A takisto nemôže prechádzať cez to isté miesto dvakrát, lebo by to bolo príliš
nebezpečné. Viete jej pomôcť nájsť správnu cestu?

\heading{Úloha}

Na vstupe dostanete neorientovaný ohodnotený graf a tri rôzne vrcholy: miesto kde sa nachádza
Olívia, budovu FIITky a miesto, kde bude čakať helikoptéra. Nájdite najkratšiu cestu od Olívii k
helikoptére, aby táto cesta prechádzala cez FIITku a nešla cez žiaden vrchol viackrát.

\heading{Vstup}

Na prvom riadku je päť celých čísiel $n$, $m$, $b$, $e$ a $h$.
($1 \le b, e, h \le n$, $b \neq e \neq h \neq b$)

Graf obsahuje $n$ vrcholov očíslovaných $1$ až $n$ a $m$ hrán. Čísla $b$, $e$ a $h$ sú čísla
vrcholov zodpovedajúcich pozícii Olívie, FIITky a helikoptéry.

Nasleduje $m$ hrán popisujúcich hrany grafu. Každá hrana je popísaná trojicou čísiel $u$, $w$ a $d$
($1 \leq u,v \leq n$, $1 \leq d \leq 1\,000\,000$) -- začiatok a koniec hrany a dĺžku príslušnej
hrany.

Graf neobsahuje násobné hrany a slučky.

\smallskip
Platí, že $3 \leq n \leq 1\,000$ a $0 \leq m \leq 1\,000$.

V $30\%$ vstupoch naviac platí, že $n\leq 20$.

\heading{Výstup}

Vypíšte jeden riadok s číslom, ktoré zodpovedá dĺžke najkratšej cesty od Olívii k helikoptére cez
nepriateľskú budovu.

Ak takáto cesta neexistuje, vypíšte $-1$.

\heading{Príklady}


\sampleIN
3 2 1 2 3
1 2 10
2 3 20
\sampleOUT
30
\sampleEND


\bigskip


\sampleIN
3 0 2 1 3
\sampleOUT
-1
\sampleEND


\bigskip


\sampleIN
4 4 3 2 4
2 3 5
3 1 1
1 4 1
2 4 100
\sampleOUT
105
\sampleEND

\end{document}
