\input sys/inputs.tex

\begin{document}

\bigheading{O hrozienkach}

% \info{task_name}{infile}{outfile}{points}{timelimit}{memlimit}
% leave this values, if you are not interested
\info{cutting}{text files}{text files}{100}{-}{-}

Mimino má narodeniny!
Na to aký úžasný programátor je, má iba $N$
rokov.\footnote{Miminov názor na to, čo je to iba $N$
by nemal ovplyvniť vaše očakávania na optimálne riešenie}
Kamilin kompilátor (ten zovčera) mu pripravil obrovský\footnote{tak veľký, že si ho môžete predstaviť ako nekonečnú rovinu}
čokoládový koláč a položil naň $N$ hrozienok\footnote{sviečky nechutia ako koník}.
Mimino s kamarátmi sú už hladní a keďže kamarátov je viac ako hrozienok,
treba koláč rozdeliť na kúsky s najviac jedným hrozienkom.
Rezať obrovský koláč je však namáhavé a teda to treba spraviť
na čo najmenej rezov.

\heading{Úloha}

Máme rovinu z $N$ bodmi. Rozdeľte ju čo najmenším počtom priamok na oblasti obsahujúce najviac jeden bod.

Súradnice bodov sú celé čísla. Priamky sú zadané pomocou dvoch celočíselných súradníc.
Priamky sú orientované z prvého bodu do druhého.
Orientáca sa používa na určenie do ktorej časti patrí
vstupný bod, ak ho priamka pretína --
patrí do časti napravo od priamky.

Priamky môžu byť horizontálne, vertikálne alebo diagonálne.

\heading{Vstup}

Prvý riadok obsahuje jedno celé číslo $N$,
počet bodov.
Každý z nasledujúcich $N$ riadkov obsahuje
2 medzerou oddelené celočíslené súradnice.

\heading{Výstup}

Prvý riadok obsahuje celé číslo $L$, počet priamok.
Nasledujúcich $L$ riadkov popisuje priamky.
$(i+1)$-ty riadok pozostáva zo štyroch medzerou
oddelených celých čísel $X_1$, $Y_1$, $X_2$, $Y_2$.
Body ($X_1$, $Y_1$) a ($X_2$, $Y_2$) musia byť rôzne a
navyše musí byť splnená aspoň jedna z nasledujúcich podmienok:

\begin{itemize}
  \item $X_1 = X_2$ (vertikálna priamka)
  \item $Y_1 = Y_2$ (horizontálna priamka)
  \item $X_1 - X_2 = Y_1 - Y_2$
  \item $X_1 - X_2 = Y_2 - Y_1$
\end{itemize}

Môžete vypísať najviac $10\,000$ priamok a absolútne hodnoty súradníc môžu byť najviac $10\,000$.

\heading{Hodnotenie}
Toto je opendata úloha. Z testovača si môžete stiahnuť všetkých $10$ vstupov.
Submitujete iba výstupy.

Ak vaše riešenie nedodrží formát alebo nerozdelí body správne, dostanete $0$ bodov.

V opačnom prípade dostanete za každý vstup
$ 10 \cdot \left(1 - \sqrt{1 - L_{min} / L}\right)$ bodov,
kde $L$ je počet priamok vo vašom riešení a $L_{min}$ je najlepšie riešenie zo všetkých súťažiacich.
Počet bodov sa vyhodnotí až na konci súťaže.

\heading{Príklady}

\sampleIN
4
3 1
4 5
6 6
8 4
\sampleOUT
2
3 2 8 7
2 8 8 2
\sampleEND

\end{document}
