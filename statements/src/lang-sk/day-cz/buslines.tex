\input sys/inputs.tex

\begin{document}

\bigheading{Autobusy}

% \info{task_name}{infile}{outfile}{points}{timelimit}{memlimit}
% leave this values, if you are not interested
\info{buslines}{stdin}{stdout}{100}{3000 ms}{1 GB}

V českom meste Kocourkov\footnote{narozdiel od nášho Kocúrkova je Kocourkov fiktívny}
majú zaujímavý systém verejnej dopravy.
Pozostáva z $N$ zastávok pospájaných pomocou $N-1$ obojsmerných ciest tak,
že z ľubovoľnej zastávky sa možno dostať na ľubovoľnú inú,
použitím nejakej postupnosti ciest.

Každej dvojici rôznych zastávok $x$ a $y$ je pridelený autobus, ktorý vždy ráno vyrazí z počiatočnej
zastávky $x$ do konečnej zastávky $y$, pričom ide po jedinej možnej ceste, ktorá spája tieto dve
zastávky. Dokopy teda vyrazí $N(N-1)$ autobusov. Každý autobus naviac zastaví na všetkých
zastávkach, cez ktoré prechádza počas svojej cesty.

V Kocourkove sa koná budúcoročné VPCPC\footnote{ignorujte predošlú poznámku}
a teda zháňame miesto na ubytovanie.
Samozrejme, musí byť v blízkosti zastávky, lebo sa nikomu nechce šlapať.
Ale nepáči sa nám, ak tam bude jazdiť priveľa autobusov, lebo by vás rušili počas contestu.
A preto nám pre každú zastávku zistite, koľko autobusov na nej stojí.

\heading{Úloha}

Dostanete popis systému verenej dopravy v Kocourkove.
Pre každú zastávku zistite, koľko autubusov na nej stojí.

\heading{Vstup}

Prvý riadok obsahuje celé číslo $N$,
počet zastávok v meste.
Zastávky sú číslované od $1$ po $N$.
Nasledujúcich $N-1$ riadkov popisuje cesty v meste.
Každý riadok obsahuje $2$ celé čísla $1 \le x, y \le N$
-- čísla zastávok, ktoré spája táto cesta.

\bigskip
\noindent
Platí $1 \leq N \leq 10^6$.\\
V $20\%$ vstupov platí $N \leq 100$.\\
V $40\%$ vstupov platí $N \leq 1000$.

\heading{Výstup}

Výstup obsahuje $N$ riadkov.
V $i$-tom riadku vypíšte jedno celé číslo --
počet autobusov, ktoré stoja na $i$-tej zastávke.

\heading{Príklady}

\sampleIN
6
1 2
2 3
3 4
4 5
5 6
\sampleOUT
10
18
22
22
18
10
\sampleEND

\bigskip

\sampleIN
5
4 5
2 1
3 2
2 5
\sampleOUT
8
18
8
8
14
\sampleEND

\end{document}
