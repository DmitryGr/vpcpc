\input sys/inputs.tex

\begin{document}

\bigheading{Autobusy}

% \info{task_name}{infile}{outfile}{points}{timelimit}{memlimit}
% leave this values, if you are not interested
\info{buslines}{stdin}{stdout}{100}{1000 ms}{1 GB}

V českom meste Kocourkov\footnote{narozdiel od nášho Kocúrkova je Kocourkov fiktívny}
majú zaujímavý systém verejnej dopravy.
Pozostáva z $N$ zastávok pospájaných pomocou $N-1$ obojsmerných ciest tak,
že z ľubovoľnej zastávky sa možno dostať na ľubovoľnú inú
použitím nejakej postupnosti ciest.

Každé ráno, pre každú dvojicu rôznych zastávok $x$ a $y$,
vyrazí autobus z počiatočnej zastávky $x$ do konečnej zastávky $y$
po jedinej možnej ceste medzi týmito zastávkami.
Dokopy teda vyrazí $N(N-1)$ autobusov.
Každý autobus stojí na všetkých zastávkach
po ceste do konečnej.

V Kocourkove sa koná budúcoročné VPCPC\footnote{ignorujte predošlú poznámku}
a teda zháňame miesto na ubytovanie.
Samozrejme, musí byť v blízkosti zastávky.
(Komu sa chce šlapať?)
Ale nepáči sa nám ak tam bude jazdiť priveľa autobusov.
(Budú vás rušiť počas kontestu.)
A preto nám pre každú zastávku zistíte,
koľko autobusov na nej stojí.

\heading{Úloha}

Dostanete popis systému verenej dopravy v Kocourkove.
Pre každú zastávku zistite, koľko autubusov na nej stojí.

\heading{Vstup}

Prvý riadok obsahuje celé číslo $N$,
počet zastávok v meste.
Zastávky sú číslované od $1$ po $N$.
Nasledujúcich $N-1$ riadkov popisuje cesty v meste.
Každý riadok obsahuje 2 celé čísla $1 \le x, y \le N$
-- čísla zastávok, ktoré spája táto cesta.

\bigskip
\noindent
Platí $1 \leq N \leq 10^6$.\\
V $20\%$ vstupov platí $N \leq 100$.\\
V $40\%$ vstupov platí $N \leq 1000$.

\heading{Výstup}

Vypíšte $N$ riadkov.
V $i$-tom riadku vypíšte jedno celé číslo --
počet autobusov, ktoré stoja na $i$-tej zastávke.

\heading{Príklady}

\sampleIN
6
1 2
2 3
3 4
4 5
5 6
\sampleOUT
10
18
22
22
18
10
\sampleEND

\bigskip

\sampleIN
5
4 5
2 1
3 2
2 5
\sampleOUT
8
18
8
8
14
\sampleEND

\end{document}
